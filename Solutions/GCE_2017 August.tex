%GCE of WPI
%by Jiamin JIAN

\documentclass[12pt,a4paper]{ctexart}
\usepackage{CJK}
\usepackage{lipsum}
\usepackage{amsmath}
\usepackage{geometry}
\usepackage{titlesec}
\usepackage{amssymb}
\usepackage{epsfig}
\usepackage{float}
\usepackage{graphicx}
\usepackage{tabularx}
\usepackage{longtable}
\usepackage{amstext}
\usepackage{blkarray}
\usepackage{amsfonts}
\usepackage{bbm}
\usepackage{listings}
\geometry{left=2.5cm,right=2.5cm,top=2.5cm,bottom=2.5cm}

\begin{document}


\begin{center}
\textbf{ GCE August, 2017}
\vspace{8pt}

Jiamin JIAN
\end{center}

\vspace{12pt}

$\underline{\textbf{Exercise 1:}}$

Let $h_{n}$ be a sequence of non-negative, borel measurable functions on the interval $(0, 1)$ such that $h_{n} \rightarrow 0$ in $L^{1}((0, 1))$.

(i) Show $\sqrt{h_{n}} \rightarrow 0$ in $L^{1}((0, 1))$.

(ii) Given an example to show that $h_{n}^{2}$ need not converge to zero in $L^{1}((0, 1))$.

(iii) If $g_{n}$ is in $L^{1}(\mathbb{R})$ such that $|g_{n}|^{\frac{1}{2}}$ is in $L^{1}(\mathbb{R})$, and $g_{n}$ converges to zero in $L^{1}(\mathbb{R})$ as $n$ tends to infinity, does $|g_{n}|^{\frac{1}{2}}$ converges to zero in $L^{1}(\mathbb{R})$?

\vspace{8pt}

$\textbf{Solution:}$

(i) We want to show that $\int_{0}^{1} |\sqrt{h_{n}} - 0| \, d \mu \rightarrow 0$ as $n \rightarrow \infty$. Since $h_{n} \rightarrow 0$ in $L^{1}((0, 1))$ and by the H\"older inequality, we have
\begin{eqnarray*}
\int_{0}^{1} |\sqrt{h_{n}} - 0| \, d \mu & \leq & \Big{(} \int_{0}^{1} |(\sqrt{h_{n}})^{2}| \, d \mu \Big{)}^{\frac{1}{2}} \Big{(} \int_{0}^{1} 1^{2} \, d \mu \Big{)}^{\frac{1}{2}} \\
& = & \Big{(} \int_{0}^{1} h_{n} \, d \mu \Big{)}^{\frac{1}{2}} \Big{(} \int_{0}^{1} 1 \, d \mu \Big{)}^{\frac{1}{2}} \\
& = & \Big{(} \int_{0}^{1} |h_{n}-0| \, d \mu \Big{)}^{\frac{1}{2}}.
\end{eqnarray*}
So when $n$ goes to infinity, we have $\int_{0}^{1} |\sqrt{h_{n}} - 0| \, d \mu \rightarrow 0$. Thus we know that $\sqrt{h_{n}} \rightarrow 0$ in $L^{1}((0, 1))$.

(ii) For $n \in \mathbb{N}$, let
\begin{equation*}
   h_{n} (x) = n^{\frac{3}{2}} x \mathbb{I}_{[\frac{1}{n^{2}}, \frac{1}{n})} (x).
\end{equation*}
Then we have
\begin{equation*}
   \int_{0}^{1} n^{\frac{3}{2}} x \mathbb{I}_{[\frac{1}{n^{2}}, \frac{1}{n})} (x) \, d x  = n^{\frac{3}{2}} \int_{\frac{1}{n^{2}}}^{\frac{1}{n}} x \, d x  = \frac{1}{2} (\frac{1}{\sqrt{n}} - \frac{1}{n^{\frac{5}{2}}}),
\end{equation*}
when $n \to + \infty$, we get $\|h_{n} \|_{1} \to 0$, so we know that $h_{n} \to 0$ in $L^{1}((0, 1))$. But for the $h_{n}^{2}(x)$, we have
\begin{equation*}
   \int_{0}^{1} n^{3} x^{2} \mathbb{I}_{[\frac{1}{n^{2}}, \frac{1}{n})} (x) \, d x  = n^{3} \int_{\frac{1}{n^{2}}}^{\frac{1}{n}} x^{2} \, d x  = \frac{1}{3} n^{3} (\frac{1}{n^{3}} - \frac{1}{n^{6}}) = \frac{1}{3} - \frac{1}{3 n^{3}}.
\end{equation*}
When $n$ tends to infinity, $\int_{0}^{1} n^{3} x^{2} \mathbb{I}_{[\frac{1}{n^{2}}, \frac{1}{n})} (x) \, d x \rightarrow \frac{1}{3}$, which is not goes to $0$. So, we know that $h_{n}^{2}(x)$ don't converge to zero in $L^{1}((0, 1))$. 

The counter example on above is hard and not elegant. For all $n \in \mathbb N$, let $h_n(x) = n 1_{(0, 1/n^2)} $, then $h_{n}$ be a sequence of non-negative, borel measurable functions on the interval $(0, 1)$ and
$$\|h_n\|_1 = \int_{0}^{1} n 1_{(0, \frac{1}{n^2})} (x) \, d x = n \frac{1}{n^2} = \frac{1}{n} \to 0$$
as $n \to \infty$, thus $h_n \to 0$ in $L^1((0,1))$. But for each $n \in \mathbb N$,
$$\int_{0}^{1} h_n^2 (x) \, d x = \int_{0}^{1} n^2 1_{(0, \frac{1}{n^2})} \, d x = n^2 \frac{1}{n^2} = 1.$$
Therefore $h_n^2$ does not converges to $0$ in $L^1((0,1))$.

(iii) No, $|g_{n}|^{\frac{1}{2}}$ need not converge to zero in $L^{1}(\mathbb{R})$. We can give a counter example. Suppose $g_{n}(x) = \frac{1}{x^{2}} \mathbb{I}_{[n, n^{2}]} (x)$, then we have
\begin{equation*}
   \int_{\mathbb{R}}^{} |g_{n}(x)| \, d x  = \int_{n}^{n^{2}} \frac{1}{x^{2}} \, d x  = \frac{1}{n} - \frac{1}{n^{2}}.
\end{equation*}
When $n$ goes to infinity, we have $\|g_{n}(x) \|_{1} \to 0$, so $g_{n} (x)$ is in $L^{1}(\mathbb{R})$ and $g_{n}$ converges to zero in $L^{1}(\mathbb{R})$. For the $|g_{n}|^{\frac{1}{2}} = \frac{1}{x} \mathbb{I}_{[n, n^{2}]} (x)$, for any $n \in \mathbb{N}$ we have 
\begin{equation*}
   \int_{\mathbb{R}}^{} |g_{n}(x)|^{\frac{1}{2}} \, d x  = \int_{n}^{n^{2}} \frac{1}{x} \, d x  = \ln n.
\end{equation*}
When $n$ goes to infinity, we have $\int_{\mathbb{R}}^{} |g_{n}(x)|^{\frac{1}{2}} \, d x \to + \infty$, so $|g_{n}|^{\frac{1}{2}}$ is in $L^{1}(\mathbb{R})$ for each $n \in \mathbb N$, but $g_{n}$ don't converges to zero in $L^{1}(\mathbb{R})$.

Another counter example is as follows. For each $n \in \mathbb N$, let 
$$g_n (x) = \frac{1}{n^2} 1_{[0, n]},$$
then
$$\int_{\mathbb R} |g_n| = \int_{\mathbb R} \frac{1}{n^2} 1_{[0, n]}(x) \, d x = \frac{1}{n} \to 0 $$
as $n \to \infty$. Thus $g_n \in L^1(\mathbb R)$ and $g_n$ converges to $0$ in $L^1(\mathbb R)$ as $n $ goes to infinity. And note that
$$\int_{\mathbb R} |g_n|^{\frac{1}{2}} = \int_{\mathbb R} \frac{1}{n} 1_{[0,n]} (x) \, d x = 1 < \infty,$$
thus $|g_n|^{1/2} \in L^1(\mathbb R)$. But we have that $|g_n|^{1/2} $ does not converges to $0$ in $L^1(\mathbb R)$.
 

\newpage


$\underline{\textbf{Exercise 2:}}$

Let $f$ be in $L^{\infty} ((0, 1))$. Show that $\|f \|_{p} \rightarrow \|f \|_{\infty}$ as $p \rightarrow \infty$.

\vspace{8pt}
$\textbf{Solution:}$

Since $f \in L^{\infty} ((0,1))$ and $\mu((0, 1)) = 1 < \infty$, then for all $p \geq 1$,
$$\int_{(0,1)} |f|^p \, d \mu \geq \|f\|_{\infty}^p \mu((0,1)) < \infty,$$
thus $f \in L^{p}((0, 1))$. Let
$$A = \{x \in (0,1): f(x) > \|f\|_{\infty} - \epsilon \},$$
by the definition of $\|f\|_{\infty}$, we know that $\mu(A) > 0$. For all $p \in [1, \infty)$, since
\begin{eqnarray*}
\|f\|_{p} & = & \Big{(} \int_{(0, 1)}^{} |f|^{p} \, d \mu \Big{)}^{\frac{1}{p}} \geq \Big{(} \int_{A}^{} |f|^{p} \, d \mu \Big{)}^{\frac{1}{p}} \\
& \geq & \Big{(} (\|f\|_{\infty} - \epsilon)^{p} \mu(A)\Big{)}^{\frac{1}{p}} = (\|f\|_{\infty} - \epsilon) (\mu(A))^{\frac{1}{p}},
\end{eqnarray*}
and since $\mu(A) \leq 1$, we have
\begin{equation*}
   \liminf_{p \to + \infty} \|f\|_{p} \geq \liminf_{p \to + \infty} (\|f\|_{\infty} - \epsilon) (\mu(A))^{\frac{1}{p}} = \|f\|_{\infty} - \epsilon.
\end{equation*}
By the arbitrary of $\epsilon > 0$, we have
\begin{equation*}
   \liminf_{p \to + \infty} \|f\|_{p} \geq \|f \|_{\infty} .
\end{equation*}
On the other hand, as $|f(x)| \leq \|f\|_{\infty}$ for almost every $x \in (0, 1)$, then for $1 \leq q < p$, since $f(x)$ is in $L^{p}((0, 1))$ and $f(x)$ is in $L^{q}((0, 1))$, we have
\begin{eqnarray*}
\|f\|_{p} & = & \Big{(} \int_{(0, 1)}^{} |f|^{p} \, d \mu \Big{)}^{\frac{1}{p}} \\
& = & \Big{(} \int_{(0, 1)}^{} |f|^{q} |f|^{p - q} \, d \mu \Big{)}^{\frac{1}{p}} \\
& \leq & (\|f\|_{\infty})^{\frac{p - q}{p}} (\|f\|_{q})^{\frac{q}{p}}.
\end{eqnarray*}
Since $\|f\|_{q} < + \infty$, then when $p \to + \infty$, we know that
\begin{equation*}
   \limsup_{p \to + \infty} \|f\|_{p} \leq \|f \|_{\infty} .
\end{equation*}
We also can get $\limsup_{p \to + \infty} \|f\|_{p} \leq \|f \|_{\infty}$ directly as follows
\begin{eqnarray*}
\|f\|_{p} & = & \Big{(} \int_{(0, 1)}^{} |f|^{p} \, d \mu \Big{)}^{\frac{1}{p}} \\
& \leq  & \Big{(} \int_{(0, 1)}^{} \|f\|_{\infty}^p \, d \mu \Big{)}^{\frac{1}{p}} \\
& \leq & \|f\|_{\infty} (\mu((0,1)))^{\frac{1}{p}}.
\end{eqnarray*}
Thus we have
\begin{equation*}
   \limsup_{p \to + \infty} \|f\|_{p} \leq \|f \|_{\infty} \leq \liminf_{p \to + \infty} \|f\|_{p},
\end{equation*}
then we know that $\|f \|_{p} \rightarrow \|f \|_{\infty}$ as $p \rightarrow \infty$.


\newpage

$\underline{\textbf{Exercise 3:}}$

Let $a_{n}$ be a sequence in $[0, 1]$ such that the set $S = \{a_{n}: n = 1, 2, \dots \}$ is dense in $[0, 1]$. Set
\begin{equation*}
   f(x) = \sum_{n = 1}^{\infty} \frac{|x - a_{n}|^{- \frac{1}{2}}}{n^{2}}.
\end{equation*}

(i) Show that $f$ is in $L^{1}([0, 1])$.

(ii) Is $f$ in $L^{2} ([0, 1])$?

(iii) Is there a continuous function
\begin{equation*}
   g : [0, 1] \setminus S \rightarrow \mathbb{R}
\end{equation*}
such that $f = g$ almost everywhere?

\vspace{8pt}
$\textbf{Solution:}$

(i) We check $f \in L^{1}([0,1])$ by definition, since
\begin{eqnarray*}
\int_{0}^{1} \sum_{n = 1}^{\infty} \frac{|x - a_{n}|^{- \frac{1}{2}}}{n^{2}} \, d x  & = & \sum_{n = 1}^{\infty} \frac{1}{n^{2}} \int_{0}^{1} |x - a_{n}|^{- \frac{1}{2}} \, d x \\
& = & \sum_{n = 1}^{\infty} \frac{1}{n^{2}} \Big{[}  \int_{0}^{a_{n}} (a_{n} - x)^{- \frac{1}{2}} \, d x + \int_{a_{n}}^{1} (x - a_{n})^{- \frac{1}{2}} \, d x  \Big{]} \\
& = & \sum_{n = 1}^{\infty} \frac{1}{n^{2}} \Big{[} 2 (a_{n})^{\frac{1}{2}} + 2 (1 - a_{n})^{\frac{1}{2}}  \Big{]}
\end{eqnarray*}
and $a_{n} \in [0, 1]$, then we know that
\begin{equation*}
   \int_{0}^{1} \sum_{n = 1}^{\infty} \frac{|x - a_{n}|^{- \frac{1}{2}}}{n^{2}} \, d x \leq 4 \sum_{n = 1}^{\infty} \frac{1}{n^{2}} < + \infty
\end{equation*}
as $\sum_{n = 1}^{\infty} \frac{1}{n^{2}} = \frac{\pi^{2}}{6}$. Thus we know that $f \in L^{1}([0,1])$.

(ii) No, we can show that $f \notin L^{2}([0,1])$. For $x \in [0, 1]$, we have
\begin{eqnarray*}
\|f\|_{2} & =& \int_{0}^{1} \big{(} \sum_{n = 1}^{\infty} \frac{|x - a_{n}|^{- \frac{1}{2}}}{n^{2}}\big{)}^{2}  \, d x  \\
& \geq & \int_{0}^{1} \sum_{n = 1}^{\infty} \big{(} \frac{|x - a_{n}|^{- \frac{1}{2}}}{n^{2}}\big{)}^{2}  \, d x  \\
& = & \sum_{n = 1}^{\infty} \frac{1}{n^{4}} \int_{0}^{1} |x - a_{n}|^{-1} \, d x .
\end{eqnarray*}
To show $f \notin L^{2}([0,1])$, we just need to prove that $\int_{0}^{1} |x - a_{n}|^{-1} \, d x = + \infty$. We denote $y = x - a_{n}$, then we have
\begin{equation*}
   \int_{0}^{1} |x - a_{n}|^{-1} \, d x = \int_{-a_{n}}^{1 - a_{n}} |y|^{-1} \, d y.
\end{equation*}
Since there exists $k > 0$ such that $\frac{1}{k} < a_{n}$, then we have $- \frac{1}{k} < 0 < 1 - a_{n}$ and
\begin{equation*}
   \int_{0}^{1} |x - a_{n}|^{-1} \, d x \geq \int_{-a_{n}}^{- \frac{1}{k}} |y|^{-1} \, d y = \int_{\frac{1}{k}}^{a_{n}} y^{-1} \, d y = \ln a_{n} + \ln k.
\end{equation*}
When $k \to + \infty$, we have $\ln k+ \ln a_{n} \to \infty$. So, we know that $\int_{0}^{1} |x - a_{n}|^{-1} \, d x = + \infty$. Thus $\|f\|_{2} = + \infty$, then we have $f \notin L^{2}([0,1])$.


(iii) To show that there is a continuous function $g : [0, 1] \setminus S \rightarrow \mathbb{R}$ such that $f = g$ almost everywhere, we just need to prove that $f$ is continuous in $[0, 1] \setminus S$. So for $x \in [0, 1] \setminus S$, we want to show that: $\forall \epsilon > 0$, $\exists \delta > 0$ such that $\forall y \in [0, 1] \setminus S$ with $|x - y| < \delta$, we have $|f(x) - f(y)| < \epsilon$. Note that
\begin{eqnarray*}
|f(x) - f(y)| & =& \Big{|} \sum_{n = 1}^{\infty} \frac{|x - a_{n}|^{- \frac{1}{2}}}{n^{2}} - \sum_{n = 1}^{\infty} \frac{|y - a_{n}|^{- \frac{1}{2}}}{n^{2}} \Big{|} \\
& = &  \Big{|} \sum_{n = 1}^{\infty} \frac{1}{n^{2}} (|x - a_{n}|^{- \frac{1}{2}} - |y - a_{n}|^{- \frac{1}{2}}) \Big{|} \\
& \leq & \sum_{n = 1}^{\infty} \frac{1}{n^{2}}  \Big{|} |x - a_{n}|^{- \frac{1}{2}} - |y - a_{n}|^{- \frac{1}{2}} \Big{|}.
\end{eqnarray*}
Since $g(x) = |x - a_{n}|^{- \frac{1}{2}}$ is continuous on $(0, 1]$, then $\forall \epsilon > 0$, $\exists \delta > 0$ such that $\forall y \in (0, 1]$ with $|x - y| < \delta$, we have 
\begin{equation*}
   \Big{|} |x - a_{n}|^{- \frac{1}{2}} - |y - a_{n}|^{- \frac{1}{2}} \Big{|} < \frac{6}{\pi^{2}} \epsilon.
\end{equation*}
Since $S$ is countable and dense in $[0, 1]$, then for the above $\epsilon$ and $\delta$, $\forall y \in [0, 1] \setminus S$ with $|x - y| < \delta$, we have
\begin{equation*}
   |f(x) - f(y)| \leq \sum_{n = 1}^{\infty} \frac{1}{n^{2}}  \Big{|} |x - a_{n}|^{- \frac{1}{2}} - |y - a_{n}|^{- \frac{1}{2}} \Big{|} < \frac{\pi^{2}}{6} \times \frac{6}{\pi^{2}} \epsilon = \epsilon.
\end{equation*}
Thus we know that $f(x)$ is continuous in $[0, 1] \setminus S$, then $f(x)$ is continuous almost everywhere in $[0, 1]$. So, there exists a continuous function $g : [0, 1] \setminus S \rightarrow \mathbb{R}$ such that $f = g$ almost everywhere.


\newpage

$\underline{\textbf{Exercise 4:}}$

Let $\mathcal{R}$ be the set of all rectangles $(a_{1}, b_{1}) \times (a_{2}, b_{2})$ in $\mathbb{R}^{2}$ such that $a_{1}, b_{1}, a_{2}, b_{2}$ are rational numbers.

(i) Let $V$ be an open set in $\mathbb{R}^{2}$. Show that 
\begin{equation*}
   V = \bigcup_{R \in \mathcal{R}, R \subset V} R.
\end{equation*}

(ii) Recall that the Borel sets of $\mathbb{R}^{2}$ are the sets in the smallest sigma algebra of $\mathbb{R}^{2}$ containing all open sets. Show that the smallest sigma algebra of $\mathbb{R}^{2}$ containing $\mathcal{R}$ is equal to the set set of Borel sets of $\mathbb{R}^{2}$. 

\vspace{8pt}
$\textbf{Solution:}$

(i) Since $\bigcup_{R \in \mathcal{R}, R \subset V} R \subset V$, to prove $V = \bigcup_{R \in \mathcal{R}, R \subset V} R$, we just need to show that $V \subset \bigcup_{R \in \mathcal{R}, R \subset V} R$. Suppose that $\vec{x} = (x_{1}, x_{2}) \in V$, since $V$ is an open set, then there exists an open ball such that $B(\vec{x}, r) \subset V$, where $r$ is a positive constant and it is called the radius of the ball. So we can find a rectangle $ R = (a_{1}, b_{1}) \times (a_{2}, b_{2})$, whose center is exactly $\vec{x}$. We denote $d((a_{1}, b_{1}), (a_{2}, b_{2}))$ is the distance between $(a_{1}, b_{1})$ and $(a_{2}, b_{2})$. Suppose $d((a_{1}, b_{1}), (a_{2}, b_{2})) < r$, then when know that $\vec{x} \in R$, $R \subset V$ and $R \in \mathcal{R}$. For any $x \in V$ we can do same thing, so we have $V \subset \bigcup_{R \in \mathcal{R}, R \subset V} R$. Thus we know that $V = \bigcup_{R \in \mathcal{R}, R \subset V} R$.

(ii) We denote $\sigma(\mathcal{R})$ is the sigma algebra on $\mathbb{R}^{2}$ generated by sets in $\mathcal{R}$. And we denote $\mathcal{B}(\mathbb{R}^{2})$ as the Borel sets of $\mathbb{R}^{2}$. Since $R$ is open rectangle in $\mathbb{R}^{2}$ and $\mathcal{R} = \{(a_{1}, b_{1}) \times (a_{2}, b_{2}) | a_{i}, b_{i} \in \mathbb{Q}, i = 1, 2\}$, so $\mathcal{R}$ is the open set in $\mathbb{R}^{2}$. Then we know that $\sigma(\mathcal{R}) \subset \mathcal{B}(\mathbb{R}^{2})$. On the other hand, $V$ is open set and by the result we get in (i), we have $V = \bigcup_{R \in \mathcal{R}, R \subset V} R$. Since the number of set $R$ is countable, then we have $V \in \sigma(\mathcal{R})$. Thus the open sets in $\mathbb{R}^{2}$ is subset of $\sigma(\mathcal{R})$. Since $\mathcal{B}(\mathbb{R}^{2})$ is generated by the open sets in $\mathbb{R}^{2}$, then we have $\mathcal{B}(\mathbb{R}^{2}) \subset \sigma(\mathcal{R})$. So we can get $\mathcal{B}(\mathbb{R}^{2}) = \sigma(\mathcal{R})$. Then we know that the smallest sigma algebra of $\mathbb{R}^{2}$ containing $\mathcal{R}$ is equal to the set set of Borel sets of $\mathbb{R}^{2}$.

\end{document}

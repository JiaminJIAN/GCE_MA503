%GCE of WPI
%by Jiamin JIAN

\documentclass[12pt,a4paper]{ctexart}
\usepackage{CJK}
\usepackage{lipsum}
\usepackage{amsmath}
\usepackage{geometry}
\usepackage{titlesec}
\usepackage{amssymb}
\usepackage{epsfig}
\usepackage{float}
\usepackage{graphicx}
\usepackage{tabularx}
\usepackage{longtable}
\usepackage{amstext}
\usepackage{blkarray}
\usepackage{amsfonts}
\usepackage{bbm}
\usepackage{listings}
\geometry{left=2.5cm,right=2.5cm,top=2.5cm,bottom=2.5cm}

\begin{document}


\begin{center}
\textbf{ GCE January, 2019}
\vspace{8pt}

Jiamin JIAN
\end{center}

\vspace{12pt}

$\underline{\textbf{Exercise 1:}}$

Let $E:= [0, 1] - S_{\mathbb{Q}} = [0, 1] \bigcap (S_{\mathbb{Q}})^{c}$ where $S_{\mathbb{Q}} := \{x \in [0, 1] | x = \frac{\sqrt{p}}{q}   \,  \text{for some} \, p, q \in \mathbb{Z}^{+} \}$. Prove or disprove: There exists a closed, uncountable subset $F \subset E$.

\vspace{8pt}

$\textbf{Solution:}$

We can prove that there exists a closed, uncountable subset $F \subset E$. Since $S_{\mathbb{Q}}$ is a countable set, there exists a bijection between $S_{\mathbb{Q}}$ and the positive integer number in the interval $[0, 1]$, so we can enumerate the set $S_{\mathbb{Q}}$ as $\{a_{n} | n \in \mathbb{N} \}$. Thus we have $S_{\mathbb{Q}} = \{a_{n} | n \in \mathbb{N} \}$. And then we consider the union 
$$\bigcup_{n = 1}^{\infty}(a_{n} - \frac{\epsilon}{2^{n+1}}, a_{n} + \frac{\epsilon}{2^{n+1}}),$$
it is the union of countable many open intervals, thus it is an open set. Let
$$A = \Big{\{} \bigcup_{n = 1}^{\infty}(a_{n} - \frac{\epsilon}{2^{n+1}}, a_{n} + \frac{\epsilon}{2^{n+1}}) \Big{\}}\bigcap [0,1],$$
thus $A$ is an open subset of $[0,1]$. And we have $S_{\mathbb{Q}} \subset A$.

Since $A$ is an open subset of $[0,1]$, then $[0, 1] \bigcap A^c$ is closed in $[0,1]$. We define $F = [0, 1] \bigcap A^{c}$. Since the measure of set $A$ is
\begin{equation*}
    m(A) = 2 \sum_{n = 1}^{\infty}  \frac{\epsilon}{2^{n+1}} = \epsilon,
\end{equation*}
then we have $m(F) = 1 - \epsilon > 0$ for all $\epsilon < 1$. Thus the set $F$ is uncountable. Since $F \subset E$ and it is both closed and uncountable, then the proposition is true.

Note that for any countable set $S$, $S \subset [0, 1]$, let $E = [0, 1] - S$, we can find a closed, uncountable subset $F \subset E$, and we have the supremum of the measure of $F$ is 1.


\newpage

$\underline{\textbf{Exercise 2:}}$

For $x$ in $[-1, 1]$ set $P_{n} (x) = c_{n} (1 - x^{2})^{n}$ where $c_{n}$ is such that $\int_{-1}^{1} P_{n} = 1.$

(i) Show that there is a positive constant $C$ such that $c_{n} \leq C \sqrt{n}$.

(ii) Let $f$ be a real valued continuous function on $[0, 1]$ such that $f(0) = f(1) = 0$. Set for $x$ in $[0, 1]$
\begin{equation*}
    f_{n}(x) = \int_{0}^{1} P_{n}(x-t) f(t) \, d t
\end{equation*}
Show that $f_{n}$ is uniformly convergence to $f$.

(iii) Let $g$ be in $L^{1}((0, 1))$. Defining $g_{n}(x) = \int_{0}^{1} P_{n} (x- t) g(t) \, d t$, is $g_{n}$ uniformly convergence to $g$ in $(0, 1)$? Does $g_{n}$ converge to $g$ in $L^{1}((0, 1))$?

\vspace{8pt}
$\textbf{Solution:}$

(i) Method 1:

Since $\int_{-1}^{1} c_{n} (1-x^{2})^{n}\, d x = 1$, we have
\begin{equation*}
   c_{n} = \frac{1}{2 \int_{0}^{1}(1-x^{2})^{n} \, d x }.
\end{equation*}
Next we need to find a lower bound of the integral term $\int_{0}^{1}(1-x^{2})^{n} \, d x$. For $n > 1$,
\begin{eqnarray*}
\int_{0}^{1}(1-x^{2})^{n} \, d x &\geq& \int_{0}^{\frac{1}{\sqrt{n}}}(1-x^{2})^{n} \, d x  \\
            &\geq& \frac{1}{\sqrt{n}} (1 - \frac{1}{n})^{n},
\end{eqnarray*}
then we have $c_{n} \leq \frac{\sqrt{n}}{2 (1-\frac{1}{n})^{n}}$. We just need to find a lower bound of $(1 - \frac{1}{n})^{n}$. Since $(1 - \frac{1}{n})^{n} = 1 - C_{n}^{1} \frac{1}{n} + C_{n}^{2} \frac{1}{n^{2}} + \cdots + (-\frac{1}{n})^{n} > \frac{1}{3} - \frac{2}{6n^{2}}  > \frac{1}{4}$ when $n > 1$, choose $C = 2$, we have $c_{n} \leq C \sqrt{n}$ for $n>1$. And for $n = 1$, we get $c_{1} = \frac{3}{4} < 2$, for $C = 2$, we also have $c_{n} \leq C \sqrt{n}$ holds.

Method 2:

We change the element and denote $x = \sin y$, then we have $\int_{0}^{\frac{\pi}{2}}  c_{n} \cos^{2n+1}y \, d y = \frac{1}{2}$. Since 
\begin{equation*}
   \int_{0}^{\frac{\pi}{2}} \cos^{2n+1}y \, d y = 2n \int_{0}^{\frac{\pi}{2}} \cos^{2n-1}y \, d y - 2n \int_{0}^{\frac{\pi}{2}} \cos^{2n+1}y \, d y,
\end{equation*}
we denote $I_{2n + 1} = \int_{0}^{\frac{\pi}{2}} \cos^{2n+1}y \, d y$, then we can get that $(2n + 1)I_{2n+1} = 2n I_{2n-1}$. Since $I_{1} = \int_{0}^{\frac{\pi}{2}} \cos y \, d y = 1$, we have $\int_{0}^{\frac{\pi}{2}} \cos^{2n+1}y \, d y = \frac{(2n)!!}{(2n+1)!!}$. And since
\begin{eqnarray*}
\frac{(2n)!!}{(2n+1)!!} &=&  \frac{2n (2n-2) \cdots 2}{(2n+1) (2n-1) \cdots 3}  \\
            &\geq& \frac{\sqrt{2n+1}\sqrt{2n-1}\sqrt{2n-1}\sqrt{2n-3} \cdots \sqrt{3}\sqrt{1}}{(2n+1) (2n-1) \cdots 3} \\
            &=& \frac{1}{\sqrt{2n+1}},
\end{eqnarray*}
then we have $c_{n} \leq \frac{\sqrt{2n+1}}{2}$. Choose $C = 1$, we have $c_{n} \leq C \sqrt{n}$.


(ii) Firstly we extend $f(x)$ to a function from $\mathbb{R}$ to $\mathbb{R}$ by zero. Then
\begin{equation*}
    f_{n}(x) = \int_{0}^{1} P_{n}(x-t) f(t) \, d t = \int_{\mathbb{R}}^{} P_{n}(x-t) f(t) \, d t.
\end{equation*}
Denote $x - t = y$, we have
\begin{equation*}
    f_{n}(x) = \int_{\mathbb{R}}^{} P_{n}(y) f(x - y) \, d y.
\end{equation*}
Then we know that
\begin{eqnarray*}
|f_{n}(x) - f(x)| &=& \Big{|}\int_{\mathbb{R}}^{} P_{n}(y) f(x - y) \, d y - \int_{-1}^{1} P_{n}(y) f(x) \, d y \Big{|}  \\
&=& \Big{|} \int_{-1}^{1} P_{n}(y) (f(x - y) - f(x)) \, d y + \int_{([-1,1])^{c}}^{} P_{n}(y) f(x-y) \, d y \Big{|} \\
&\leq& \int_{-1}^{1} P_{n}(y) | (f(x - y) - f(x))| \, d y + \int_{([-1,1])^{c}}^{} |P_{n}(y) f(x-y)| \, d y.
\end{eqnarray*}
When $x \in [0, 1]$ and $y \in ([-1, 1])^{c}$, we have $x - y > 1$ or $x - y < 0$, then we know that $f(x -y) = 0$, thus
\begin{equation*}
    |f_{n}(x) - f(x)| \leq \int_{-1}^{1} P_{n}(y) | (f(x - y) - f(x))| \, d y.
\end{equation*}
As $f$ is continuous on $\mathbb R$, $\forall \epsilon > 0$, there exists $\delta > 0$ such that when $|x - y -x| < \delta$, we have $|f(x-y) - f(x)| < \epsilon/2$. Denote $S = [-1,1] \bigcap (-\delta, \delta)^c$. Since $f(x)$ is continuous in $\mathbb{R}$, we denote $\sup_{x \in [0, 1]} f(x) = \sup_{x \in \mathbb R} f(x) = M$, then $M < + \infty$ and
\begin{eqnarray*}
|f_{n}(x) - f(x)| &\leq& \int_{-\delta}^{\delta} P_{n}(y) | (f(x - y) - f(x))| \, d y  + \int_{S}^{} P_{n}(y) | (f(x - y) - f(x))| \, d y  \\
&\leq& \frac{\epsilon}{2} \int_{-\delta}^{\delta} P_{n}(y) \, d y  + 2M \int_{S}^{} P_{n}(y) \, d y   \\
&\leq& \frac{\epsilon}{2} + 2M \int_{S}^{} c_{n} (1 - y^{2})^{n} \, d y   \\
&\leq& \frac{\epsilon}{2} + 4M C \sqrt{n} \int_{\delta}^{1} (1 - y^{2})^{n} \, d y   \\
&\leq& \frac{\epsilon}{2} + 4M C \sqrt{n} (1 - \delta)(1 - \delta^{2})^{n},
\end{eqnarray*}
where $C$ is a constant from the (i).
Since $\lim_{n \to + \infty} 4M C \sqrt{n} (1 - \delta)(1 - \delta^{2})^{n} = 0 $, then there exists a $N \in \mathbb{N}$ such that when $n > N$, we have $4M C \sqrt{n} (1 - \delta)(1 - \delta^{2})^{n} < \epsilon/2$. Overall, we know that $\forall \epsilon > 0$, there exists a $N \in \mathbb{N}$ such that for all $n > N$ and $x \in [0,1]$, we have $|f_{n}(x) - f(x)| < \epsilon$, Thus $f_{n}$ converges uniformly to $f$ on $[0,1]$.


(iii) Firstly, the $g_{n}(x)$ is not uniformly convergent to $g$ on $(0, 1)$, we can give an counter example as following. Let
\begin{equation*}
g(x) =
\left\{
             \begin{array}{cl}
             1, & x = \frac{1}{2} \\
             0, & x \in (0, \frac{1}{2}) \bigcup (\frac{1}{2}, 1),
             \end{array}
\right.
\end{equation*}
obviously $g(x)$ is not continuous in $(0, 1)$, but we have $g_{n} (x) = \int_{0}^{1} P_{n}(x -t)g(t) \, d t = 0, \forall x \in (0, 1)$. Thus $g_{n} (x)$ is continuous in $[0, 1]$.  Therefore $g_{n}(x)$ does not convergent uniformly to $g(x)$ on $(0, 1)$.

Secondly, we can show that $g_{n}(x)$ convergent to $g(x)$ in $L^{1}((0, 1))$. Since the continuous functions with compact support are dense in $L^{1}$ space, then for all $\epsilon > 0$, there exist a continuous function $f(x) \in C_{c}([0, 1])$ such that $\|f- g \|_{1} < \epsilon$. We define the $f_{n}(x)$ as (ii), then by triangle inequality for the norm,
\begin{equation*}
\|g- g_{n} \|_{1} \leq  \|g- f \|_{1} + \|f- f_{n} \|_{1} + \|f_{n}- g_{n} \|_{1}.
\end{equation*}
Since $f_{n}$ is uniformly converges to $f$, for all $\epsilon > 0$, there exists a $N \in \mathbb{N}$ such that for all $n > N$ and $x \in (0,1)$, we have $|f(x)- f_{n}(x) | < \epsilon/2$. Thus for the above $\epsilon > 0$, for all $n \geq N$,
$$\|f - f_n\|_1 = \int_{(0,1)} |f(x) - f_n(x)| \, d x \leq \frac{\epsilon}{2} m((0,1)) = \frac{\epsilon}{2}.$$ 

 Since $P_{n}(x-t)$ is continuous for $t \in [0, 1]$, we can find the upper bound for  $P_{n}(x-t)$, which is denoted as $C$. And since
\begin{eqnarray*}
\|f_{n}- g_{n} \|_{1} &=& \int_{0}^{1} \Big{|} \int_{0}^{1}  P_{n}(x-t) g(t) - \int_{0}^{1}  P_{n}(x-t) f(t) \, d t \Big{|} d x  \\
&=& \int_{0}^{1} \Big{|} \int_{0}^{1}  P_{n}(x-t) (g(t) - f(t)) \, d t \Big{|} d x  \\
&\leq& \int_{0}^{1}  \int_{0}^{1}  P_{n}(x-t) |g(t) - f(t)|\, d t  d x,
\end{eqnarray*}
we have
\begin{eqnarray*}
\|f_{n}- g_{n} \|_{1} &\leq&  \int_{0}^{1}  \int_{0}^{1}  P_{n}(x-t) |g(t) - f(t)|\, d t  d x \\
&\leq&  C \int_{0}^{1}  \int_{0}^{1} |g(t) - f(t)|\, d t  d x \\
&=&  C \int_{0}^{1} |g(t) - f(t)|\, d t \\
&=& C \|g-f \|_{1}.
\end{eqnarray*}
For the above $\epsilon > 0$, by the property that continuous function is dense in $L^{1}$ space, we have $\|f- g \|_{1} < \epsilon/(2(C+1))$. Thus for all $n \geq N$,
\begin{eqnarray*}
\| g_{n}  -g \|_{1} &\leq&  \|g- f \|_{1} + \|f- f_{n} \|_{1} + \|f_{n}- g_{n} \|_{1} \\
&\leq&  \|g- f \|_{1} + \frac{\epsilon}{2} + C \|g - f \|_1 \\
&=& (C+1) \|g - f \|_1 + \frac{\epsilon}{2}  \\
& \leq & \epsilon.
\end{eqnarray*}
Therefore we have $g_{n}$ convergent to $g$ in $L^{1}((0, 1))$.


\newpage


$\underline{\textbf{Exercise 3:}}$

Give an example of $f_n, f : \mathbb{R} \mapsto [0, \infty)$ such that $f_{n} \in L^{1}(\mathbb{R})$ for every $n \in \mathbb{N}$, $f \in L^{2}(\mathbb{R})$, $f_{n} \leq f$ for every $n \in \mathbb{N}$, $f_{n} \to 0$ a.e., and $\int_{}^{} f_{n} \nrightarrow 0$.

\vspace{8pt}
$\textbf{Solution:}$

Let $f(x) = \frac{1}{x} \mathbb{I}_{[1, \infty)}$ and $f_{n}(x) = \frac{1}{x} \mathbb{I}_{[n, 2 n]}$. For each $n \in \mathbb N$, we have
\begin{equation*}
\int_{\mathbb{R}}^{} |f_{n}(x)| \, d x = \int_{n}^{2 n} \frac{1}{x} \, d x = \ln{2},
\end{equation*}
thus $f_{n} \in L^{1}(\mathbb{R})$ for every $n \in \mathbb{N}$. And since
\begin{equation*}
\int_{\mathbb{R}}^{} |f(x)|^{2} \, d x = \int_{1}^{ \infty} \frac{1}{x^{2}} \, d x = 1,
\end{equation*}
we have $f \in L^{2}(\mathbb{R})$. For all $n \in \mathbb N$, $f_{n}$ is just the restriction of $f$ on the interval $[n, 2n]$, then $f_{n} \leq f$ for every $n \in \mathbb{N}$. For each $n \in \mathbb N$, we have $f_{n}(x) \leq \frac{1}{n}$ for all $x \in [n, 2n]$, thus $f_{n} \to 0$ almost everywhere. But we have
\begin{equation*}
\int_{\mathbb{R}}^{} f_{n}(x) \, d x = \int_{n}^{2 n} \frac{1}{x} \, d x = \ln{2},
\end{equation*}
for all $n \in \mathbb N$, thus $\int_{}^{} f_{n} \nrightarrow 0$.



\end{document}

%GCE of WPI
%by Jiamin JIAN

\documentclass[12pt,a4paper]{ctexart}
\usepackage{CJK}
\usepackage{lipsum}
\usepackage{amsmath}
\usepackage{geometry}
\usepackage{titlesec}
\usepackage{amssymb}
\usepackage{epsfig}
\usepackage{float}
\usepackage{graphicx}
\usepackage{tabularx}
\usepackage{longtable}
\usepackage{amstext}
\usepackage{blkarray}
\usepackage{amsfonts}
\usepackage{bbm}
\usepackage{listings}
\geometry{left=2.5cm,right=2.5cm,top=2.5cm,bottom=2.5cm}

\begin{document}


\begin{center}
\textbf{ GCE August, 2020}
\vspace{8pt}

Jiamin JIAN
\end{center}

\vspace{12pt}

$\underline{\textbf{Exercise 1:}}$

Let $(X, \rho)$ be a metric space and $S$ and $T$ two non-empty subsets of $X$. Define
\begin{equation*}
    d(S, T) = \max \{\sup_{x \in S} \inf_{y \in T} \rho(x, y), \sup_{y \in T} \inf_{x \in S} \rho(x, y)\}.
\end{equation*}
Show that $d(S, T) = 0$ if and only if $S$ and $T$ have the same closure.

\vspace{8pt}
$\textbf{Solution:}$

Firstly we show that if $d(S, T) = 0$, $S$ and $T$ have the same closure. By the definition of $d(S, T)$, if $d(S, T) = 0$, then
$$\max \{\sup_{x \in S} \inf_{y \in T} \rho(x, y), \sup_{y \in T} \inf_{x \in S} \rho(x, y)\} = 0.$$
For any $x \in S, y \in T$, we have $\sup_{x \in S} \inf_{y \in T} \rho(x, y) \geq 0$ and $\sup_{y \in T} \inf_{x \in S} \rho(x, y) \geq 0$, then
$$\sup_{x \in S} \inf_{y \in T} \rho(x, y) = 0 \quad  \text{and}  \quad \sup_{y \in T} \inf_{x \in S} \rho(x, y) = 0.$$
Thus we have
$$\forall x \in S, \inf_{y \in T} \rho(x, y) = 0 \quad  \text{and}  \quad \forall y \in T, \inf_{x \in S} \rho(x, y) = 0.$$
Let $\epsilon > 0$ be given. For fixed $x \in S$, if $\inf_{y \in T} \rho(x, y) = 0$, there exists $y \in T$ such that $\rho(x, y) - \epsilon < 0$, which implies $x$ is a point of closure of $T$, i.e. $x \in \bar{T}$, where $\bar{T}$ is the closure of $T$. By the arbitrary of $x \in S$, we have $S \subset \bar{T}$, thus $\bar{S} \subset \bar{T}$. Similarly, $\forall y \in T$, $\inf_{x \in S} \rho(x, y) = 0$, we also have $T \subset \bar{S}$, thus $\bar{T} \subset \bar{S}$. Therefore $\bar{S} = \bar{T}$, $S$ and $T$ have the same closure.

Next we want to prove that if $S$ and $T$ have the same closure, then $d(S, T) = 0$. If $\bar{S} = \bar{T}$, $\bar{S} \subset \bar{T}$, and then $S \subset \bar{S} \subset \bar{T}$. Let $\epsilon > 0$ be given. For any fixed $x \in S$, there exists $y \in T$ such that $\rho(x, y) < \epsilon$. By the arbitrary of $\epsilon$, $\forall x \in S$, we have $\inf_{y \in T} \rho(x, y) = 0$. Thus $\sup_{x \in S} \inf_{y \in T} \rho(x, y) = 0$. Similarly, since $T \subset \bar{T} \subset \bar{S}$, we have $\sup_{y \in T} \inf_{x \in S} \rho(x, y) = 0$. Therefore,
$$d(S, T) = \max \{\sup_{x \in S} \inf_{y \in T} \rho(x, y), \sup_{y \in T} \inf_{x \in S} \rho(x, y)\} = 0.$$

\newpage 

$\underline{\textbf{Exercise 2:}}$

Show that for every set $S \subset \mathbb R$ there exists a Borel set $B$ such that $S \subset B$ and $m^{*}(S) = m^{*}(B)$, where $m^*$ is the Lebesgue outer measure. Then show that for such $S$ and $B$ with $m^{*}(S) < \infty$, $S$ is measurable if and only if $m^{*} (B \setminus S) = 0$.

\vspace{8pt}
$\textbf{Solution:}$

Firstly we show that for every set $S \subset \mathbb R$ there exists a Borel set $B$ such that $S \subset B$ and $m^{*}(S) = m^{*}(B)$. If $m^{*}(S) = \infty$, choose $B = \mathbb R$. Then $B$ is a Borel set, $S \subset B = \mathbb R$ and $m^{*}(S) = m^{*}(B) = \infty$. If $m^{*}(S) < \infty$. Let $\epsilon > 0$ be given. By the definition of $m^{*}(S)$, there exists a countable collection $\{I_{k}\}$ of open intervals such that 
$$S \subset \bigcup_{k = 1}^{\infty} I_{k} \quad \text{and} \quad \sum_{k = 1}^{\infty} \ell(I_{k}) < m^{*}(S) + \epsilon.$$
Let $O = \bigcup_{k = 1}^{\infty} I_{k}$, then $O$ is an open set, $S \subset O$ and
$$m^{*}(O \setminus S) \leq m^{*}(O) - m^{*}(S) = m^{*}(\bigcup_{k = 1}^{\infty} I_{k}) - m^{*}(S) \leq \sum_{k = 1}^{\infty} \ell(I_{k}) - m^{*}(S) < \epsilon.$$
Thus for each $n \in \mathbb N$, there exists an open set $O_n$ such that
$$S \subset O_n \quad \text{and} \quad m^{*}(O_n \setminus S) < \frac{1}{n}.$$
Let $B = \bigcap_{n = 1}^{\infty} O_n$, then $S \subset B$, $B$ is a Borel set and
$$m^{*}(B) - m^{*}(S) \leq m^{*}(B \setminus S) \leq m^{*}(O_n \setminus S) < \frac{1}{n}, \quad \forall n \in \mathbb N.$$
Let $n \to \infty$, then $m^{*}(B) - m^{*}(S) = 0$.

Next we show that for such $S$ and $B$ with $m^{*}(S) < \infty$, $S$ is measurable if and only if $m^{*} (B \setminus S) = 0$. If $S$ is measurable and $m^{*}(S) < \infty$, by the construction of $B$ and the arguement on the above, we have
$$m^{*}(B \setminus S) \leq m^{*}(O_n \setminus S) < \frac{1}{n}, \quad \forall n \in \mathbb N.$$
Thus $m^{*}(B \setminus S) = 0$. If $m^{*}(S) < \infty$ and $m^{*} (B \setminus S) = 0$, we want to prove that $S$ is measurable. $B$ is a Borel set, thus $B$ is measurable. Since $m^{*} (B \setminus S) = 0$, we have $B \setminus S$ is measurable. Therefore $S = B \setminus (B \setminus S)$ is measurable as the family of measurable sets is a $\sigma$-algebra.







\newpage 

$\underline{\textbf{Exercise 3:}}$

Suppose $f_n, g_n$ are Lebesgue measurable function on $\mathbb R$, with $f_n, g_n \geq 0, \forall n \in \mathbb N$. Suppose also that $f_n \to f$ a.e., $g_n \to g$ a.e.,
$$\int f_n \to \int f < \infty,$$
and
$$\int g_n \to \int g < \infty.$$
Prove or give a counterexample: if $\{f_n g_n\}$ is bounded in $L^1$, then
$$\int f_n g_n \to \int f g.$$

\vspace{8pt}
$\textbf{Solution:}$

We can give a counterexample as follows: for each $n \in \mathbb N$, let
$$f_n (x) = g_n (x) = \sqrt{n} 1_{[0, \frac{1}{n}]}(x) + 1_{[0,1]}(x)$$
and let $f(x) = g(x) = 1_{[0,1]}(x)$. Then we have $f_n, g_n$ are Lebesgue measurable function on $\mathbb R$, with $f_n, g_n \geq 0, \forall n \in \mathbb N$. And we have $f_n \to f$ a.e., $g_n \to g$ a.e.. And since
$$\int f_n = \int g_n = \int_{0}^{1} \Big{(} \sqrt{n} 1_{[0, \frac{1}{n}]}(x) + 1 \Big{)} \, d x = \frac{1}{\sqrt{n}} + 1 \to 1$$
as $n \to \infty$ and 
$$\int f = \int g = \int_{0}^{1} 1 \, d x = 1,$$
then $\int f_n \to \int f < \infty$ and $\int g_n \to \int g < \infty$. For each $n \in \mathbb N$,
\begin{eqnarray*}
    \int f_n g_n  & = & \int_{0}^{1} f_n^2(x) \, d x \\
    & = & \int_{0}^{1} \Big{(} n 1_{[0,\frac{1}{n}]}(x) + 2 \sqrt{n} 1_{[0, \frac{1}{n}]}(x) + 1 \Big{)} \, d x \\
    & = & 2 + \frac{2}{\sqrt{n}} \leq 4 < \infty,
\end{eqnarray*}
thus $\{f_n g_n\}$ is bounded in $L^1$, and we know that
\begin{equation*}
    \int f_n g_n = 2 + \frac{2}{\sqrt{n}} \to 2
\end{equation*}
as $n \to \infty$. But
$$\int f g = \int (1_{[0,1] (x)})^2 \, d x = \int_{0}^{1} 1 \, d x = 1,$$
then we have $\int f_n g_n$ does not converge to $\int f g$.

\end{document}

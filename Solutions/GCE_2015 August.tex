%GCE of WPI
%by Jiamin JIAN

\documentclass[12pt,a4paper]{ctexart}
\usepackage{CJK}
\usepackage{lipsum}
\usepackage{amsmath}
\usepackage{geometry}
\usepackage{titlesec}
\usepackage{amssymb}
\usepackage{epsfig}
\usepackage{float}
\usepackage{graphicx}
\usepackage{tabularx}
\usepackage{longtable}
\usepackage{amstext}
\usepackage{blkarray}
\usepackage{amsfonts}
\usepackage{bbm}
\usepackage{listings}
\geometry{left=2.5cm,right=2.5cm,top=2.5cm,bottom=2.5cm}

\begin{document}


\begin{center}
\textbf{ GCE August, 2015}
\vspace{8pt}

Jiamin JIAN
\end{center}

\vspace{12pt}

$\underline{\textbf{Exercise 1:}}$

Use the Fubini theorem to prove that
\begin{equation*}
    \int_{\mathbb{R}^{n}}^{} e^{- |\textbf{x}|^{2}} \, d \textbf{x} = \pi^{\frac{n}{2}}
\end{equation*}
Here $\textbf{x} = (x_{1}, x_{2}, \dots, x_{n})$. Hint: For $n = 2$, use polar coordinates.


\vspace{8pt}

$\textbf{Solution:}$

Firstly, for $a > 0$, we define 
\begin{equation*}
    I(a) = \int_{-a}^{a} e^{- x^{2}} \, d x,
\end{equation*}
then we have
\begin{equation*}
    I^{2}(a) = \int_{-a}^{a} e^{- x^{2}} \, d x \int_{-a}^{a} e^{- y^{2}} \, d y.
\end{equation*}
As $(-a, a)$ is an interval with finite measure and $|e^{-x^{2}}| \leq 1$, by the Fubini theorem, we have
\begin{equation*}
    I^{2}(a) = \int_{-a}^{a} \int_{-a}^{a} e^{- (x^{2}+y^{2})} \, d x \, d y.
\end{equation*}
Take the polar coordinates transformation as follows,
\begin{equation*}
\left\{
             \begin{array}{cl}
             x = r \cos \theta \\
             y = r \sin \theta, 
             \end{array}
\right.
\end{equation*}
then we have
\begin{equation*}
    \int_{0}^{2 \pi} \int_{0}^{a} r e^{-r^{2}} \, d r\, d \theta < I^{2}(a) < \int_{0}^{2 \pi} \int_{0}^{\sqrt{2} a} r e^{-r^{2}}  \, d r \, d \theta.
\end{equation*}
By calculation, we can get the inequalities
\begin{equation*}
    (1 - e^{- a^{2}}) \pi < I^{2}(a) < (1 - e^{- 2a^{2}}) \pi .
\end{equation*}
Let $a \to \infty$, we have \begin{equation*}
    \lim_{a \to \infty} I^{2}(a) = \int_{\mathbb{R}^{2}}^{} e^{- |\textbf{x}|^{2}} \, d \textbf{x} = \pi,
\end{equation*}
then we know that $\lim_{a \to \infty} I(a) = \int_{\mathbb{R}}^{} e^{- x^{2}} \, d x = \sqrt{\pi}$. For the $n$ dimensional domain, we have
\begin{eqnarray*}
    \int_{\mathbb{R}^{n}}^{} e^{- |\textbf{x}|^{2}} \, d \textbf{x} & = & \int_{\mathbb{R}}^{} \int_{\mathbb{R}}^{} \cdots \int_{\mathbb{R}}^{} e^{- (x_{1}^{2} + x_{2}^{2} + \cdots x_{n}^{2})} \, d x_{1} d x_{2} \cdots d x_{n} \\
    & = & \Big{(} \int_{\mathbb{R}}^{} e^{- x_{1}^{2} } \, d x_{1} \Big{)}^{n} = \pi^{\frac{n}{2}}.
\end{eqnarray*}

\newpage

$\underline{\textbf{Exercise 2:}}$

Let $(X, \mathcal{A}, \mu)$ be a measure space, and $f$ be in $L^{1}(X)$. Let for all positive integers $n$ set $B_{n} = \{x \in X: n-1 \leq |f(x)| < n \}$.

(i) Show that $\mu(B_{n}) < \infty$ for all $n \geq 2$.

(ii) Show that $\sum_{n=2}^{\infty} n \mu(B_{n}) < \infty$.

(iii) Define $C_{n} = \{x \in X : n-1 \leq |f(x)| \leq n \}$. Is the sum $\sum_{n=2}^{\infty} n \mu(C_{n})$ finite?

(iv) Show that $$\sum_{n=2}^{\infty} \sum_{m=2}^{n} \frac{m^{2}}{n^{2}} \mu(B_{m}) < \infty.$$

(v) Show that for $n \geq 2$
\begin{equation*}
    \int_{}^{} |f|^{2} 1_{\{|f| < n\}} = \int_{}^{} |f|^{2} 1_{\{|f| < 1\}} + \sum_{m=2}^{n} \int_{}^{} |f|^{2} 1_{B_{m}}
\end{equation*}
and infer that
\begin{equation*}
    \sum_{n=1}^{\infty} \frac{1}{n^{2}} \int_{}^{} |f|^{2} 1_{\{|f| < n\}} < \infty 
\end{equation*}
 
 
 
\vspace{8pt}
$\textbf{Solution:}$

(i) Since $f \in L^{1}(X)$, we have $\int_{X}^{} |f| \, d \mu < \infty$. For $n \geq 2$, we know that
\begin{equation*}
    \int_{X}^{} |f| \, d \mu \geq \int_{B_{n}}^{} |f| \, d \mu \geq (n-1) \int_{B_{n}}^{} 1 \, d \mu = (n-1) \mu(B_{n}).
\end{equation*}
When $n \geq 2$, we have $n-1 \geq 1$, then we know that $\mu(B_{n}) < \infty$.  So, for any $n \geq 2$, we have $\mu(B_{n}) < \infty$.

(ii) Since $B_{n} = \{ x \in X: n-1 \leq |f(x)| < n \} = \{x \in X: n \leq |f(x)|+1 < n+1 \}$, we have
\begin{eqnarray*}
    \sum_{n=2}^{\infty} n \mu(B_{n}) & = & \sum_{n=2}^{\infty} \int_{B_{n}}^{} n \, d \mu \\
& \leq & \sum_{n=2}^{\infty}  \int_{B_{n}}^{} |f(x)| + 1 \, d \mu \\
& = & \sum_{n=2}^{\infty}  \int_{B_{n}}^{} |f(x)| \, d \mu + \sum_{n=2}^{\infty}  \int_{B_{n}}^{} 1 \, d \mu \\ 
& \leq & 2 \int_{\bigcup_{n=2}^{\infty} B_{n}}^{} |f(x)| \, d \mu \\
& \leq & 2 \int_{X}^{} |f(x)| \, d \mu < \infty.
\end{eqnarray*}

(iii) We claim that the sum $\sum_{n=2}^{\infty} n \mu(C_{n})$ is finite. As $C_{n} = \{x \in X : n-1 \leq |f(x)| \leq n \} \subset B_{n} \cup B_{n+1}$, we have
\begin{equation*}
    \mu(C_{n}) \leq \mu(B_{n} \cup B_{n+1}) \leq \mu(B_{n}) + \mu(B_{n+1}),
\end{equation*}
thus
\begin{equation*}
    \sum_{n=2}^{\infty} n \mu(C_{n}) \leq \sum_{n=2}^{\infty} n \mu(B_{n}) +  \sum_{n=2}^{\infty} n \mu (B_{n+1}).
\end{equation*}
Since $\int_{B_{n+1}}^{} |f| \, d \mu  \geq n \int_{B_{n+1}}^{} 1 \, d \mu = n \mu(B_{n+1})$, we have
\begin{eqnarray*}
    \sum_{n=2}^{\infty} n \mu(B_{n+1}) & \leq & \sum_{n=2}^{\infty} \int_{B_{n+1}}^{} |f| \, d \mu \\
    & = & \int_{\bigcup_{n=2}^{\infty} B_{n+1}}^{} |f| \, d \mu \\
    & < & \int_{X}^{} |f| \, d \mu < \infty.
\end{eqnarray*}
As we showed $\sum_{n=2}^{\infty} n \mu(B_{n}) < \infty$ in (ii), hence we have
\begin{equation*}
    \sum_{n=2}^{\infty} n \mu(C_{n}) \leq \sum_{n=2}^{\infty} n \mu(B_{n}) +  \sum_{n=2}^{\infty} n \mu (B_{n+1}) < \infty.
\end{equation*}

(iv) We can rewrite the $\sum_{n=2}^{\infty} \sum_{m=2}^{n} \frac{m^{2}}{n^{2}} \mu(B_{m}) $ and get
\begin{eqnarray*}
    \sum_{n=2}^{\infty} \sum_{m=2}^{n} \frac{m^{2}}{n^{2}} \mu(B_{m}) & = & \sum_{m=2}^{\infty} \mu(B_{m}) m^{2} \sum_{n=m}^{\infty} \frac{1}{n^{2}}  \\
    & = & \sum_{m=2}^{\infty} m \mu(B_{m}) \sum_{n=m}^{\infty} \frac{m}{n^{2}}.
\end{eqnarray*}
Next we need to show that $\sum_{n=m}^{\infty} \frac{m}{n^{2}}$ is bounded. When $m \geq 2$, we have
\begin{equation*}
    \sum_{n=m}^{\infty} \frac{m}{n^{2}} < m \int_{m-1}^{\infty} \frac{1}{x^{2}} \, d x = \frac{m}{m-1} \leq 2,
\end{equation*}
then we know that
\begin{equation*}
    \sum_{n=2}^{\infty} \sum_{m=2}^{n} \frac{m^{2}}{n^{2}} \mu(B_{m}) < 2 \sum_{m=2}^{\infty} m \mu(B_{m}) < \infty.
\end{equation*}

Or by the inequalities as follows,
\begin{eqnarray*}
    \sum_{n=2}^{\infty} \sum_{m=2}^{n} \frac{m^{2}}{n^{2}} \mu(B_{m}) & = & \sum_{m=2}^{\infty} \mu(B_{m}) m^{2} \sum_{n=m}^{\infty} \frac{1}{n^{2}}  \\
    & < & \sum_{m=2}^{\infty} m^2 \mu(B_{m}) \sum_{n=m}^{\infty} \frac{1}{n(n-1)} \\
    & = & \sum_{m=2}^{\infty} m^2 \mu(B_{m}) \frac{1}{m-1} \\
    & = & \sum_{m=2}^{\infty} m \mu(B_{m}) \frac{m}{m-1} \\
    & < & \sum_{m=2}^{\infty} 2 m \mu(B_{m}) < \infty.
\end{eqnarray*}

(v) Firstly, we show that 
\begin{equation*}
    \int_{}^{} |f|^{2} 1_{\{|f| < n\}} \, d \mu = \int_{}^{} |f|^{2} 1_{\{|f| < 1\}} \, d \mu + \sum_{m=2}^{n} \int_{}^{} |f|^{2} 1_{B_{m}} \, d \mu.
\end{equation*}
By calculation, we have
\begin{eqnarray*}
    \int_{}^{} |f|^{2} 1_{\{|f| < n\}} \, d \mu & = & \int_{}^{} |f|^{2} 1_{\{|f| < 1\}} \, d \mu + \int_{}^{} |f|^{2} 1_{\{1 \leq |f| < n\}} \, d \mu \\
    & = & \int_{}^{} |f|^{2} 1_{\{|f| < 1\}} \, d \mu + \int_{}^{} |f|^{2} \sum_{m=2}^{n} 1_{\{m-1 \leq |f| < m\}} \, d \mu \\
    & = & \int_{}^{} |f|^{2} 1_{\{|f| < 1\}} \, d \mu + \sum_{m=2}^{n} \int_{}^{} |f|^{2}  1_{\{m-1 \leq |f| < m\}} \, d \mu \\
    & = & \int_{}^{} |f|^{2} 1_{\{|f| < 1\}} \, d \mu + \sum_{m=2}^{n} \int_{}^{} |f|^{2} 1_{B_{m}} \, d \mu,
\end{eqnarray*}
then we get the equation we wanted. Next we show that $\sum_{n=1}^{\infty} \frac{1}{n^{2}} \int_{}^{} |f|^{2} 1_{\{|f| < n\}} < \infty $. Note that
\begin{eqnarray*}
   &  & \sum_{n=1}^{\infty} \frac{1}{n^{2}} \int_{}^{} |f|^{2} 1_{\{|f| < n\}} \, d \mu  \\
    & = &  \sum_{n=1}^{\infty} \frac{1}{n^{2}} \Big{(} \int_{}^{} |f|^{2} 1_{\{|f| < 1\}} \, d \mu + \sum_{m=2}^{n} \int_{}^{} |f|^{2} 1_{B_{m}} \, d \mu \Big{)} \\
    & = & \sum_{n=1}^{\infty} \frac{1}{n^{2}} \int_{}^{} |f|^{2} 1_{\{|f| < 1\}} \, d \mu + \sum_{n=1}^{\infty} \frac{1}{n^{2}} \sum_{m=2}^{n} \int_{}^{} |f|^{2} 1_{B_{m}} \, d \mu.
\end{eqnarray*}
For the first term in the right hand side of the above equation, we have
\begin{equation*}
    \sum_{n=1}^{\infty} \frac{1}{n^{2}} \int_{}^{} |f|^{2} 1_{\{|f| < 1\}} \, d \mu < \sum_{n=1}^{\infty} \frac{1}{n^{2}} \int_{X}^{} |f| \, d \mu  < \infty.
\end{equation*}
And for the second term in the right hand side of the above equation, we have
\begin{eqnarray*}
    \sum_{n=1}^{\infty} \frac{1}{n^{2}} \sum_{m=2}^{n} \int_{}^{} |f|^{2} 1_{B_{m}} \, d \mu & < & \sum_{n=2}^{\infty} \frac{1}{n^{2}} \sum_{m=2}^{n} \int_{}^{} m^{2} 1_{B_{m}} \, d \mu \\
   & = & \sum_{n=2}^{\infty} \sum_{m=2}^{n} \frac{m^{2}}{n^{2}} \mu(B_{m}) < \infty.
\end{eqnarray*}
Thus we have
\begin{equation*}
    \sum_{n=1}^{\infty} \frac{1}{n^{2}} \int_{}^{} |f|^{2} 1_{\{|f| < n\}} \, d \mu  < \infty.
\end{equation*}








\newpage

$\underline{\textbf{Exercise 3:}}$

Prove or disprove: suppose that $f, g: \mathbb{R} \rightarrow \mathbb{R}$, with $f$ being a measurable function, and $g$ being a continuous function. Then $f \circ g$ is measurable. By definition, $(f \circ g)(x) = f(g(x))$, that is, it is the composition of the two functions.


\vspace{8pt}
$\textbf{Solution:}$

No, the statement is not true and we can find a counter example as follows. Suppose that $C$ is the Cantor set and we define a mapping $\phi$: for any $x \in C$, let $0.c_{1}c_{2}c_{3} \cdots$ be its ternary expansion, where $c_{n} = 0$ or $c_{n} = 2$, $n = 1, 2, \cdots$ and let
\begin{equation*}
    \phi(x) = 0.\frac{c_{1}}{2}\frac{c_{2}}{2}\frac{c_{3}}{2}\cdots,
\end{equation*}
where the expansion on the right is now interpreted as a binary expansion in terms of digits $0$ and $1$. It is clear that the image of $C$, under $\phi$, is a subset of $[0, 1]$. And next we extend the domain to the entire unit interval $[0, 1]$. If $x \in [0, 1] \setminus C $, then $x$ is a member of one of the open intervals $(a, b)$ removed from $[0, 1]$ in the construction of $C$, and therefore $\phi(a) = \phi(b)$. And we define $\phi(x) = \phi(a) = \phi(b)$. Since $\phi(\cdot)$ is increasing on $[0, 1]$, and since the range of $\phi(\cdot)$ is the entire interval $[0, 1]$, $\phi(\cdot)$ has no jump discontinuities. Since a monotonic function can have no discontinuities other than jump discontinuities, we know that $\phi(\cdot)$ is continuous. Then we define
$$\varphi (x) = x + \phi(x), \, x \in [0, 1]$$
with range $[0, 2]$. Since $\phi(\cdot)$ is increasing on $[0, 1]$ and continuous there, $\varphi$ is strictly increasing and topological there (continuous and one-to-one with a continuous inverse on the range $\varphi$). Since each open interval removed from $[0,1]$ in the construction of the Cantor set $C$ is mapped by $\varphi$ onto an interval of $[0, 2]$ of the equal length, $\mu(\varphi(I \setminus C)) = \mu(I \setminus C) = 1$. Since $C$ is a set of measure zero, $\varphi$ is an example of a topological mapping that maps a set of measure zero onto a set of positive measure.

Now let $D$ is a non-measurable subset of $\varphi(C)$ and let $E = \varphi^{-1}(D)$. Then the characteristic function $f = 1_{E}(x)$ of the set $E$ is measurable and $g = \varphi^{-1}$ is continuous, but the composite function $f(g(x))$ is non-measurable characteristic function of the non-measurable set $D$.

$\textbf{Claim:}$ suppose that $f, g: \mathbb{R} \rightarrow \mathbb{R}$, with $f$ being a measurable function, and $g$ being a continuous function. Then $g \circ f$ is measurable.

$\textbf{Proof:}$ Since $f: (\mathbb{R}, \mathcal{B}_{\mathbb{R}}) \rightarrow (\mathbb{R}, \mathcal{B}_{\mathbb{R}})$ is Lebesgue-measurable and as $g: \mathbb{R} \rightarrow \mathbb{R}$ is continuous, it is Borel-measurable. Take any $B \in \mathcal{B}_{\mathbb{R}}$, we want to show that $(g \circ f)^{-1}(B) \in \mathcal{B}_{\mathbb{R}}$. By measurability of $g$, since $B \in \mathcal{B}_{\mathbb{R}}$, we have $B^{'} = g^{-1}(B) \in \mathcal{B}_{\mathbb{R}}$. By the measurability of $f$, this implies that $f^{-1}(B^{'}) \in \mathcal{B}_{\mathbb{R}}$. This shows that $g \circ f$ is measurable for the $\sigma$-algebras $\mathcal{B}_{\mathbb{R}}$.













 

\end{document}

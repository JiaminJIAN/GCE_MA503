%GCE of WPI
%by Jiamin JIAN

\documentclass[12pt,a4paper]{ctexart}
\usepackage{CJK}
\usepackage{lipsum}
\usepackage{amsmath}
\usepackage{geometry}
\usepackage{titlesec}
\usepackage{amssymb}
\usepackage{epsfig}
\usepackage{float}
\usepackage{graphicx}
\usepackage{tabularx}
\usepackage{longtable}
\usepackage{amstext}
\usepackage{blkarray}
\usepackage{amsfonts}
\usepackage{bbm}
\usepackage{listings}
\geometry{left=2.5cm,right=2.5cm,top=2.5cm,bottom=2.5cm}

\begin{document}


\begin{center}
\textbf{ GCE May, 2019}
\vspace{8pt}

Jiamin JIAN
\end{center}

\vspace{12pt}

$\textbf{Exercise1:}$

Let $V$ be a normed vector space and $S$ a subset of $V$. Let $S^{c}$ be the complement of $S$. Let $x$ be in $S$ and $y$ be in $S^{c}$. The line segment $[x, y]$ is by definition the set
\begin{equation*}
    \{(1-t)x + t y : t \in [0, 1] \}.
\end{equation*}

Show that the intersection of $[x, y]$ and $\partial S$ is non empty, where $\partial S$ is the boundary of $S$ (by definition the boundary of $S$ is the set of points that are in the closure of $S$ and that are not in the interior of $S$).

\vspace{8pt}
$\textbf{Solution:}$

We want to prove that the intersection of $[x, y]$ and $\partial S$ is non empty, then we need to find a $t^{*} \in [0, 1]$, $\forall \delta >0$, $B((1-t^{*})x + t^{*}y, \delta) \bigcap S \neq \varnothing$ and $B((1-t^{*})x + t^{*}y, \delta) \bigcap S^{c} \neq \varnothing$, where $B((1-t^{*})x + t^{*}y, \delta) = \{(1-t)x + t y: |t - t^{*} | < \delta, t \in [0, 1] \}.$ Then we need to find that $t^{*}$. We define
\begin{equation*}
    Z = \{t: (1-t) x + t y \in S, t \in [0,1]\},
\end{equation*}
and we denote $t^{*} = \text{sup} {Z}$. And we denote $B((1-t^{*})x + t^{*}y, \delta) = B_{t^{*}, \delta}$. 

Firstly, we show that $B_{t^{*}, \delta} \bigcap S \neq \varnothing$. Since $t^{*} = \text{sup} Z$, by the definition of $t^{*}$ then we have $\forall \delta > 0, \exists \epsilon = \frac{\delta}{2}, \big(1 - (t^{*} - \epsilon)x \big) + (t^{*} - \epsilon) y \in S$. And since $|t^{*} - \epsilon - t^{*}| = \epsilon < \delta$, then $\big(1 - (t^{*} - \epsilon)x \big) + (t^{*} - \epsilon) y \in B_{t^{*}, \delta}$, such that $B_{t^{*}, \delta} \bigcap S \neq \varnothing$.

Secondly, we need verify that $B_{t^{*}, \delta} \bigcap S^{c} \neq \varnothing$. Suppose $B_{t^{*}, \delta} \bigcap S^{c} = \varnothing$, then we have that $B_{t^{*}, \delta} \subset S$. Since $t^{*} = \text{sup} Z$, by the definition of $t^{*}$ then we have $\forall \delta > 0, \exists \epsilon = \frac{\delta}{2}, \big(1 - (t^{*} + \epsilon)x \big) + (t^{*} + \epsilon) y \notin S$. And since $|t^{*} - \epsilon - t^{*}| = \epsilon < \delta$, then $\big(1 - (t^{*} + \epsilon)x \big) + (t^{*} + \epsilon) y \in B_{t^{*}, \delta}$. It is contradict with $B_{t^{*}, \delta} \subset S$, then we know that $B_{t^{*}, \delta} \bigcap S^{c} \neq \varnothing$.

Overall, we find $t^{*} \in [0, 1]$, $(1- t^{*})x + t^{*} y \in [x, y]$, $\forall \delta > 0$, we have $B_{t^{*}, \delta} \bigcap S \neq \varnothing$ and $B_{t^{*}, \delta} \bigcap S^{c} \neq \varnothing$, such that $(1- t^{*})x + t^{*} y \in \partial{S}$. So, we conclude that the intersection of $[x, y]$ and $\partial S$ is non empty.

\vspace{8pt}
$\textbf{Exercise2:}$

Let $(X, \mathcal{A}, \mu)$ be a measure space. Let $g$ be a measurable function defined on $X$. Set 
\begin{equation*}
    p_{g} (t) = \mu ({x \in X : |g(x)| > t}).
\end{equation*}

(i) If $f$ is in $L^{1}(X)$ show that there is a constant $C > 0$ such that $p_{f}(t) \leq \frac{C}{t}$.

(ii) Find a measurable function $h$ defined almost everywhere on $\mathbb{R}$ such that $\exists C > 0$, $p_{h} (t) \leq \frac{C}{t}$ and $h$ is not in $L^{1}(\mathbb{R})$.

\vspace{8pt}
$\textbf{Solution:}$

(i) Since $f \in L^{1}(X)$, then $\exists C > 0$, $\int_{X}^{} |f| \, d \mu \leq C < + \infty$. We can decompose the integral as following:
\begin{eqnarray*}
\int_{X}^{} |f| \, d \mu &=& \int_{X}^{} |f| \mathbb{I}_{\{|f| > t\}} \, d \mu + |f| \mathbb{I}_{\{|f| \leq t\}} \, d \mu  \\
            &=& \int_{X}^{} |f| \mathbb{I}_{\{|f| > t\}} \, d \mu + \int_{X}^{} |f| \mathbb{I}_{\{|f| \leq t\}} \, d \mu  \\
            &\geq & \int_{X}^{} |f| \mathbb{I}_{\{|f| > t\}} \, d \mu  \\
            & \geq & t \int_{X}^{} \mathbb{I}_{\{|f| > t\}} \, d \mu \\
            & = & t p_{f}(t).
\end{eqnarray*}
Then we have $t p_{f}(t) \leq C$, such that $p_{f}(t) \leq \frac{C}{t}$.

(ii) We suppose that 
\begin{equation*}
h(x) =
\left\{
             \begin{array}{cl}
             0, & x = 0 \\
             \frac{1}{|x|}, & x \neq 0,
             \end{array}
\right.
\end{equation*}
then $h(x) \notin L^{1}(\mathbb{R})$ since $\frac{1}{x} \notin L^{1}([0, + \infty))$. Since
\begin{equation*}
p_{t}(t) = \int_{\mathbb{R}}^{} \mathbb{I}_{\{|h| > t\}} \, d \mu  = \int_{\mathbb{R}}^{} \mathbb{I}_{\{|x| < \frac{1}{t} \}} \, d \mu  =  \int_{\{|x| < \frac{1}{t} \}}^{} \, d \mu,
\end{equation*}
hence we can set $C = 2$ and $p_{h} (t) \leq \frac{C}{t}$ and $h$ is not in $L^{1}(\mathbb{R})$.

\vspace{8pt}

$\textbf{Exercise3:}$

Let $\{f_{n}\} : [0, 1] \mapsto [0, \infty) $ be a sequence of functions, each of which is non-decreasing on the interval $[0, 1]$. Suppose the sequence is uniformly bounded in $L^{2}([0, 1])$. Show that there exists a sub sequence that converges in $L^{1}([0, 1])$.

\vspace{8pt}
$\textbf{Solution:}$

Since $f_{n}$ is non-decreasing, then for $x \in [0, 1]$, we have $\int_{x}^{1} f_{n}(y) \, d y \geq (1-x) f_{n}(x)$. On the other hand, since the sequence is uniformly bounded in $L^{2}([0, 1])$, we have $\forall n \in \mathbb{N}$, $\exists C > 0$, and $\| f_{n} \|_{2} \leq C$. And then we have
\begin{eqnarray*}
\int_{x}^{1} f_{n}(y) \, d y &=& \int_{0}^{1} f_{n}(y) \mathbb{I}_{[x, 1]} (y) \, d y  \\
            &\leq & \Big( \int_{0}^{1} f^{2}_{n}(y) \, d y \Big)^{\frac{1}{2}} \Big( \int_{0}^{1} \mathbb{I}^{2}_{[x, 1]} (y) \, d y \Big)^{\frac{1}{2}} \\
            &\leq & C (1-x)^{\frac{1}{2}}.
\end{eqnarray*}
Such that we have $(1-x) f_{n}(x) \leq C (1-x)^{\frac{1}{2}}$, then $f_{n}(x) \leq C (1-x)^{- \frac{1}{2}}$. Until now we find a type of function $f(x) = C (1-x)^{- \frac{1}{2}}$ that can control the sequence $f_{n}$, where $C$ is from the bound of $f_{n}$ in the $L^{2}([0, 1])$.

\vspace{8pt}

$\textbf{Exercise4:}$

Consider the sequence of functions $f_{n}: [0, 1] \mapsto \mathbb{R}$ where $f_{1}(x) = \sqrt{x}, f_{2}(x) = \sqrt{x + \sqrt{x}}, f_{3}(x) = \sqrt{x + \sqrt{x + \sqrt{x}}}$, and in general $f_{n}(x) = \sqrt{x + \sqrt{x + \sqrt{\dots + \sqrt{x}}}}$ with $n$ roots.

(i) Show that this sequence converges pointwise on $[0, 1]$ and find the limit function $f$ such that $f_{n} \rightarrow f$.

(ii) Does this sequence converge uniformly on $[0, 1]$? Prove or disprove uniform convergence.

\vspace{8pt}
$\textbf{Solution:}$

(i) Firstly, we show that the sequence $f_{n}(x)$ is non-decreasing for the fixed $x$. We use the mathematical induction. For the fixed $x \in [0, 1]$, when $k = 1$, since $f_{k} (x) = \sqrt{x}$ and $f_{k+1} (x) = \sqrt{x + \sqrt{x}}$, then $f_{k}(x) \leq f_{k+1}(x)$. We suppose when $k = n-1$, the formula $f_{k}(x) \leq f_{k+1}(x)$ holds, which is equivalent to $f_{n-1}(x) \leq f_{n}(x)$ . We want to verify $f_{n}(x) \leq f_{n+1}(x)$. Since $f_{n}(x) = \sqrt{x + f_{n-1}(x)}$ and $f_{n+1}(x) = \sqrt{x + f_{n}(x)}$, when $f_{n-1}(x) \leq f_{n}(x)$, we have $\sqrt{x + f_{n-1}(x)} \leq \sqrt{x + f_{n}(x)}$, such that $f_{n}(x) \leq f_{n+1}(x)$. So when $k = n$, the formula $f_{k}(x) \leq f_{k+1}(x)$ can also hold. Thus we know that the sequence $f_{n}(x)$ is non-decreasing for the fixed $x$.

Then, we show that the sequence $f_{n}(x)$ is uniformly bounded. We also use the mathematical induction. When $k =1$, $f_{k}(x) = \sqrt{x} < \sqrt{3}$. We suppose that when $k = n-1$, we have $f_{k}(x) < \sqrt{3}$. When $k =n$, $f_{n}(x) = \sqrt{f_{n-1}(x) + x} < \sqrt{\sqrt{3}+1} < \sqrt{3}$. Such that we get a uniform bound of sequence $f_{n}$.

Overall, since the sequence $f_{n}(x)$ is non-decreasing for the fixed $x$, and the sequence $f_{n}(x)$ has uniformly bound $\sqrt{3}$, then this sequence converges pointwise on $[0, 1]$. We suppose the sequence $f_{n}(x)$ converges pointwise on $[0, 1]$ to $f(x)$. Since $f_{n+1}(x) = \sqrt{x + f_{n}(x)}$, when $n \to \infty$, we have $f(x) = \sqrt{x + f(x)}$. So we can get $f^{2}(x) - f(x) - x = 0$, such that $f(x) = \frac{1 + \sqrt{1 + 4 x}}{2}$ as $f(x) \geq 0$. When $x =  0$, we have $f_{n} (x) = 0, \forall n$. Then we have
\begin{equation*}
f(x) =
\left\{
             \begin{array}{cl}
             0, & x = 0 \\
             \frac{1 + \sqrt{1 + 4 x}}{2}, & x \in (0, 1].
             \end{array}
\right.
\end{equation*}

(ii) Since for all $n \in \mathbb{N}$, $ f_{n}(x)$ is continuous, if the sequence $f_{n}(x)$ converge uniformly on $[0, 1]$ to $f(x)$, then $f(x)$ should be continuous. Since the $f(x)$ we get in (i) is not a continuous function, then this sequence $f_{n}(x)$ is not converge uniformly on $[0, 1]$.

\vspace{8pt}

$\textbf{Exercise5:}$

$S$ is a normed space, and we define $B_{1} = \{ x \in S: \|x\| \leq 1 \}$. Prove or disprove: $B_{1}$ is compact.

\vspace{8pt}
$\textbf{Solution:}$

The $B_{1}$ is not compact, we can find a counter example. We consider $S = l^{2}$ and $B_{1} = \{ x \in l^{2}: \|x\| = 1 \}$. 

Firstly, we can show that $B_{1}$ is bounded and closed. By the definition of $B_{1}$, we know that $B_{1}$ is bounded by 1. $\forall x, y \in B_{1}$, since $\|x\| \leq \|x - y \| + \|y\|$ and $\|y\| \leq \|x - y \| + \|x\|$, we have $| \|x\| - \|y\| | \leq \|x - y\|$, such that the norm is continuous from $l^{2}$ to $\mathbb{R}$. Since the image set $\{1\}$ is closed, then we know the inverse image of $\{1\}$ is also closed, which is actually $B_{1}$. So, $B_{1}$ is bounded and closed.

Next, we verify that $\exists \epsilon > 0$, $B_{1}$ cannot be covered by finitely many balls with radius $\epsilon$. We define $e_{i}$ as follow:
\begin{equation*}
e_{i,m} =
\left\{
             \begin{array}{cl}
             1, & m = i \\
             0, & m \neq i
             \end{array},
\right.
\end{equation*}
such that $e_{i} \in l^{2}$. Clearly, we have $\forall i, j$, if $i \neq j$,  we have  $\|e_{i} - e_{j} \| = \sqrt{2}$. Suppose $B_{1}$ can be covered by the finite balls with radius $\frac{\sqrt{2}}{2}$. Since $\{e_{i}\}_{i = 1}^{+ \infty}$ is infinity, hence at least one of such ball contains at least $e_{j}$ and $e_{k}$ with $j \neq k$. Let $x$ be the center of this ball, then we have $\|e_{j}  - e_{k}\| \leq \|e_{j} - x\| + \|e_{k} - x\| < \frac{\sqrt{2}}{2} + \frac{\sqrt{2}}{2} =  \sqrt{2}$. It is contradict with $\forall k, j$, if $k \neq j$,  we have  $\|e_{i} - e_{j} \| = \sqrt{2}$. Hence $\exists \epsilon > 0$, $B_{1}$ cannot be covered by finitely many balls with radius $\epsilon$. Then we know  that $B_{1}$ is  not compact.

\end{document}

%GCE of WPI
%by Jiamin JIAN

\documentclass[12pt,a4paper]{ctexart}
\usepackage{CJK}
\usepackage{lipsum}
\usepackage{amsmath}
\usepackage{geometry}
\usepackage{titlesec}
\usepackage{amssymb}
\usepackage{epsfig}
\usepackage{float}
\usepackage{graphicx}
\usepackage{tabularx}
\usepackage{longtable}
\usepackage{amstext}
\usepackage{blkarray}
\usepackage{amsfonts}
\usepackage{bbm}
\usepackage{listings}
\geometry{left=2.5cm,right=2.5cm,top=2.5cm,bottom=2.5cm}

\begin{document}


\begin{center}
\textbf{ GCE May, 2018}
\vspace{8pt}

Jiamin JIAN
\end{center}

\vspace{12pt}

$\textbf{Exercise 1:}$

Let $(X, \rho)$ be a metric space and $K_{n}$ a sequence of compact subsets of $X$ such that $K_{n+1} \subset K_{n}$. Set
\begin{equation*}
   d_{n} = \sup \{ \rho (x, y): x \in K_{n}, y \in K_{n} \}
\end{equation*}
Assuming that $d_{n}$ converges to zero show that $\bigcap_{n = 1}^{\infty} K_{n}$ is a singleton.

\vspace{8pt}

$\textbf{Solution:}$

Since $\lim_{n \to + \infty} d_{n} = 0$, it means the diameter of the intersection of the $K_{n}$ is zero. So, $\bigcap_{n = 1}^{\infty} K_{n}$ is either empty or consists of a single point. For any $n \in \mathbb{N}$, we pick an element $a_{n} \in K_{n}$. So we can get a point sequence $\{a_{n}\}$, and we have $\{a_{n}: n \in \mathbb{N}\} \in K_{1}$. Since $K_{1}$ is compact, then we know there exists a sub-sequence of $a_{n}$, which is denoted as $a_{n_{k}}$, converges to a point $a$. For any $n \in \mathbb{N}$, since each $K_{n}$ is compact, and $a$ is the limit of a sequence in $K_{n}$, we have $a \in K_{n}$. Thus $a \in \bigcap_{n = 1}^{\infty} K_{n}$. So we know that $\bigcap_{n = 1}^{\infty} K_{n}$ is a singleton.

\noindent\rule[0.25\baselineskip]{\textwidth}{0.5pt}

\vspace{8pt}
$\textbf{Exercise 2:}$

(i) Let $[a, b]$ be an interval in $\mathbb{R}$. If $\tilde{f}$ is continuous on $[a, b]$, $g$ is differentiable on $[a, b]$ and monotonic, and $g'$ is continuous on $[a, b]$, show that there is a $c$ in $[a, b]$, such that
\begin{equation*}
   \int_{a}^{b} \tilde{f}g = g(a) \int_{a}^{c} \tilde{f} + g(b) \int_{c}^{b} \tilde{f}.
\end{equation*}
$\textbf{Hint:}$ Introduce $F(x) = \int_{a}^{x} \tilde{f}$ and integral by parts.

(ii) Show that if $g$ is as specified above and $f$ is in $L^{1} ([a, b])$, there is a $c$ in $[a, b]$ such that 
\begin{equation*}
   \int_{a}^{b} f g = g(a) \int_{a}^{c} f + g(b) \int_{c}^{b} f.
\end{equation*}

\vspace{8pt}
$\textbf{Solution:}$

(i) Since $\tilde{f}$ is continuous on $[a, b]$, we can introduce $F(x) = \int_{a}^{x} \tilde{f}$, so we know that $F'(x) = \tilde{f}(x)$. Then through integral by parts, we have 
\begin{eqnarray*}
\int_{a}^{b} \tilde{f(x)} g(x) \, d x & = & \int_{a}^{b} g(x) \, d F(x) \\
& = & g(b) F(b) - g(a) F(a) - \int_{a}^{b} g'(x) F(x) \, d x  \\
& = & g(b) \int_{a}^{b} \tilde{f} (x) \, d x - g(a) \int_{a}^{a} \tilde{f} (x) \, d x - \int_{a}^{b} g'(x) F(x) \, d x  \\
& = &  g(b) \int_{a}^{b} \tilde{f} (x) \, d x - \int_{a}^{b} g'(x) F(x) \, d x.
\end{eqnarray*}

Since $g$ is differentiable on $[a, b]$ and monotonic, and $g'$ is continuous on $[a, b]$, we know that $g'$ is integrable in $[a, b]$ and $g'(x) \geq 0$ for all $x \in [a, b]$. By the mean value theorem for integral, there exists $c \in [a, b]$, and
\begin{equation*}
   \int_{a}^{b} g'(x) F(x) \, d x = F(c) \int_{a}^{b} g'(x) \, d x = F(c) (g(b) - g(a)).
\end{equation*}
Thus for this $c \in [a, b]$, we have
\begin{eqnarray*}
\int_{a}^{b} \tilde{f(x)} g(x) \, d x & = & g(b) \int_{a}^{b} \tilde{f} (x) \, d x - F(c) (g(b) - g(a)) \\
& = & g(b) \int_{a}^{b} \tilde{f} (x) \, d x - (g(b) - g(a)) \int_{a}^{c} \tilde{f} (x) \, d x \\
& = & g(b) \int_{a}^{b} \tilde{f} (x) \, d x - g(b) \int_{a}^{c} \tilde{f} (x) \, d x + g(a) \int_{a}^{c} \tilde{f} (x) \, d x  \\
& = &  g(b) \int_{c}^{b} \tilde{f} (x) \, d x + g(a) \int_{a}^{c} \tilde{f} (x) \, d x.
\end{eqnarray*}

(ii) Since $C_{c}([a, b])$ is dense in $L^{1}([a, b])$, then we know that for any $f \in L^{1}([0, 1])$, there exists a function sequence $\{f_{n}\} \subset C_{c}([a, b])$ and $\int_{a}^{b} |f_{n} - f| \to 0$ as $n \to + \infty$.
Since $g$ is differentiable on $[a,b]$ and monotonic, we know there exists $K > 0$, and $\forall x \in [a, b]$, we have $|g(x)| \leq K$. So, we have
\begin{equation*}
   \lim_{n \to + \infty} \int_{a}^{b} |g f - g f_{n}| \leq K \lim_{n \to + \infty} \int_{a}^{b}|f - f_{n}| = 0,
\end{equation*}
then by the conclusion we get from (i) we have
\begin{equation*}
   \int_{a}^{b} f g = \lim_{n \to + \infty} \int_{a}^{b} f_{n} g = \lim_{n \to + \infty} \Big{(} g(a) \int_{a}^{c_{n}} f_{n} + g(b) \int_{c_{n}}^{b} f_{n} \Big{)},
\end{equation*}
where $c_{n}$ is depends on $f_{n}$ for each n.

Since $\{c_{n}\} \subset [a, b]$ and $[a, b]$ is compact, there exists a subsequence of $\{c_{n}\}$, which is denoted as $\{c_{n_{k}}\}$, converges to $c$ and $c \in [a, b]$. Thus we have
\begin{eqnarray*}
\int_{a}^{b} f g & = & \lim_{k \to + \infty} \Big{(} g(a) \int_{a}^{c_{n_{k}}} f_{n_{k}} + g(b) \int_{c_{n_{k}}}^{b} f_{n_{k}} \Big{)} \\
& = & \lim_{k \to + \infty} \Big{(} g(a) \int_{a}^{c} f_{n_{k}} + g(a) \int_{c}^{c_{n_{k}}} f_{n_{k}} + g(b) \int_{c_{n_{k}}}^{c} f_{n_{k}} + g(b) \int_{c}^{b} f_{n_{k}} \Big{)} \\
& = &  g(a) \int_{a}^{c} f + g(b) \int_{c}^{b} f + \lim_{k \to + \infty} \Big{(} g(a) \int_{c}^{c_{n_{k}}} f_{n_{k}} +  g(b) \int_{c_{n_{k}}}^{c} f_{n_{k}} \Big{)} \\
& = & g(a) \int_{a}^{c} f + g(b) \int_{c}^{b} f.
\end{eqnarray*}

\noindent\rule[0.25\baselineskip]{\textwidth}{0.5pt}

\vspace{8pt}

$\textbf{Exercise 3:}$

Let $\{f_{n}\}$ be a sequence of functions $f_{n}: [0, 1] \rightarrow \mathbb{R}$.

(i) Define equicontinuity for this sequence.

(ii) Show that if each $f_{n}$ is differentiable on $[0, 1]$ and $|f'_{n}(x)| \leq 1$ for all $x$ in $[0, 1]$ and $n \in \mathbb{N}$, the sequence is equicontinuous.

(iii) Suppose the sequence is uniformly bounded and that (ii) holds. Show that $f_{n}$ has a subsequence which converges uniformly to a continuous function.

(iv) Show through an example that the limit may not be differentiable.

\vspace{8pt}
$\textbf{Solution:}$

(i) The definition of equicontinuity of sequence $\{f_{n}\}$ at point $x$ is as follows: $\forall \epsilon > 0, \exists \delta > 0$, such that $|x - y| < \delta$ and $\forall n \in \mathbb{N}$, we have $|f_{n}(x) - f_{n}(y)| < \epsilon$. And the definition of uniformly equicontinuity of sequence $\{f_{n}\}$ is as follows:
 $\forall x \in [0, 1], \forall \epsilon > 0, \exists \delta > 0$, such that $|x - y| < \delta$ and $\forall n \in \mathbb{N}$, we have $|f_{n}(x) - f_{n}(y)| < \epsilon$.

(ii) Since $f_{n}$ is differentiable on $[0, 1]$, by the mean value theorem, we know that $\forall x, y \in [0, 1]$, there exists a $c \in [x, y]$ and we have
\begin{equation*}
   |f_{n}(y) - f_{n}(x)| = |f'_{n}(c)| |y - x|.
\end{equation*}
Since $|f'_{n}(x)| \leq 1$ for all $x \in [0, 1]$ and $n \in \mathbb{N}$, then we have
\begin{equation*}
   |f_{n}(y) - f_{n}(x)| \leq |y - x|.
\end{equation*}
We set $\delta = \epsilon$, then for all $\epsilon > 0$, there exists $\delta = \epsilon$, such that when $|y - x| < \delta$, $\forall n \in \mathbb{N}$, we have $|f_{n}(y) - f_{n}(x)| < \epsilon$. So we know the sequence $\{f_{n}\}$ is equicontinuous.

(iii) By the Arzel$\grave{a}$-Ascoli theorem, we can get $f_{n}$ has a subsequence which converges uniformly to a continuous function directly. Next we can show the proof of Arzel$\grave{a}$-Ascoli theorem.

We enumerate $\{x_{i}\}_{i \in \mathbb{N}}$ as the rational number in $[0, 1]$. Since the sequence $\{f_{n}\}$ is uniformly bounded, then the set of points $\{f_{n}(x_{1})\}$ is bounded, by the Bolzano-Weierstrass theorem, there is a subsequence $\{f_{n1}(x_{1})\}$ converges. Repeating the same argument for the sequence points $\{f_{n1}(x_{2})\}$, there is a subsequence $\{f_{n2}\}$ of $\{f_{n1}\}$ such that $\{f_{n2}(x_{2})\}$ converges. By induction this process can be continued forever, and so there is a chain of subsequences
\begin{equation*}
   \{f_{n}\} \supset \{f_{n1}\} \supset \{f_{n2}\} \supset \cdots
\end{equation*}
Such that for each $k \in \mathbb{N}$, the subsequence$\{f_{nk}\}$ converges at point $x_{k}$. We choose the diagonal subsequence $\{f_{kk}\}$. Except for the first $n$ functions, $\{f_{kk}\}$ is a subsequence of the $n$th row $\{f_{nk}\}$. Therefore, the sequence $\{f_{kk}\}$ converges simultaneously on all $x_{n}$.

Next we need to show that $\{f_{kk}\}$ is converges uniformly on $[a, b]$. We just need to prove the uniform Cauchy criterion holds. Given any $\epsilon > 0$ and rational $x_{k} \in [0, 1]$, there is an integer $N(\epsilon, x_{k})$ such that when $n, m > N$, we have
\begin{equation*}
   |f_{nn}(x_{k}) - f_{mm}(x_{k})| < \frac{\epsilon}{3}.
\end{equation*}
Since $\bigcap (x_{k} - \frac{1}{n}, x_{k} + \frac{1}{n})$ covers the compact interval [0, 1], then by the Heine-Borel theorem there is a finite subcover, we denote the finite subcover as $U_{1}, \dots, U_{J}$. There exists an integer $K$ such that each open interval $U_{j}$, $1 \leq j \leq J$, contains a rational number $x_{k}$ with $1 \leq k \leq K$. Finally, for any $x \in [0, 1]$, there are $j$ and $k$ so that $x$ and $x_{k}$ belong to the same interval $U_{j}$. For this k, we have
\begin{eqnarray*}
|f_{nn}(x) - f_{mm}(x)| & \leq & |f_{nn}(x) - f_{nn}(x_{k})| + |f_{nn}(x_{k}) - f_{mm}(x_{k})| + |f_{mm}(x_{k}) - f_{mm}(x)| \\
& \leq & \frac{\epsilon}{3} + \frac{\epsilon}{3} + \frac{\epsilon}{3} = \epsilon
\end{eqnarray*}
for all $ N = \max\{N(\epsilon,x_{1}), \dots, N(\epsilon,x_{K})\}$ as $f_{n}$ is equicontinuous. So, for the subsequence $\{f_{kk}\}$, the uniform Cauchy criterion holds. Thus we know that $\{f_{kk}\}$ converges to a continuous function.



(iv) We denote $f_{n} (x) = \sqrt{(x - \frac{1}{2})^{2} + \frac{1}{n}}, \, x \in [0, 1]$. Since for all $n \in \mathbb{N}$ and $x \in [0, 1]$,
\begin{equation*}
   |f'_{n}(x)| = \Big{|} \frac{x - \frac{1}{2}}{\sqrt{(x - \frac{1}{2})^{2} + \frac{1}{n}}} \Big{|} < 1
\end{equation*}
and $f_{n} (x) = \sqrt{(x - \frac{1}{2})^{2} + \frac{1}{n}} < 2$, by the conclusion we get from (ii) and (iii), we know that the sequence $\{f_{n}\}$ is equicontinuous and it has a subsequence which converges uniformly to a continuous function. When $n \to + \infty$, we have $f_{n}(x) \to f(x) = |x - \frac{1}{2}|$, which is not differentiable. So, we know that the limit of this type sequence may not be differentiable.

\noindent\rule[0.25\baselineskip]{\textwidth}{0.5pt}

\vspace{8pt}

$\textbf{Exercise 4:}$

Let $f$ be a lebesgue measurable function such that 
\begin{equation*}
   \int_{0}^{1} f(x) e^{Kx} \, d x = 0
\end{equation*}
for all $K = 1, 2, 3, \dots$. Show that necessarily $f(x) = 0$ for almost every $0 \leq x \leq 1$. 




\end{document}

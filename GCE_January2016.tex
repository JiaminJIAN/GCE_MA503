%GCE of WPI
%by Jiamin JIAN

\documentclass[12pt,a4paper]{ctexart}
\usepackage{CJK}
\usepackage{lipsum}
\usepackage{amsmath}
\usepackage{geometry}
\usepackage{titlesec}
\usepackage{amssymb}
\usepackage{epsfig}
\usepackage{float}
\usepackage{graphicx}
\usepackage{tabularx}
\usepackage{longtable}
\usepackage{amstext}
\usepackage{blkarray}
\usepackage{amsfonts}
\usepackage{bbm}
\usepackage{listings}
\geometry{left=2.5cm,right=2.5cm,top=2.5cm,bottom=2.5cm}

\begin{document}


\begin{center}
\textbf{ GCE January, 2016}
\vspace{8pt}

Jiamin JIAN
\end{center}

\vspace{12pt}

$\textbf{Exercise 1:}$

Let $f_{n}$ be a sequence of continuous functions from $[0, 1]$ to $\mathbb{R}$ which is uniformly convergent. Let $x_{n}$ be in $[0, 1]$ such that $f_{n}(x_{n}) \geq f_{n}(x)$, for all $x$ in $[0, 1]$.

(i) Is the sequence $x_{n}$ convergent?

(ii) Show that the sequence $f_{n}(x_{n})$ is convergent.

\vspace{8pt}

$\textbf{Solution:}$

(i) No, the sequence $x_{n}$ may not convergent. We assume that $f_{n} (x) = 0$ for all $x \in [0, 1]$. And for any $k \in \mathbb{N}$ we set the sequence $x_{n}$ is
\begin{equation*}
x_{n} =
\left\{
             \begin{array}{cl}
             0, & n = 2k \\
             1, & n = 2k -1,
             \end{array}
\right.
\end{equation*}
Then we know that $x_{n} \in [0, 1]$ and $f_{n}(x_{n}) = 0 = f_{n}(x)$ for any $x \in [0, 1]$, but the sequence $x_{n}$ is not convergent.

(ii) We suppose $f_{n}$ is uniformly converges to $f$ on $[0, 1]$. Since $f_{n}$ is continuous, then $f$ is also a continuous function. For any $y \in [0, 1]$, there exist a $x$, such that $f(y) \leq f(x)$. And since $f_{n}$ is uniformly converges to $f$ on $[0, 1]$, for any $\epsilon > 0$, there exists a $N_{1} \in \mathbb{N}$, when $n > N_{1}$, for any $y \in [0, 1]$, we have
\begin{equation*}
    |f_{n}(y) - f(y)| < \epsilon,
\end{equation*}
which is equivalent to $f(y) - \epsilon < f_{n}(y) < f(y) + \epsilon$. We use the $x_{n}$ to substitute the $y$, then we have $f_{n}(x_{n}) \leq f(x_{n}) + \epsilon \leq f(x) + \epsilon$. 

On the other hand, for the above $x$, we have $f_{n}(x_{n}) \geq f_{n}(x)$. As $f_{n}$ is uniformly converges to $f$ on $[0, 1]$, for the above $\epsilon > 0$, there exists a $N_{2} \in \mathbb{N}$, when $n > N_{2}$, for the above $x$, we have $f_{n}(x) > f(x) - \epsilon$. And then we have $f_{n}(x_{n}) > f(x) - \epsilon$. Thus for the above $\epsilon$ and $x$, there exists a $N^{*}$, which is the biggest one we related, then when $n > N^{*}$, we have
\begin{equation*}
    f(x) - \epsilon < f_{n}(x_{n}) < f(x) + \epsilon .
\end{equation*}
So, we know that the sequence $f_{n}(x_{n})$ is convergent.



\noindent\rule[0.25\baselineskip]{\textwidth}{0.5pt}

\vspace{8pt}
$\textbf{Exercise 2:}$

Let $\mathbb{I}$ be the set of all irrational number $(\mathbb{I} \subset \mathbb{R})$.

(i) Using that $\mathbb{Q} = \mathbb{R} \setminus \mathbb{I}$ (the set of all rationals) is countable, show that given $\epsilon > 0$, there is a closed subset $F \subset \mathbb{I}$ such that $|\mathbb{I} \setminus F| < \epsilon$.

(ii) Is $F$ compact? Please explain why or why not.
 
\vspace{8pt}
$\textbf{Solution:}$

(i) We rearrange the rational number and denote it as $\{a_{n}\}_{n=1}^{\infty}$. It is a countable set. For $\epsilon > 0$, and for each $a_{n} \in \mathbb{Q}$, we can find an open set
\begin{equation*}
    a_{n} \in (a_{n} - \frac{\epsilon}{2^{n+1}}, a_{n} + \frac{\epsilon}{2^{n+1}}),
\end{equation*}
then we know that $\cup_{n=1}^{\infty} (a_{n} - \frac{\epsilon}{2^{n+1}}, a_{n} + \frac{\epsilon}{2^{n+1}}) $ is an open coverage of $\mathbb{Q}$, and
\begin{equation*}
    \Big{|} \cup_{n=1}^{\infty} (a_{n} - \frac{\epsilon}{2^{n+1}}, a_{n} + \frac{\epsilon}{2^{n+1}}) \Big{|} \leq \sum_{n=1}^{\infty} \frac{\epsilon}{2^{n}} = \epsilon.
\end{equation*}
We denote $S = \cup_{n=1}^{\infty} (a_{n} - \frac{\epsilon}{2^{n+1}}, a_{n} + \frac{\epsilon}{2^{n+1}})$, then $\mathbb{R} \setminus S \subset \mathbb{R} \setminus \mathbb{Q} = \mathbb{I}$. We set $F = \mathbb{R} \setminus S$, as $S$ is an open set, then $F$ is closed. And we have
\begin{equation*}
    |\mathbb{I} \setminus F| = |\mathbb{I}| - |\mathbb{R} \setminus S| = |\mathbb{I}| - |\mathbb{R} | + |S| < \epsilon.
\end{equation*}

(ii) No, $F$is not a compact set. Suppose $F$ is compact, then $F$ is closed and bounded, thus $F$ has finite measure. Since we have $(\mathbb{I} \setminus F) \cup F$, then there exists a $M > 0$ such that 
\begin{equation*}
    |\mathbb{I}| = |(\mathbb{I} \setminus F) \cup F| \leq |\mathbb{I} \setminus F| + |F| < \epsilon + M,
\end{equation*}
which is contradictory with $|\mathbb{I}| = \infty$. Thus $F$ is not compact.


\noindent\rule[0.25\baselineskip]{\textwidth}{0.5pt}

\vspace{8pt}

$\textbf{Exercise 3:}$

Find with proof:
\begin{equation*}
    \lim_{n \to \infty} \int_{0}^{1} \frac{1 + n x^{3}}{(1 + x^{2})^{n}} \, d x 
\end{equation*}

\vspace{8pt}
$\textbf{Solution:}$

For $x \in (0, 1)$, we denote $f_{n}(x) = \frac{1 + n x^{3}}{(1 + x^{2})^{n}}$. Firstly, for $x \in (0, 1)$, since $(1 + x^{2})^{n} \geq 1 + n x^{2}$, then we have
\begin{equation*}
    f_{n}(x) \leq \frac{1 + n x^{3}}{1 + n x^{2}} \leq 1 \in L^{1}((0, 1)).
\end{equation*}
And for $x \in (0, 1)$, since $(1 + x^{2})^{n} \geq \frac{1}{2}n(n-1) x^{4}$, we have
\begin{equation*}
    f_{n}(x) = \frac{1 + n x^{3}}{(1 + x^{2})^{n}} \leq \frac{2 + 2 n x^{3}}{n(n-1) x^{4}}  = \frac{\frac{2}{x^{4}}}{n(n - 1)} + \frac{\frac{1}{x}}{n-1},
\end{equation*}
so for any fixed $x \in (0, 1)$, we have $\lim_{n \to \infty} f_{n} (x) = 0$, thus we know that $f_{n}(x)$ converges to $0$ pointwise. By the dominate convergence theorem, we have
\begin{equation*}
    \lim_{n \to \infty} \int_{0}^{1} \frac{1 + n x^{3}}{(1 + x^{2})^{n}} \, d x = \int_{0}^{1} \lim_{n \to \infty} \frac{1 + n x^{3}}{(1 + x^{2})^{n}} \, d x = 0.
\end{equation*}

\vspace{8pt}

$\textbf{Exercise 4:}$

Let $(X, \mathcal{A}, \mu)$ be a measure space such that $\mu(X) = 1$. Let $f$ be in $L^{1}(X)$ such that $f \geq 0$ almost everywhere.

(i) show that
\begin{equation*}
    \lim_{p \to 0^{+}} \int_{}^{} f^{p} = \mu(\{x \in X:  f(x) > 0\})
\end{equation*}

(ii) If $\mu(\{x \in X:  f(x) > 0\}) < 1$, find
\begin{equation*}
    \lim_{p \to 0^{+}} \Big{(} \int_{}^{} f^{p} \Big{)}^{\frac{1}{p}}. 
\end{equation*}


\vspace{8pt}
$\textbf{Solution:}$

(i) Since
\begin{eqnarray*}
    \int_{X}^{} f^{p} \, d \mu & = & \int_{\{x \in X: f > 0\}}^{} f^{p} \, d \mu + \int_{\{x \in X: f = 0\}}^{} f^{p} \, d \mu \\
    & = & \int_{\{x \in X: f > 0\}}^{} f^{p} \, d \mu,
\end{eqnarray*}
as $f$ be in $L^{1}(X)$ and $f \geq 0$ almost everywhere, by the Fatou's lemma,
\begin{equation*}
    \mu(\{x \in X:  f(x) > 0\}) = \int_{}^{} \mathbb{I}_{\{x \in X: f > 0\}} (x) \, d \mu \leq \liminf_{p \to 0^{+}} \int_{\{x \in X: f > 0\}}^{} f^{p} \, d \mu.
\end{equation*}
On the other hand, we know that
\begin{eqnarray*}
    \int_{\{x \in X: f > 0\}}^{} f^{p} \, d \mu & = & \int_{\{x \in X: 0 < f < n\}}^{} f^{p} \, d \mu  + \int_{\{x \in X: f \geq n\}}^{} f^{p} \, d \mu \\
    & \leq & \int_{\{x \in X: f \geq n\}}^{} f^{p} \, d \mu + n^{p} \mu(\{x \in X:  f(x) > 0\}).
\end{eqnarray*}
For $0 < p < 1$, when $x \in \{x \in X:  f(x) > n\}$, we have $f^{p} < f$, thus we have
\begin{eqnarray*}
    \limsup_{p \to 0^{+}} \int_{\{x \in X: f > 0\}}^{} f^{p} \, d \mu & \leq &  \mu(\{x \in X:  f(x) > 0\}) + \limsup_{p \to 0^{+}} \int_{\{x \in X: f \geq n\}}^{} f^{p} \, d \mu \\
    & \leq & \mu(\{x \in X:  f(x) > 0\}) + \int_{\{x \in X: f \geq n\}}^{} f \, d \mu \\
    & \leq & \mu(\{x \in X:  f(x) > 0\}) + \int_{X}^{} f \, \mathbb{I}_{\{x \in X: f \geq n\}}(x) \, d \mu
\end{eqnarray*}
Since $f \cdot \mathbb{I}_{\{x \in X: f \geq n\}}(x) \leq f \in L^{1}(X)$ and $\lim_{n \to \infty} f \, \mathbb{I}_{\{x \in X: f \geq n\}}(x) = 0 $, by the dominate convergence theorem, we have
\begin{equation*}
    \limsup_{p \to 0^{+}} \int_{\{x \in X: f > 0\}}^{} f^{p} \, d \mu \leq \mu(\{x \in X:  f(x) > 0\}),
\end{equation*}
thus we know that
\begin{equation*}
    \lim_{p \to 0^{+}} \int_{}^{} f^{p} = \mu(\{x \in X:  f(x) > 0\}).
\end{equation*}

(ii) We denote $S = \{x \in X: f > 0\}$, then
\begin{eqnarray*}
    \int_{X}^{} f^{p} \, d \mu & = & \int_{\{x \in X: f > 0\}}^{} f^{p} \, d \mu + \int_{\{x \in X: f = 0\}}^{} f^{p} \, d \mu \\
    & = & \int_{S}^{} f^{p} \, d \mu.
\end{eqnarray*}
And we denote that $F(p) = \log (\int_{S}^{} f^{p} \, d \mu)$, then we know that
\begin{equation*}
    \lim_{p \to 0^{+}} \Big{(} \int_{}^{} f^{p} \Big{)}^{\frac{1}{p}} = \lim_{p \to 0^{+}} e^{\frac{F(p)}{p}}.
\end{equation*}
As $F(0) = \log(\mu(S))$, then we have
\begin{eqnarray*}
    \lim_{p \to 0^{+}} \Big{(} \int_{}^{} f^{p} \Big{)}^{\frac{1}{p}} & = & \lim_{p \to 0^{+}} \exp \Big{\{} \frac{F(p) - \log(\mu(S)) + \log(\mu(S))}{p} \Big{\}} \\
    & = & \lim_{p \to 0^{+}} (\mu(S))^{\frac{1}{p}} \exp \Big{\{} \frac{F(p) - \log(\mu(S))}{p - 0} \Big{\}}.
\end{eqnarray*}
As $F(p) = \log (\int_{S}^{} f^{p} \, d \mu)$, we have 
\begin{equation*}
    F^{'}(p) = \frac{\int_{S}^{} f^{p} \cdot \log f \, d \mu}{\int_{S}^{} f^{p} \, d \mu},
\end{equation*}
thus we have $F^{'}(0) = \frac{\int_{S}^{} \log f \, d \mu}{\mu(S)}$. Then we know that
\begin{eqnarray*}
    \lim_{p \to 0^{+}} \Big{(} \int_{}^{} f^{p} \Big{)}^{\frac{1}{p}} & = & \lim_{p \to 0^{+}} (\mu(S))^{\frac{1}{p}} \exp \Big{\{} \lim_{p \to 0^{+}} \frac{F(p) - F(0)}{p - 0} \Big{\}} \\
    & = & \lim_{p \to 0^{+}} (\mu(S))^{\frac{1}{p}} e^{F^{'}(0)} \\
    & = & 0
\end{eqnarray*}
as $\mu(S) < 1$.



\end{document}

%GCE of WPI
%by Jiamin JIAN

\documentclass[12pt,a4paper]{ctexart}
\usepackage{CJK}
\usepackage{lipsum}
\usepackage{amsmath}
\usepackage{geometry}
\usepackage{titlesec}
\usepackage{amssymb}
\usepackage{epsfig}
\usepackage{float}
\usepackage{graphicx}
\usepackage{tabularx}
\usepackage{longtable}
\usepackage{amstext}
\usepackage{blkarray}
\usepackage{amsfonts}
\usepackage{bbm}
\usepackage{listings}
\geometry{left=2.5cm,right=2.5cm,top=2.5cm,bottom=2.5cm}

\begin{document}


\begin{center}
\textbf{ GCE January, 2017}
\vspace{8pt}

Jiamin JIAN
\end{center}

\vspace{12pt}

$\textbf{Exercise 1:}$

Consider the sequence of functions $f_{n}$ defined on the non-negative reals by $f_{n} (x) = 2 n x P(x) e^{-n x^{2}}$, where $P$ is a polynomial function.

(i) Is $f_{n}$ pointwise convergent on $[0, + \infty)$? Is $f_{n}$ uniformly convergent on $[0, + \infty)$? Explain your answers to both questions.

(ii) Let $g_{n}$ be a sequence of continuous functions defined on $[0, + \infty)$ and valued in $\mathbb{R}$. Assume that each $g_{n}$ is in $L^{1}([0, + \infty))$ and that sequence $g_{n}$ is uniformly convergent to zero. Prove or disprove: $\lim_{n \to \infty} \int_{0}^{\infty} g_{n} = 0$. 

(iii) Determine (with proof) $\lim_{n \to \infty} \int_{0}^{\infty}  f_{n}$.
  
\vspace{8pt}

$\textbf{Solution:}$

(i) When $x = 0$, $f_{n} (x) = 0$ for any $n \in \mathbb{N}$. When $x > 0$, since
\begin{equation*}
    \lim_{n \to \infty} f_{n} (x) = \lim_{n \to \infty} 2 n x P(x) e^{-n x^{2}} = \lim_{n \to \infty} \frac{2 n x P(x)}{e^{n x^{x}}} = 0,
\end{equation*}
then for any $\epsilon > 0$, there exist a $N \in \mathbb{N}$, such that $n > N$ we have 
\begin{equation*}
    |2 n x P(x) e^{- n x^{2}} - 0| < \epsilon,
\end{equation*}
thus we know that $f_{n}$ converges to $f(x) = 0$ pointwise on $[0, \infty)$. But $f_{n}$ is not uniformly convergent to $f(x) = 0$. We suppose $P(x) = 1$, then we have $f_{n}(x) = 2 n x e^{-n x^{2}} $. When $x = \frac{1}{\sqrt{n}}$, 
\begin{equation*}
    f_{n} (x) = 2 n \frac{1}{\sqrt{n}} e^{- n \frac{1}{n}} = 2 \sqrt{n} e^{-1},
\end{equation*}
so we have
\begin{equation*}
    \sup_{x \in [0, \infty)} |f_{n} (x) - 0| \geq 2 \sqrt{n} e^{-1} \to \infty
\end{equation*}
when $n$ goes to $+ \infty$. Thus we know that $f_{n}$ is not uniformly converges on $[0, + \infty)$. 

(ii) The statement is not true. We suppose 
\begin{equation*}
g_{n}(x) =
\left\{
             \begin{array}{cl}
             \frac{4}{n^{2}} x, & x \in [0, \frac{n}{2}) \\
             \frac{4}{n} - \frac{4}{n^{2}} x, & x \in [\frac{n}{2}, n]  \\
             0, & x \in (n, + \infty),
             \end{array}
\right.
\end{equation*}
then we know that for any $n \in \mathbb{N}$,
\begin{equation*}
    \int_{[0, \infty)}^{} g_{n} (x) \, d x = \int_{0}^{\frac{n}{2}} \frac{4}{n^{2}} x \, d x + \int_{\frac{n}{2}}^{n} \frac{4}{n} - \frac{4}{n^{2}} x \, d x = 1,
\end{equation*}
so we know that $g_{n} (x) \in L^{1}([0, \infty))$. When $x \in [0, \frac{n}{2})$, $g_{n}(x) = \frac{4}{n^{2}} x \leq \frac{2}{n}$ and when $x \in [\frac{n}{2}, n]$, $g_{n} (x) = \frac{4}{n} - \frac{4}{n^{2}} x \leq \frac{2}{n}$, so we know that $g_{n}$ uniformly converges to $0$. But since for any $n \in \mathbb{N}$, $\int_{0}^{\infty} g_{n}(x) \, d x  = 1$, then we have
\begin{equation*}
    \lim_{n \to + \infty} \int_{[0, \infty)}^{} g_{n} (x) \, d x = \lim_{n \to \infty} 1 = 1.
\end{equation*}
Thus $\lim_{n \to \infty} \int_{0}^{\infty} g_{n} = 0$ can not hold.

(iii) We denote $y = n x^{2}$, then we have
\begin{equation*}
    \int_{0}^{\infty} 2 n x P(x) e^{- n x^{2}} \, d x = \int_{0}^{\infty} e^{-y} P \Big{(} \sqrt{\frac{y}{n}} \Big{)} \, d y.
\end{equation*}
Since $P(x)$ is a polynomial function, for any fixed $y$, when $n \to \infty$, $P(\sqrt{\frac{y}{n}}) \to P(0)$ and then $e^{-y} P(\sqrt{\frac{y}{n}}) \to e^{-y} P(0)$. Since $P(x)$ is a polynomial function, there exist a $M > 0$, such that when $y \in [M, \infty)$, $e^{-y} P(\sqrt{\frac{y}{n}}) < \frac{1}{y^{2}}$, then we have
\begin{eqnarray*}
\lim_{n \to \infty} \int_{0}^{\infty} f_{n} & = & \lim_{n \to \infty} \int_{0}^{M} f_{n} + \lim_{n \to \infty} \int_{M}^{\infty} f_{n} \\
& = & \lim_{n \to \infty} \int_{0}^{M} e^{-y} P \Big{(} \sqrt{\frac{y}{n}} \Big{)} \, d y + \lim_{n \to \infty} \int_{M}^{\infty} e^{-y} P \Big{(} \sqrt{\frac{y}{n}} \Big{)} \, d y .
\end{eqnarray*}
Since $P(x)$ is a polynomial function, then $P(\sqrt{\frac{y}{n}})$ is continuous on $y \in [0,M]$, then we have when $y \in [0, M]$,
\begin{equation*}
    \Big{|} e^{-y} P \Big{(} \sqrt{\frac{y}{n}} \Big{)} \Big{|} \leq e^{-y} \|P\|_{\infty}.
\end{equation*}
Since $e^{-y} \|P\|_{\infty} \in L^{1}([0, M])$ and $\frac{1}{y^{2}} \in L^{1}([M, + \infty))$, by the dominate convergence theorem, we have
\begin{equation*}
    \lim_{n \to \infty} \int_{0}^{M} e^{-y} P \Big{(} \sqrt{\frac{y}{n}} \Big{)} \, d y = \int_{0}^{M} e^{-y} P(0) \, d y = P(0) (1 - e^{- M}),
\end{equation*}
and
\begin{equation*}
    \lim_{n \to \infty} \int_{M}^{\infty} e^{-y} P \Big{(} \sqrt{\frac{y}{n}} \Big{)} \, d y = \int_{M}^{\infty} e^{-y} P(0) \, d y = P(0) e^{- M}.
\end{equation*}
Thus we know that
\begin{eqnarray*}
\lim_{n \to \infty} \int_{0}^{\infty} f_{n} & = &  \lim_{n \to \infty} \int_{0}^{M} e^{-y} P \Big{(} \sqrt{\frac{y}{n}} \Big{)} \, d y + \lim_{n \to \infty} \int_{M}^{\infty} e^{-y} P \Big{(} \sqrt{\frac{y}{n}} \Big{)} \, d y \\
& = & P(0) (1 - e^{- M}) + P(0) e^{- M} \\
& = & P(0).
\end{eqnarray*}

\newpage


$\textbf{Exercise 2}  (\text{all answers require proofs: })$

Let $f_{n}$ be the sequence in $L^{2}(\mathbb{R})$ defined by $f_{n} = \mathbb{I}_{[n, n+1]}$.

(i) Let $g$ be in $L^{2}(\mathbb{R})$. Does $\int_{}^{} f_{n} g$ have a limit as $n$ tends to infinity?

(ii) Does the sequence $f_{n}$ converge in $L^{2}(\mathbb{R})$?

\vspace{8pt}
$\textbf{Solution:}$

(i) Firstly we show that $f_{n} = \mathbb{I}_{[n, n+1]} (x)$ converges to $f(x) = 0$ pointwise on $\mathbb{R}$. Since 
\begin{equation*}
    |f_{n} - f| = |\mathbb{I}_{[n, n+1]} (x) - 0| = \mathbb{I}_{[n, n+1]} (x), 
\end{equation*}
for any fixed $x \in \mathbb{R}$, $\forall \epsilon > 0$,we can find a $N = [x] + 1$, such that $n > N$, we have 
\begin{equation*}
    |f_{n} - f| = \mathbb{I}_{[n, n+1]} (x) = 0 < \epsilon.
\end{equation*}
Thus we know that $f_{n}$ converges to $f(x) = 0$ pointwisely on $\mathbb{R}$. Since  
\begin{equation*}
    \Big{|} \int_{\mathbb{R}}^{} f_{n} g \, d x \Big{|} \leq \int_{\mathbb{R}}^{} | f_{n} g | \, d x = \int_{n}^{n + 1} | g (x) | \, d x,
\end{equation*}
by Cauchy-Schwarz inequality, we have
\begin{eqnarray*}
\Big{|} \int_{\mathbb{R}}^{} f_{n} g \, d x \Big{|} & \leq & \int_{n}^{n + 1} | g (x) | \, d x \\
& \leq & \Big{(} \int_{n}^{n+1} |g|^{2} \, d x \Big{)}^{\frac{1}{2}} \Big{(} \int_{n}^{n+1} 1^{2} \, d x \Big{)}^{\frac{1}{2}} \\
& = & \Big{(} \int_{n}^{n+1} |g|^{2} \, d x \Big{)}^{\frac{1}{2}} \\ 
& = & \Big{(} \int_{\mathbb{R}}^{} |g|^{2} \mathbb{I}_{[n, n+1]} (x) \, d x \Big{)}^{\frac{1}{2}}.
\end{eqnarray*}
Since $|g|^{2} \mathbb{I}_{[n, n+1]} (x) \leq |g(x)|^{2}$ and since $g \in L^{2}(\mathbb{R})$, we have $\int_{\mathbb{R}}^{} |g(x)|^{2} \, d x < + \infty$, then we know that $|g(x)|^{2} \in L^{1}(\mathbb{R})$, by the dominate convergence theorem, we have
\begin{eqnarray*}
\lim_{n \to \infty} \Big{|} \int_{\mathbb{R}}^{} f_{n} g \, d x \Big{|}^{2} & \leq & \lim_{n \to \infty} \Big{(} \int_{\mathbb{R}}^{} |g|^{2} \mathbb{I}_{[n, n+1]} (x) \, d x \Big{)} \\
& = & \int_{\mathbb{R}}^{} \lim_{n \to \infty} \big{(} |g|^{2} \mathbb{I}_{[n, n+1]} (x)  \big{)} \, d x.
\end{eqnarray*}
Since $f_{n} = \mathbb{I}_{[n, n+1]} (x)$ converges to $f(x) = 0$ pointwisely on $\mathbb{R}$, we can show that $|g|^{2} \mathbb{I}_{[n, n+1]} (x)$ also converges to $f(x) = 0$ pointwisely on $\mathbb{R}$, then we have
\begin{equation*}
    \lim_{n \to \infty} \Big{|} \int_{\mathbb{R}}^{} f_{n} g \, d x \Big{|}^{2} \leq \int_{\mathbb{R}}^{} \lim_{n \to \infty} \big{(} |g|^{2} \mathbb{I}_{[n, n+1]} (x)  \big{)} \, d x = 0.
\end{equation*}
Thus we know that $\lim_{n \to \infty} \int_{\mathbb{R}}^{} f_{n}(x) g(x) \, d x = 0$.

(ii) Since $f_{n} = \mathbb{I}_{[n, n+1]} (x)$ converges to $f(x) = 0$ pointwisely on $\mathbb{R}$, but 
\begin{equation*}
    \int_{\mathbb{R}}^{} |f_{n}(x) - 0|^{2} \, d x = \int_{\mathbb{R}}^{} f_{n}^{2} \, d x = \int_{n}^{n + 1} 1 \, d x = 1,
\end{equation*}
we know that $f_{n}$ does not converges to $f(x) = 0$ in $L^{2}(\mathbb{R})$. Since $f_{n} = \mathbb{I}_{[n, n+1]} (x)$ converges to $f(x) = 0$ pointwisely on $\mathbb{R}$, then we can know that the sequence $f_{n}$ does not converge in $L^{2}(\mathbb{R})$.

\vspace{8pt}

$\textbf{Exercise 3:}$

Let $X$ be a matrix space. For any subset $A$ of $X$, we denote by $\bar{A}$ the closure of $A$ and $\mathring{A}$ the union of all open subsets contained in $A$. We set $\partial A = \bar{A} \setminus \mathring{A} $.

(i) Show that $A$ is closed if and only if $\partial A \subset A$.

(ii) Show that $A$ is open if and only if $\partial A \cap A = \emptyset$.

(iii) Is the identity $\partial (\partial B) = \partial B$ valid for all subsets $B$ of $X$.
  
(iv) Show that if $A$ is closed then $\partial(\partial A) = \partial A$.

\vspace{8pt}
$\textbf{Solution:}$

(i) When $A$ is closed, we have $A = \bar{A}$, since $\mathring{A}$ the union of all open subsets contained in $A$, then $\mathring{A} \subset A$. Thus we have $\partial A = \bar{A} \setminus \mathring{A} = A \setminus \mathring{A}  \subset A$ as $\mathring{A} \subset A$.

When $\partial A \subset A$, we have $\partial A \cup A \subset A \cup A = A$, then we know that $\bar{A} \subset A$. Since $A \subset \bar{A}$, we can get $\bar{A} = A$, thus $A$ is closed.

(ii) When $A$ is open, since $\mathring{A}$ the union of all open subsets contained in $A$, then we have $A \subset \mathring {A}$. And we know that $\mathring{A} \subset A$, then we can get $A = \mathring{A}$. As $\partial A = \bar{A} \setminus \mathring{A} $, we have $\partial A = \bar{A} \setminus A $, then it is obviously that $\partial A \cap  A  = \emptyset$.

When $\partial A \cap A = \emptyset$, we suppose $A$ is not an open set, then there exists a element $x \in A$ such that no open set containing $x$ is a subset of $A$. Since $\mathring{A}$ the union of all open subsets contained in $A$, we have $x \notin \mathring{A}$. And as $x \in A$, we know that $x \in \bar{A}$, then we have $x \in \bar{A} \setminus \mathring{A} = \partial A$. Then we can get $x \in  \partial A \cap A $, it is contradict with the condition we have. So, the statement that $A$ is not an open set is wrong. Thus we have $A$ is an open set.

(iii) No, the statement is not true. We suppose $B = \mathbb{Q} \cap [0, 1]$, which represents the rational number in the interval $[0, 1]$. Then we have $\partial B = [0, 1]$ and $\partial (\partial B) = \{0, 1\}$, which is not equal to $\partial B$.

(iv) Since $\bar{A}$ is closed and $\mathring{A}$ is open, we have $\partial A = \bar{A} \setminus \mathring{A} $ is closed, then we can get $\overline{\partial A} = \partial A$. By the definition of $\partial A$, we have $\partial (\partial A) = \overline{\partial A} \setminus \mathring{\partial A} = \partial A \setminus \mathring{\partial A} \subset \partial A$. Next we need to show that $\partial A \subset \partial (\partial A) = \partial A \setminus \mathring{\partial A}$, then we just need to prove that $\mathring{\partial A} = \emptyset$ when $A$ is closed. 

When $A$ is closed, since $\partial A = \bar{A} \setminus \mathring{A} = A \setminus \mathring{A}$. As $A \setminus \mathring{A} \subset A$, then we have $\mathring{\partial A} \subset \mathring{A}$. And since the union of subsets in $(A \setminus \mathring{A})$ is the subset of $ A \setminus \mathring{A}$, we have $\mathring{\partial A} \subset A \setminus \mathring{A}$. Then we know that $\mathring{\partial A} \subset \mathring{A}$ and $\mathring{\partial A} \subset A \setminus \mathring{A}$. Thus we can get $\mathring{\partial A} \subset \mathring{A} \cap (A \setminus \mathring{A}) = \emptyset$. So, we have showed that $\mathring{\partial A} = \emptyset$. In conclusion, we have $\partial(\partial A) = \partial A$ when $A$ is closed.

\newpage



$\textbf{Exercise 4:}$

Let $X$ be a measure space, $f_{n}$ a sequence in $L^{1} (X)$ and $f$ an element of $L^{1}(X)$ such that $f_{n}$ converges to $f$ almost everywhere and $\lim_{n \to \infty} \int_{}^{} |f_{n}| = \int_{}^{} |f|$. Show that $\lim_{n \to \infty} \int_{}^{} |f_{n} - f| = 0$.

\vspace{8pt}
$\textbf{Solution:}$

Since $|f_{n} - f| \leq |f_{n}| + |f|$ holds on $X$, we know that $|f_{n}| + |f| - |f_{n} - f|$ is a non-negative function. By the Fatou's lemma, we have
\begin{equation*}
    \int_{}^{} \lim_{n \to \infty} (|f_{n}| + |f| - |f_{n} - f|)  \leq \liminf_{n \to \infty} \int_{}^{} (|f_{n}| + |f| - |f_{n} - f|).
\end{equation*}
Since $f_{n}$ converges to $f$ almost everywhere, then we know that $|f_{n}|$ converges to $|f|$ almost everywhere. Thus we have
\begin{equation*}
    \lim_{n \to \infty} (|f_{n}| + |f| - |f_{n} - f|) = 2 |f|.
\end{equation*}
Then we can get that
\begin{eqnarray*}
\int_{}^{} 2 |f| & \leq & \liminf_{n \to \infty} \int_{}^{} (|f_{n}| + |f| - |f_{n} - f|) \\
& \leq & \liminf_{n \to \infty} \int_{}^{} (|f_{n}| + |f|) - \limsup_{n \to \infty}  \int_{}^{} (|f_{n} - f|) \\
& = & \int_{}^{} 2 |f| - \limsup_{n \to \infty}  \int_{}^{} (|f_{n} - f|),
\end{eqnarray*}
so we have
\begin{equation*}
    \limsup_{n \to \infty}  \int_{}^{} (|f_{n} - f|) \leq 0.
\end{equation*}
On the other hand, we have
\begin{equation*}
    0 \leq \liminf_{n \to \infty}  \int_{}^{} (|f_{n} - f|)
\end{equation*}
as $|f_{n} - f| \geq 0$. Thus we know that
\begin{equation*}
    \limsup_{n \to \infty}  \int_{}^{} (|f_{n} - f|) \leq 0 \leq \liminf_{n \to \infty}  \int_{}^{} (|f_{n} - f|),
\end{equation*}
which is equivalent to
\begin{equation*}
    \limsup_{n \to \infty}  \int_{}^{} (|f_{n} - f|) = \liminf_{n \to \infty}  \int_{}^{} (|f_{n} - f|) = 0.
\end{equation*}
So we have
\begin{equation*}
    \lim_{n \to \infty} \int_{}^{} |f_{n} - f| = 0.
\end{equation*}



\end{document}

\documentclass[12pt]{article}
\usepackage{amsmath, amssymb, amscd, amsthm, amsfonts,booktabs}
\usepackage{graphicx}
\usepackage{hyperref}
\usepackage{float}

\oddsidemargin 0pt
\evensidemargin 0pt
\marginparwidth 40pt
\marginparsep 10pt
\topmargin -20pt
\headsep 10pt
\textheight 8.7in
\textwidth 6.2in
\linespread{1.2}

\title{GCE, MA503}
\author{Jiamin Jian \\
Worcester polytechnic institute}
\date{}

\newtheorem{theorem}{Theorem}
\newtheorem{lemma}[theorem]{Lemma}
\newtheorem{conjecture}[theorem]{Conjecture}

\newcommand{\rr}{\mathbb{R}}

\newcommand{\al}{\alpha}
\DeclareMathOperator{\conv}{conv}
\DeclareMathOperator{\aff}{aff}

\begin{document}

\maketitle

\section{GCE August, 2014}

$\textbf{Exercise 1:}$

(i) Suppose that $f: \mathbb{R} \to \mathbb{R}$ is bounded. Given an example, with proof, of such a function $f$ whose improper Riemann integral on $(-\infty, \infty)$ exists and finite, but which is not in $L^{1}(\mathbb{R})$.

\vspace{4pt}

(ii) Suppose $- \infty < a < b < \infty$. Prove that if the proper Riemann integral of a function $g$ on $[a, b]$ exists, then the Lebesgue integral of $g$ on $[a, b]$ exists and equals the value of the proper Riemann integral.

\vspace{8pt}

$\textbf{Solution:}$

(i) We set
\begin{equation*}
    f(x) = \frac{\sin x}{x} \mathbb{I}_{[0, \infty)} (x),
\end{equation*}
and we want to show the integral of $f(x)$ on $\mathbb{R}$ is converges by the Cauchy convergence theorem for the improper Riemann integral. For any $A_{2} > A_{1} > 0$, we have
\begin{equation*}
    \int_{A_{1}}^{A_{2}} \frac{\sin x}{x} \, d x = \frac{\cos A_{1}}{A_{1}} - \frac{\cos A_{2}}{A_{2}} - \int_{A_{1}}^{A_{2}} \frac{\cos x}{x^{2}} \, d x,
\end{equation*}
then we know that
\begin{equation*}
    \Big{|} \int_{A_{1}}^{A_{2}} \frac{\sin x}{x} \, d x \Big{|} \leq \frac{1}{A_{1}} + \frac{1}{A_{2}} + \int_{A_{1}}^{A_{2}} \frac{1}{x^{2}} \, d x = \frac{2}{A_{1}}.
\end{equation*}
For any $\epsilon > 0$, we set $A = \frac{2}{\epsilon}$, when $A_{2} > A_{1} > A$, we have
\begin{equation*}
    \Big{|} \int_{A_{1}}^{A_{2}} \frac{\sin x}{x} \, d x \Big{|} \leq \frac{2}{A_{1}} < \frac{2}{A} < \epsilon,
\end{equation*} 
thus we know that $\int_{0}^{\infty} \frac{\sin x}{x} \, d x$ converges. Next we show that $\int_{0}^{\infty} \frac{\sin x}{x} \, d x = \frac{\pi}{2}$. We have
\begin{eqnarray*}
    \lim_{a \to \infty} \int_{0}^{a} \frac{\sin t}{t} \, d t & = & \lim_{a \to \infty} \int_{0}^{\infty} e^{- t x} \sin x \, d x \, d t \\
    & = & \int_{0}^{\infty} \int_{0}^{\infty} e^{- t x} \sin x \, d x \, d t  \\
    & =: & \int_{0}^{\infty} I(t) \, d t,
\end{eqnarray*}
and since
\begin{equation*}
    I(t) = \int_{0}^{\infty} e^{- t x} \sin x \, d x = 1 - t^{2} I(t),
\end{equation*}
we know that $I(t) = \frac{1}{1+ t^{2}}$ and
\begin{equation*}
    \lim_{a \to \infty} \int_{0}^{a} \frac{\sin t}{t} \, d t = \int_{0}^{\infty} \frac{1}{1+ t^{2}} \, d t = \frac{\pi}{2}.
\end{equation*}
Next we need to show that $f(x)$ is not in $L^{1}(\mathbb{R})$. Let $N \in \mathbb{N}$ and $N > 1$, we have
\begin{eqnarray*}
    \int_{0}^{2 \pi N} \Big{|} \frac{\sin x}{x} \Big{|} \, d x & = & \sum_{n=0}^{N-1} \int_{2n \pi}^{2 \pi (n+1)} \Big{|} \frac{\sin x}{x} \Big{|} \, d x  \\
    & \geq & \sum_{n=0}^{N-1} \frac{1}{2(n+1) \pi} \int_{2n \pi}^{2 \pi (n+1)} |\sin x| \, d x \\
    & = & \sum_{n=0}^{N-1} \frac{1}{2(n+1) \pi} \int_{0}^{2 \pi} |\sin x| \, d x \\
    & = & \sum_{n=0}^{N-1} \frac{2}{(n+1) \pi}.
\end{eqnarray*}
Let $N \to \infty$, we know that $\int_{0}^{\infty} |\frac{\sin x}{x}| \, d x $ diverges, so $f(x)$ is not in $L^{1}(\mathbb{R})$ but improper Riemann integral of $f(x)$ on $(-\infty, \infty)$ exists and $f(x)$ is finite.

\vspace{8pt}

(ii) Riemann integral is defined for functions $g$ on a closed and bounded interval $[a, b]$ as follows: for any partition $P = \{ a = x_{0} < x_{1} < \dots < x_{n} = b\}$, the corresponding lower sum $L(g, P)$ and upper sum $U(g, P)$ are defined by
\begin{equation*}
    L(g, P) = \sum_{i = 1}^{n} \inf_{x \in [x_{i-1}, x_{i}]} g(x) (x_{i} - x_{i-1})
\end{equation*}
\begin{equation*}
    U(g, P) = \sum_{i = 1}^{n} \sup_{x \in [x_{i-1}, x_{i}]} g(x) (x_{i} - x_{i-1})
\end{equation*}
Function $g$ is Riemann integrable if $\sup_{P} L(g, P) = \inf_{P} U(g, P)$, and the integral $\int_{a}^{b} f(x) \, d x$ then equals to this common value. For every partition $P$ , define the functions
\begin{equation*}
    \phi_{g, P} = \sum_{i = 1}^{n} \inf_{x \in [x_{i-1}, x_{i}]} g(x), \quad \text{if} \, \,   x \in (x_{i-1}, x_{i})
\end{equation*}
\begin{equation*}
    \psi_{g, P} = \sum_{i = 1}^{n} \sup_{x \in [x_{i-1}, x_{i}]} g(x), \quad \text{if} \, \,  x \in (x_{i-1}, x_{i})
\end{equation*}
At the nodes $x_{i}$, the functions $\phi_{g, P}$ and $\psi_{g, P}$ are equal to $0$. Then $\phi_{g, P}$ and $\psi_{g, P}$ are step functions, and by definition, the lower and upper sums are their integrals,
\begin{equation*}
     L(g, P) = \int \phi_{g, P} , \quad  U(g, P) = \int \psi_{g, P} , 
\end{equation*}
with respect to Lebesgue measure and 
\begin{equation*}
    \phi_{g, P} \leq g \leq \psi_{g, P}
\end{equation*}
except at the nodes $x_{i}$.

It is known from the theory of Riemann integration that if $g$ is Riemann integrable, then there exists a sequence of partitions $P_{k}$ such that
\begin{equation*}
    \int_{a}^{b} f(x) \, d x = \lim_{k \to \infty} L(g, P) = \lim_{k \to \infty} U(g, P)
\end{equation*}
and $P_{k+1}$ is a refinement of $P_{k}$, thus
\begin{equation*}
    \phi_{g, P_{k}} \leq \phi_{g, P_{k+1}}  \leq g \leq \psi_{g, P_{k+1}} \leq \psi_{g, P_{k}} 
\end{equation*}
except at the nodes of the partitions $P_{k}$, which is a countable set. Hence
\begin{eqnarray*}
    \int |\phi_{g, P_{k+m}} - \phi_{g, P_{k}} | & = & \int \phi_{g, P_{k+m}} - \phi_{g, P_{k}}   \\
    & = & \int \phi_{g, P_{k+m}} - \int \phi_{g, P_{k}}  \\
    & = & L(g, P_{k+m}) - L(g, P_{k}) \\
    & = & |L(g, P_{k+m}) - L(g, P_{k})|.
\end{eqnarray*}
Since the sequence $\{ L (g, P_{k}) \}$ converges, it is Cauchy sequence in $\mathbb{R}$, and, consequently, $\{\phi_{g, P_{k}} \}$ is $L^{1}$ Cauchy sequence of step maps. Similarly,  $\{\psi_{g, P_{k}} \}$ is $L^{1}$ Cauchy sequence of step maps. So we have $\{\phi_{g, P_{k}} \}$ and $\{\psi_{g, P_{k}} \}$ converge a.e. on [a, b], and since $\phi_{g, P} \leq g \leq \psi_{g, P}$ a.e., they converge to $f$ a.e. Thus the limits of the sequences of the integrals of the step maps $\phi_{g, P}$ and $\psi_{g, P}$ equal to the Lebesgue integral of $f$. Since the integrals of the step maps equal to the lower and upper Riemann sums, whose limit is the Riemann integral, the Riemann integral equals to the Lebesgue integral.


\noindent\rule[0.25\baselineskip]{\textwidth}{0.5pt}

\vspace{8pt}
$\textbf{Exercise 2:}$

Let $f_{n}$ be a sequence of measurable functions from $[0, 1]$ to $\mathbb{R}$. Assume that each function $f_{n}$ is finite almost everywhere. Show that $f_{n}$ converges in measure to zero if and only if 
\begin{equation*}
    \lim_{n \to \infty} \int_{0}^{1} \frac{|f_{n}|}{1 + |f_{n}|} = 0
\end{equation*}

\textbf{Hint:} Recall that by definition $f_{n}$ converges in measure to $f$ if and only if, given any $\epsilon > 0$,
\begin{equation*}
    \lim_{n \to \infty} |\{ |f_{n} - f| > \epsilon \}| = 0.
\end{equation*}

\vspace{8pt}
$\textbf{Solution:}$

Firstly suppose that $f_{n} \to 0$ in measure, for any fixed $\epsilon > 0$, we have
\begin{eqnarray*}
    \int_{0}^{1} \frac{|f_{n}|}{1 + |f_{n}|} \, d \mu & = & \int_{\{|f_{n}| \geq \epsilon\} \cap [0, 1]}^{} \frac{|f_{n}|}{1 + |f_{n}|} \, d \mu + \int_{\{|f_{n}| < \epsilon\} \cap [0, 1]}^{} \frac{|f_{n}|}{1 + |f_{n}|} \, d \mu \\
    & \leq & \mu({|f_{n}| \geq \epsilon}) + \epsilon \mu(\{|f_{n}| \leq \epsilon\} \cap [0, 1]) \\
    & \leq & \mu({|f_{n}| \geq \epsilon}) + \epsilon,
\end{eqnarray*}
thus we know that $\limsup_{n \to \infty} \int_{0}^{1} \frac{|f_{n}|}{1 + |f_{n}|} \, d \mu \leq \epsilon $. Let $\epsilon \to 0$, we have $\lim_{n \to \infty} \int_{0}^{1} \frac{|f_{n}|}{1 + |f_{n}|} \, d \mu = 0$.

On the other hand, suppose $\lim_{n \to \infty} \int_{0}^{1} \frac{|f_{n}|}{1 + |f_{n}|} \, d \mu = 0$, for any $\epsilon > 0$, we have
\begin{eqnarray*}
     \mu(|f_{n}| \geq \epsilon) & = & \int_{|f_{n}| \geq \epsilon}^{} 1 \, d \mu  \\
     & = & \frac{1 + \epsilon}{\epsilon} \int_{|f_{n}| \geq \epsilon}^{} \frac{\epsilon}{1 + \epsilon} \, d \mu \\
    & \leq & \frac{1 + \epsilon}{\epsilon} \int_{0}^{1} \frac{|f_{n}|}{1 + |f_{n}|} \, d \mu,
\end{eqnarray*}
thus when $n \to \infty$, we have
\begin{equation*}
    \lim_{n \to \infty} \mu(|f_{n}| \geq \epsilon) \leq  \lim_{n \to \infty} \frac{1 + \epsilon}{\epsilon} \int_{0}^{1} \frac{|f_{n}|}{1 + |f_{n}|} \, d \mu = 0.
\end{equation*}
Hence we know that $\lim_{n \to \infty} \mu(|f_{n}| \geq \epsilon) = 0$ and $f_{n}$ converges in measure to $0$.


\noindent\rule[0.25\baselineskip]{\textwidth}{0.5pt}

\vspace{8pt}

$\textbf{Exercise 3:}$

(i) Let $(X, \mathcal{A}, \mu)$ be a measure space, and $f_{n}$ a converging sequence in $L^{1}(X)$. Show that $f_{n}$ has a sub-sequence which is convergent almost everywhere.
\vspace{8pt}

(ii) Find a sequence $g_{n}$ in $L^{1}([0, 1])$ such that: $g_{n}$ converges in $L^{1}([0, 1])$ and for all $x$ in $[0, 1]$ the sequence $g_{n}(x)$ diverges.
\vspace{8pt}

(iii) In the measure space $(X, \mathcal{A}, \mu)$, let $A_{n}$ be a sequence of element of $\mathcal{A}$ such that $\lim_{n \to \infty} \mu(A_{n}) = 0$ and let $f$ be in $L^{1}(X)$. Show that $\lim_{n \to \infty} \int_{A_{n}} f = 0$. 

\vspace{8pt}

$\textbf{Solution:}$

(i) Firstly we show that when $f_{n}$ converges to $f$ in $L^{1}(X)$, then $f_{n}$ converges to $f$ in measure. For $n \geq 1$ and $\epsilon > 0$, let $A = \{|f_{n} - f| > \epsilon \}$. Note that
\begin{equation*}
    |f_{n} - f| \geq 1_{A}  |f_{n} - f| \geq \epsilon 1_{A},
\end{equation*}
integrating across the inequality yields
\begin{equation*}
    \int_{X} |f_{n} - f| \, d \mu \geq \epsilon \mu(A) .
\end{equation*}
That is
\begin{equation*}
    \mu(|f_{n} - f| \geq \epsilon) \leq \frac{1}{\epsilon} \int_{X} |f_{n} - f| \, d \mu.
\end{equation*}
Since the right hand side converges to $0$ as $n \to \infty$, we have
\begin{equation*}
    \lim_{n \to \infty} \mu(|f_{n} - f| \geq \epsilon) = 0.
\end{equation*}
Therefore we know that $f_{n}$ converges to $f$ in measure. 

Next we show that if $f_{n}$ converges to $f$ in measure, then there exists a sub-sequence $\{f_{n_{k}}\}$ such that $f_{n_{k}} \to f$ pointwise almost everywhere. Since $f_{n}$ converges to $f$ in measure, we can find $n_{1} < n_{2} < \cdots$ such that 
\begin{equation*}
    \mu(|f - f_{n_{k}}| > \frac{1}{k}) \leq \frac{1}{2^{k}}, \quad \forall n \geq n_{k}.
\end{equation*}
Define $E_{k} = \{|f - f_{n_{k}}| > \frac{1}{k}\}$ and $H_{m} = \bigcup_{k = m}^{\infty} E_{k}$, then we have
\begin{equation*}
    \mu(E_{k}) \leq \frac{1}{2^{k}}, \quad \mu(H_{m}) \leq \sum_{k = m}^{\infty} \frac{1}{2^{k}} = \frac{1}{2^{m-1}}.
\end{equation*}
Set $Z = \bigcap_{m = 1}^{\infty} H_{m}$, then
\begin{equation*}
    \mu(Z) \leq \mu(H_{m}) \leq \frac{1}{2^{m-1}}.
\end{equation*}
So we have $\mu(Z) = 0$. If $x \in Z$, then $x \notin H_{m}$ for some $m$, hence $x \notin E_{k}$ for all $k \geq m$, which implies
\begin{equation*}
    |f(x) - f_{n_{k}}| \leq \frac{1}{k}.
\end{equation*}
Thus $f_{n_{k}} \to f(x)$ for all $x \notin Z$. Since $Z$ has zero measure, we therefore have pointwise convergence of $f_{n_{k}}$ to $f$ almost everywhere. 

Thus we know that when $f_{n}$ converges to $f$ in $L^{1}(X)$, then $f_{n}$ converges to $f$ in measure, and then there exists a sub-sequence $\{f_{n_{k}}\}$ such that $f_{n_{k}} \to f$ pointwise almost everywhere.

\vspace{8pt}

(ii) We suppose that
\begin{equation*}
    g_{n} (x) = \mathbb{I}_{[\frac{n - 2^{k}}{2^{k}}, \frac{n - 2^{k} + 1}{2^{k}}]} (x),
\end{equation*}
whenever $k \geq 0, 2^{k} \leq n < 2^{k + 1}$. For any $n \in \mathbb{N}$, we have
\begin{equation*}
    \int_{0}^{1} | g_{n} (x) | \, d x = \int_{0}^{1} \mathbb{I}_{[\frac{n - 2^{k}}{2^{k}}, \frac{n - 2^{k} + 1}{2^{k}}]} (x) \, d x  = \frac{1}{2^{k}} < +\infty,
\end{equation*}
so we know that $g_{n} \in L^{1}((0, 1))$. And similarly we have
\begin{equation*}
    \int_{0}^{1} | g_{n} (x) - 0 | \, d x = \int_{0}^{1} \mathbb{I}_{[\frac{n - 2^{k}}{2^{k}}, \frac{n - 2^{k} + 1}{2^{k}}]} (x) \, d x  = \frac{1}{2^{k}} < \frac{2}{n},
\end{equation*}
then when $n \to + \infty$, $\int_{0}^{1} | g_{n} (x) - 0 | \, d x \to 0$, thus we get $g_{n} \to 0$ in $L^{1}([0, 1])$. But for any $x \in [0, 1]$, and for any $N \in \mathbb{N}$, we can find a $n > N$ with $f_{n} (x) = 1$. Thus $f_{n}$ can not converges to $0$ anywhere for $x \in (0, 1)$. And $g_{n}(x)$ is a sequence of indicator functions of intervals of decreasing length, marching across the unit interval $[0,1]$ over and over again, thus we know that $g_{n}(x)$ diverges.

\vspace{8pt}

(iii) We denote
\begin{equation*}
   f_{n}(x) = f(x) \mathbb{I}_{A_{n}} (x),
\end{equation*}
where $\mathbb{I}_{A_{n}} (\cdot)$ is a indicator function on $A_{n}$. Since $A_{n}$ is a sequence in $\mathcal{A}$ such that $\mu(A_{n}) \to 0$ as $n \to + \infty$, then we know that $f_{n}(x)$ converges to $0$ almost everywhere. As
\begin{equation*}
   |f_{n}(x)| = |f(x) \mathbb{I}_{A_{n}} (x)| \leq |f(x)|
\end{equation*}
and $f \in L^{1}(X)$, we know that $f$ is a dominate function of $f_{n}$. By the dominate convergence theorem, we have
\begin{equation*}
   \lim_{n \to \infty} \int_{X}^{} f_{n}(x) \, d \mu = \int_{X}^{} 0 \, d \mu = 0,
\end{equation*}
thus we have
\begin{equation*}
   \lim_{n \to \infty} \int_{X}^{} f_{n}(x) \, d \mu = \lim_{n \to \infty} \int_{A_{n}}^{} f \, d \mu = 0.
\end{equation*}
So, we know that $\int_{A_{n}}^{} f$ converges to zero.

\noindent\rule[0.25\baselineskip]{\textwidth}{0.5pt}

\vspace{8pt}

$\textbf{Exercise 4:}$

Suppose $f \in L^{1}(\mathbb{R})$ is such that $f > 0$, almost everywhere. Show that $\int f > 0$.

\vspace{8pt}
$\textbf{Solution:}$

Since $f > 0$, we have
\begin{equation*}
    \int f \, d \mu > \int_{\{f \geq \frac{1}{n}\}} f \, d \mu \geq \frac{1}{n} \mu(\{f \geq \frac{1}{n}\}).
\end{equation*}
Let's argue by contraction. Suppose that $\mu(\{f \geq \frac{1}{n}\}) = 0$ for any $n$, since $\{f > 0\} = \bigcup_{n=1}^{\infty} \{f \geq \frac{1}{n}\}$, we have
\begin{equation*}
    \mu(\{f > 0\}) = \mu \Big{(} \bigcup_{n=1}^{\infty} \{f \geq \frac{1}{n}\} \Big{)} \leq \sum_{n = 1}^{\infty} \mu \Big{(} \{f \geq \frac{1}{n}\} \Big{)} = 0,
\end{equation*}
which is contradictory with the condition $f > 0$ almost everywhere. So there exists $n \in \mathbb{N}$ such that $\mu(\{f \geq \frac{1}{n}\}) > 0$. Thus we know that
\begin{equation*}
    \int f \, d \mu \geq  \frac{1}{n} \mu(\{f \geq \frac{1}{n}\}) > 0.
\end{equation*}

\newpage

\section{GCE January, 2015}

$\textbf{Exercise 1:}$

Construct a subset $A \subset \mathbb{R}$ such that $A$ is closed, contains no intervals, is uncountable, and has Lebesgue measure $\frac{1}{2}$ (i.e. $|A| = \frac{1}{2}$). Also explain why your set $A$ has each of the above properties.

\textbf{Hint:} One possible approach here is to adjust the construction of the Cantor set to achieve a Cantor-like set with measure $\frac{1}{2}$, but you don't need to have seen the Cantor set to answer the question.

\vspace{8pt}

$\textbf{Solution:}$

We follow the construction of Cantor set by deleting the open middle forth from a set of line segment. We start by deleting the open middle $(\frac{3}{8}, \frac{5}{8})$ from the interval $[0, 1]$, leaving two line segments $A_{1} = [0, \frac{3}{8}] \cup [\frac{5}{8}, 1]$. Next we do the same thing by deleting $(\frac{5}{32}, \frac{7}{32})$ and $(\frac{25}{32}, \frac{27}{32})$, then we have
\begin{equation*}
    A_{2} = [0, \frac{5}{32}] \cup [\frac{7}{32}, \frac{3}{8}] \cup [\frac{5}{8}, \frac{25}{32}] \cup [\frac{27}{32}, 1].
\end{equation*}
This process is continued as $n \to \infty$, we can get the Cantor-like set $A$.

Since we only delete the open interval from $[0, 1]$ each time, then the union of the intervals we deleted is an open set, thus the Cantor-like set $A$ is closed. We denote $A^{c} = [0, 1] \setminus A$, then we have
\begin{equation*}
    |A^{c}| = \sum_{n = 1}^{\infty} \frac{2^{n-1}}{4^{n}} = \frac{1}{4}  \sum_{n = 1}^{\infty} \Big{(} \frac{1}{2} \Big{)}^{n -1} = \frac{1}{2},
\end{equation*}
thus we know that the measure of Cantor-like set is $\frac{1}{2}$ and it is uncountable. Next we need to show the set $A$ contains no intervals. Suppose the interval $(\alpha, \beta) \in A$. For the $n$-th time we delete the interval whose measure is $\frac{1}{4^{n}}$, so when $n \to \infty$, it is far smaller than $\beta - \alpha$, then we have to separate the interval $(\alpha, \beta)$. Thus similarly with the Cantor set, the Cantor-like set contains no intervals. 


\noindent\rule[0.25\baselineskip]{\textwidth}{0.5pt}

\vspace{8pt}

$\textbf{Exercise 2:}$

(i) Let $(X, \mathcal{A}, \mu)$ be a measure space, and $f_{n}$ a sequence in $L^{1}(X)$. Let $f$ be in $L^{1}(X)$. Assume that $\int f_{n}$ converges to $\int f$, $f_{n}$ converges to $f$ almost everywhere, and for each $n, f_{n} \geq 0$, almost everywhere. Show that $f_{n}$ converges to $f$ in $L^{1}(X)$.

\textbf{Hint:} Set $g_{n} = \min(f_{n}, f)$. Note that $|f_{n} - f| = f + f_{n} - 2 g_{n}$.

(ii) Find a sequence $f_{n}$ in $L^{1}(\mathbb{R})$ and $f$ in $L^{1}(\mathbb{R})$ such that $\int f_{n}$ converges to $\int f$, $f_{n}$ converges to $f$ almost everywhere, but $f_{n}$ does not converge to $f$ in $L^{1}(\mathbb{R})$.
 
\vspace{8pt}
$\textbf{Solution:}$

(i) We set $g_{n} = \min(f_{n}, f)$, then $|f_{n} - f| = f + f_{n} - 2 g_{n}$, thus we can get
\begin{equation*}
    \int_{X}^{} |f_{n} - f| \, d \mu = \int_{X}^{} (f + f_{n} - 2 g_{n}) \, d \mu.
\end{equation*}
Since $f \in L^{1}(X)$ and $f_{n} \in L^{1}(X)$, then we know that $g_{n} \in L^{1}(X)$, so we have
\begin{equation*}
    \int_{X}^{} |f_{n} - f| \, d \mu = \int_{X}^{} f \, d \mu + \int_{X}^{} f_{n} \, d \mu - 2 \int_{X}^{} g_{n} \, d \mu.
\end{equation*}
And by the definition of $g_{n}$, we know that $g_{n}$ converges to $f$ almost everywhere as $f_{n}$ converges to $f$ almost everywhere. As $f_{n} \geq 0$ almost everywhere, then $f \geq 0$ a.e. Since $|g_{n}| \leq |f|$ and $f \in L^{1} (X)$, by the dominate convergence theorem, we know that
\begin{eqnarray*}
    \lim_{n \to \infty} \int_{X}^{} |f_{n} - f| \, d \mu & = & \int_{X}^{} f \, d \mu + \lim_{n \to \infty}  \int_{X}^{} f_{n} \, d \mu - 2 \lim_{n \to \infty} \int_{X}^{} g_{n} \, d \mu \\
    & = & 2 \int_{X}^{} f \, d \mu - 2 \int_{X}^{} \lim_{n \to \infty} g_{n} \, d \mu \\
    & = & 2 \int_{X}^{} f \, d \mu - 2 \int_{X}^{} f \, d \mu = 0,
\end{eqnarray*}
hence we get $f_{n}$ converges to $f$ in $L^{1}(X)$.

(ii) We denote
\begin{equation*}
f_{n} (x) =
\left\{
             \begin{array}{cl}
             \frac{1}{n}, & x \in [-n, 0] \\
             - \frac{1}{n}, & x \in (0, n]
             \end{array}
\right.
\end{equation*}
and $f(x) = 0$, since $|f_{n} \leq \frac{1}{n}|$, we have $f_{n}$ converges to $f$ almost everywhere. As
\begin{equation*}
    \int_{\mathbb{R}}^{} f_{n} \, d \mu = \int_{-n}^{0} \frac{1}{n} \, d \mu + \int_{0}^{n} \Big{(} - \frac{1}{n} \Big{)} \, d \mu = 1 - 1 = 0,
\end{equation*}
we know that $f_{n}$ in $L^{1}(\mathbb{R})$ and $\int f_{n}$ converges to $\int f$. But since
\begin{equation*}
    \int_{\mathbb{R}}^{} |f_{n} - f| \, d \mu = \int_{-n}^{n} \frac{1}{n} \, d \mu = 2,
\end{equation*}
we can get that $f_{n}$ does not converge to $f$ in $L^{1}(\mathbb{R})$.


\noindent\rule[0.25\baselineskip]{\textwidth}{0.5pt}

\vspace{8pt}

$\textbf{Exercise 3:}$

Let $(X, \mathcal{A}, \mu)$ be a measure space.

(i) Let $f$ be in $L^{1}([0, \infty))$. Show that
\begin{equation*}
    \lim_{x \to 0^{+}} \int_{0}^{\infty} f(t) e^{- x t} \, d t = \int_{0}^{\infty} f(t) \, d t
\end{equation*}

(ii) Let $[a, b]$ be an interval in $\mathbb{R}$. If $\Tilde{f}$ is continuous on $[a, b]$ and monotonic, and $g^{'}$ is continuous on $[a, b]$, we can prove that there is a $c$ in $[a, b]$ such that
\begin{equation*}
    \int_{a}^{b} \Tilde{f} g = g(a) \int_{a}^{c} \Tilde{f} + g(b) \int_{c}^{b} \Tilde{f}.
\end{equation*}
Using this result, show that if $g$ is as specified above and $f$ is in $L^{1}([a, b])$, there is a $c$ in $[a, b]$ such that 
\begin{equation*}
    \int_{a}^{b} f g = g(a) \int_{a}^{c} f + g(b) \int_{c}^{b} f.
\end{equation*}

(iii) Let $f$ be in $L^{\infty}([0, \infty))$. Assume that there is a constant $L$ in $\mathbb{R}$ such that $\lim_{x \to \infty} \int_{0}^{x} f = L$. Show that 
\begin{equation*}
    \lim_{x \to 0^{+}} \int_{0}^{ \infty} f(t) e^{- x t} \, d t = L.
\end{equation*}


\vspace{8pt}
$\textbf{Solution:}$

(i) When $x \geq 0$ and $t \geq 0$, we know that $|f(t) e^{- x t}| \leq |f(t)|$. As $f \in L^{1}([0, \infty))$ and for any fixed $t$, $\lim_{x \to 0^{+}} f(t) e^{- x t} = f(t)$, by the dominate convergence theorem, we have
\begin{equation*}
    \lim_{x \to 0^{+}} \int_{0}^{\infty} f(t) e^{- x t} \, d t = \int_{0}^{\infty} \lim_{x \to 0^{+}} f(t) e^{- x t} \, d t = \int_{0}^{\infty} f(t) \, d t. 
\end{equation*}

(ii) Since $\tilde{f}$ is continuous on $[a, b]$, we can introduce $F(x) = \int_{a}^{x} \tilde{f}$, and we know that $F'(x) = \tilde{f}(x)$. Then through integral by parts, we have 
\begin{eqnarray*}
\int_{a}^{b} \tilde{f(x)} g(x) \, d x & = & \int_{a}^{b} g(x) \, d F(x) \\
& = & g(b) F(b) - g(a) F(a) - \int_{a}^{b} g'(x) F(x) \, d x  \\
& = & g(b) \int_{a}^{b} \tilde{f} (x) \, d x - g(a) \int_{a}^{a} \tilde{f} (x) \, d x - \int_{a}^{b} g'(x) F(x) \, d x  \\
& = &  g(b) \int_{a}^{b} \tilde{f} (x) \, d x - \int_{a}^{b} g'(x) F(x) \, d x.
\end{eqnarray*}

Since $g$ is differentiable on $[a, b]$ and monotonic, and $g'$ is continuous on $[a, b]$, we know that $g'$ is integrable in $[a, b]$ and $g'(x) \geq 0$ for all $x \in [a, b]$. By the mean value theorem for integral, there exists $c \in [a, b]$, and
\begin{equation*}
   \int_{a}^{b} g'(x) F(x) \, d x = F(c) \int_{a}^{b} g'(x) \, d x = F(c) (g(b) - g(a)).
\end{equation*}
Thus for this $c \in [a, b]$, we have
\begin{eqnarray*}
\int_{a}^{b} \tilde{f(x)} g(x) \, d x & = & g(b) \int_{a}^{b} \tilde{f} (x) \, d x - F(c) (g(b) - g(a)) \\
& = & g(b) \int_{a}^{b} \tilde{f} (x) \, d x - (g(b) - g(a)) \int_{a}^{c} \tilde{f} (x) \, d x \\
& = & g(b) \int_{a}^{b} \tilde{f} (x) \, d x - g(b) \int_{a}^{c} \tilde{f} (x) \, d x + g(a) \int_{a}^{c} \tilde{f} (x) \, d x  \\
& = &  g(b) \int_{c}^{b} \tilde{f} (x) \, d x + g(a) \int_{a}^{c} \tilde{f} (x) \, d x.
\end{eqnarray*}

Since $C_{c}([a, b])$ is dense in $L^{1}([a, b])$, then we know that for any $f \in L^{1}([0, 1])$, there exists a function sequence $\{f_{n}\} \subset C_{c}([a, b])$ and $\int_{a}^{b} |f_{n} - f| \to 0$ as $n \to + \infty$.
Since $g$ is differentiable on $[a,b]$ and monotonic, we know there exists $K > 0$, and $\forall x \in [a, b]$, we have $|g(x)| \leq K$. So, we have
\begin{equation*}
   \lim_{n \to + \infty} \int_{a}^{b} |g f - g f_{n}| \leq K \lim_{n \to + \infty} \int_{a}^{b}|f - f_{n}| = 0,
\end{equation*}
then by the conclusion we get from (i) we have
\begin{equation*}
   \int_{a}^{b} f g = \lim_{n \to + \infty} \int_{a}^{b} f_{n} g = \lim_{n \to + \infty} \Big{(} g(a) \int_{a}^{c_{n}} f_{n} + g(b) \int_{c_{n}}^{b} f_{n} \Big{)},
\end{equation*}
where $c_{n}$ is depends on $f_{n}$ for each n.

Since $\{c_{n}\} \subset [a, b]$ and $[a, b]$ is compact, there exists a subsequence of $\{c_{n}\}$, which is denoted as $\{c_{n_{k}}\}$, converges to $c$ and $c \in [a, b]$. Thus we have
\begin{eqnarray*}
\int_{a}^{b} f g & = & \lim_{k \to + \infty} \Big{(} g(a) \int_{a}^{c_{n_{k}}} f_{n_{k}} + g(b) \int_{c_{n_{k}}}^{b} f_{n_{k}} \Big{)} \\
& = & \lim_{k \to + \infty} \Big{(} g(a) \int_{a}^{c} f_{n_{k}} + g(a) \int_{c}^{c_{n_{k}}} f_{n_{k}} + g(b) \int_{c_{n_{k}}}^{c} f_{n_{k}} + g(b) \int_{c}^{b} f_{n_{k}} \Big{)} \\
& = &  g(a) \int_{a}^{c} f + g(b) \int_{c}^{b} f + \lim_{k \to + \infty} \Big{(} g(a) \int_{c}^{c_{n_{k}}} f_{n_{k}} +  g(b) \int_{c_{n_{k}}}^{c} f_{n_{k}} \Big{)} \\
& = & g(a) \int_{a}^{c} f + g(b) \int_{c}^{b} f.
\end{eqnarray*}

(iii) For any $K > 0$, we have
\begin{equation*}
    \lim_{x \to 0^{+}} \int_{0}^{ \infty} f(t) e^{- x t} \, d t =  \lim_{x \to 0^{+}} \Big{(} \int_{0}^{K} f(t) e^{- x t} \, d t + \int_{K}^{\infty} f(t) e^{- x t} \, d t \Big{)}
\end{equation*}
let $K \to \infty$, we can get 
\begin{equation*}
    \lim_{x \to 0^{+}} \int_{0}^{ \infty} f(t) e^{- x t} \, d t  =  \lim_{x \to 0^{+}} \lim_{K \to \infty} \int_{0}^{K} f(t) e^{- x t} \, d t, 
\end{equation*}
then we know that 
\begin{eqnarray*}
\lim_{x \to 0^{+}} \int_{0}^{ \infty} f(t) e^{- x t} \, d t 
& = & \lim_{x \to 0^{+}} \lim_{K \to \infty} \Big{(} \int_{0}^{K} f(t) \, d t + \int_{0}^{K} f(t) ( e^{- x t} - 1) \, d t \Big{)} \\
& = & L + \lim_{x \to 0^{+}} \lim_{K \to \infty}  \int_{0}^{K} f(t) ( e^{- x t} - 1) \, d t \\
& = & L + \lim_{K \to \infty} \lim_{x \to 0^{+}} \int_{0}^{K} f(t) ( e^{- x t} - 1) \, d t
\end{eqnarray*}
as $\int_{0}^{K} f(t) e^{- x t} \, d t$ is continuous with $x$ and $K$. As $f(t) \in L^{\infty}([0, \infty))$, we have
\begin{equation*}
    \int_{0}^{K} |f(t)| \, d t \leq K \|f\|_{\infty} < \infty,
\end{equation*}
then we know that $f(t) \in L^{1}([0, K])$. And since $|f(t) ( e^{- x t} - 1)| \leq |f(t)|$ when $x \geq 0, t \geq 0$, by the dominate convergence theorem, we have
\begin{equation*}
    \lim_{x \to 0^{+}} \int_{0}^{K} f(t) (e^{-x t} - 1) \, d t = \int_{0}^{K} f(t) \lim_{x \to 0^{+}} (e^{-x t} - 1) \, d t = 0,
\end{equation*}
hence we can get
\begin{equation*}
    \lim_{x \to 0^{+}} \int_{0}^{ \infty} f(t) e^{- x t} \, d t = L.
\end{equation*}

\newpage

\section{GCE May, 2015}

$\textbf{Exercise 1:}$

Give an example of $f_{n}, f \in L^{1}(\mathbb{R})$ such that $f_{n} \to f$ uniformly, but $\|f_{n}\|_{1}$ does not converge to $\|f\|_{1}$.


\vspace{8pt}

$\textbf{Solution:}$

Example 1: We suppose that $f_{n}(x) = \frac{1}{n} \mathbb{I}_{[1, n]} (x)$ and $f(x) = 0$. Since
\begin{equation*}
    |f_{n}(x) - 0| = |\frac{1}{n} \mathbb{I}_{[1, n]} (x) - 0| \leq \frac{1}{n},
\end{equation*}
we know that $f_{n} \to f$ uniformly. As $\|f(x)\|_{1} = 0$ and 
\begin{equation*}
    \|f_{n}\|_{1} = \int_{\mathbb{R}}^{} |f_{n}(x)| \, d x = \int_{1}^{n} \frac{1}{n} \, d x = \frac{n-1}{n} \to 1
\end{equation*}
as $n \to \infty$. So we have $\|f_{n}\|_{1}$ does not converge to $\|f\|_{1}$.

Example 2: We set $f(x) = 0$ and 
\begin{equation*}
    f_{n}(x) = \Big{(} - \frac{1}{2^{2n}} + \frac{1}{2^{n}} \Big{)} \cdot \mathbb{I}_{[0, 2^{2n}]} (x).
\end{equation*}
Since $|f_{n}(x) - f(x)| < \frac{1}{2^{n}}$, we know that $f_{n} \to f$ uniformly. And as
\begin{equation*}
    \|f_{n}(x)\|_{1} = \int_{0}^{2^{2n}}  - \frac{1}{2^{2n}} + \frac{1}{2^{n}} \, d x = 2^{n} - 1 \to + \infty,
\end{equation*}
we have $\|f_{n}\|_{1}$ does not converge to $\|f\|_{1}$.


\noindent\rule[0.25\baselineskip]{\textwidth}{0.5pt}

\vspace{8pt}

$\textbf{Exercise 2:}$

Show that for all $\epsilon > 0$ and all $f \in L^{1}(\mathbb{R}), \exists n \in \mathbb{N}$ such that $\|f - f_{n}\|_{1} < \epsilon$ for some $f_{n}$ with $|f_{n}| \leq n$ and $f_{n} = 0$ on $\mathbb{R} \setminus [-n, n]$.
 
\vspace{8pt}
$\textbf{Solution:}$

We suppose that
\begin{equation*}
    f_{n} (x) = f \cdot \mathbb{I}_{\{x: |f(x)| \leq n\} \cap \{x \in [-n, n]\}} (x),
\end{equation*}
so we know that $f_{n} = 0$ on $\mathbb{R} \setminus [-n, n]$ and $|f_{n}| \leq n$. Next we need to show that $\exists n \in \mathbb{N}$ such that $\|f - f_{n}\|_{1} < \epsilon$. We know that
\begin{eqnarray*}
    \|f_{n} - f\|_{1} &=& \int_{\mathbb{R}}^{} |f_{n} - f| \, d x \\
    &=& \int_{\{|f| \geq n\} \cup \{x \in \mathbb{R} \setminus [-n, n]\}}^{} |f| \, d x  \\
    & \leq &  \int_{\{|f| \geq n\}}^{} |f| \, d x + \int_{-\infty}^{-n} |f| \, d x + \int_{n}^{+\infty} |f| \, d x .
\end{eqnarray*}
Since 
\begin{equation*}
    \int_{\{|f| \geq n\}}^{} |f| \, d x = \int_{\mathbb{R}}^{} |f| \mathbb{I}_{\{|f|>n\}} (x) \, d x
\end{equation*}
and $|f| \mathbb{I}_{\{|f|>n\}} (x)$ goes to $0$ pointwise and $|f| \mathbb{I}_{|f|>n} (x) < |f| \in L^{1}(\mathbb{R})$, by the dominate convergence theorem, we have
\begin{equation*}
    \lim_{n \to \infty} \int_{\mathbb{R}}^{} |f| \mathbb{I}_{|f|>n} \, d x = \int_{\mathbb{R}}^{}  \lim_{n \to \infty} |f| \mathbb{I}_{|f|>n} (x) \, d x = 0.
\end{equation*}
Similarly, since 
\begin{equation*}
    \int_{n}^{+ \infty} |f| \, d x = \int_{\mathbb{R}}^{} |f| \mathbb{I}_{[n, +\infty)} (x) \, d x,
\end{equation*}
and $|f| \mathbb{I}_{[n, +\infty)} (x) \to 0$ as $n \to \infty$ pointwise and $|f| \mathbb{I}_{[n, +\infty)} (x) \leq |f| \in L^{1}(\mathbb{R})$, by the dominate convergence theorem, we can get
\begin{equation*}
  \lim_{n \to \infty} \int_{\mathbb{R}}^{} |f| \mathbb{I}_{[n, +\infty)} (x) \, d x = \int_{\mathbb{R}}^{}  \lim_{n \to \infty} |f| \mathbb{I}_{[n, +\infty)} (x) \, d x = 0.
\end{equation*}
Then we also can get
\begin{equation*}
    \lim_{n \to \infty} \int_{-\infty}^{-n} |f| \, d x = 0.
\end{equation*}
Thus we have
\begin{equation*}
    \lim_{n \to \infty} \|f_{n} - f\|_{1} \leq \lim_{n \to \infty} \Big{(} \int_{\{|f| \geq n\}}^{} |f| \, d x + \int_{-\infty}^{-n} |f| \, d x + \int_{n}^{+\infty} |f| \, d x \Big{)} = 0,
\end{equation*}
hence we know that $\exists n \in \mathbb{N}$ such that $\|f - f_{n}\|_{1} < \epsilon$.

\noindent\rule[0.25\baselineskip]{\textwidth}{0.5pt}

\vspace{8pt}

$\textbf{Exercise 3:}$

Let $(X, \mathcal{A}, \mu)$ be a measure space.

(i) If $f$ is in $L^{1}(X) \cap L^{\infty}(X)$, show that $|f|^{p} \in L^{1}(X)$ for all $p$ in $(1, \infty)$.

(ii) If $f$ is in $L^{1}(X) \cap L^{\infty}(X)$, show that
\begin{equation*}
    \lim_{p \to \infty} \Big{(} \int_{}^{} |f|^{p} \Big{)}^{\frac{1}{p}} = \|f\|_{\infty}.
\end{equation*}

(iii) Set $A = \{x \in X: |f(x)| > 0\}$. If $f$ is in $L^{\infty} (X), \mu(A) < \infty$, and $\mu(A) \neq 1$, find
\begin{equation*}
    \lim_{p \to 0^{+}} \Big{(} \int_{}^{} |f|^{p} \Big{)}^{\frac{1}{p}}.
\end{equation*}

(iv) We now assume that the set $A$ defined in (iii) satisfies $\mu(A) = 1$, that $f$ is in $L^{\infty}(X)$, and $\ln |f|$ is in $L^{1}(X)$, find
\begin{equation*}
    \lim_{p \to 0^{+}} \Big{(} \int_{}^{} |f|^{p} \Big{)}^{\frac{1}{p}}.
\end{equation*}

\vspace{8pt}
$\textbf{Solution:}$

(i) We need to show $|f|^{p} \in L^{1}(X)$, so we just need to show that for any $p \in (1, \infty)$, $f \in L^{p}(X)$. For any $p \in (1, \infty)$, since $f \in L^{1}(X) \cap L^{\infty}(X)$, we have
\begin{eqnarray*}
    \|f\|_{p} &=& \Big{(} \int_{X}^{} |f|^{p} \, d \mu \Big{)}^{\frac{1}{p}} \\
    &=& \Big{(} \int_{X}^{} |f| |f|^{p-1} \, d \mu \Big{)}^{\frac{1}{p}} \\
    & \leq &  (\|f\|_{\infty})^{\frac{p-1}{p}} (\|f\|_{1})^{\frac{1}{p}} < \infty,
\end{eqnarray*}
thus we know that $f \in L^{p}(X)$. So, we know that $|f|^{p} \in L^{1}(X)$ for all $p$ in $(1, \infty)$.

\vspace{8pt}

(ii) We denote $t \in [0, \|f \|_{\infty})$, then the set 
\begin{equation*}
   A = \{x \in X: |f(x)| \geq t \}
\end{equation*}
has positive and bounded measure. Since
\begin{eqnarray*}
\|f\|_{p} & = & \Big{(} \int_{(0, 1)}^{} |f|^{p} \, d \mu \Big{)}^{\frac{1}{p}} \geq \Big{(} \int_{A}^{} |f|^{p} \, d \mu \Big{)}^{\frac{1}{p}} \\
& \geq & \Big{(} t^{p} \mu(A)\Big{)}^{\frac{1}{p}} = t (\mu(A))^{\frac{1}{p}},
\end{eqnarray*}
if $\mu(A)$ is finite, then when $p \to + \infty$, we have $(\mu(A))^{\frac{1}{p}} \to 1$ and if $\mu(A) = \infty$, then $(\mu(A)^{\frac{1}{p}}) = \infty$, in both cases we have
\begin{equation*}
   \liminf_{p \to + \infty} \|f\|_{p} \geq t.
\end{equation*}
Since $t$ is arbitrary and $t \in [0, \|f \|_{\infty})$, we have
\begin{equation*}
   \liminf_{p \to + \infty} \|f\|_{p} \geq \|f \|_{\infty} .
\end{equation*}
On the other hand, as $f(x)$ is in $L^{1}(X)$, we have
\begin{eqnarray*}
    \|f\|_{p} &=& \Big{(} \int_{X}^{} |f|^{p} \, d \mu \Big{)}^{\frac{1}{p}} \\
    &=& \Big{(} \int_{X}^{} |f| |f|^{p-1} \, d \mu \Big{)}^{\frac{1}{p}} \\
    & \leq &  (\|f\|_{\infty})^{\frac{p-1}{p}} (\|f\|_{1})^{\frac{1}{p}} .
\end{eqnarray*}
Since $\|f\|_{1} < + \infty$, then when $p \to + \infty$, we know that
\begin{equation*}
   \limsup_{p \to + \infty} \|f\|_{p} \leq \|f \|_{\infty} .
\end{equation*}
Thus we have
\begin{equation*}
   \limsup_{p \to + \infty} \|f\|_{p} \leq \|f \|_{\infty} \leq \liminf_{p \to + \infty} \|f\|_{p},
\end{equation*}
then we know that $\|f \|_{p} \rightarrow \|f \|_{\infty}$ as $p \rightarrow \infty$.


(iii) When $\mu(A)<1$, we have
\begin{eqnarray*}
    \int_{X}^{} |f|^{p} \, d \mu &=& \int_{A}^{} |f|^{p} \, d \mu \\
    & \leq & \|f\|_{\infty}^{p} \mu(A).
\end{eqnarray*}
Since $f \in L^{\infty}(X)$ and $\mu(A) < 1$, we know that
\begin{equation*}
    \lim_{p \to 0^{+}} \Big{(} \int_{}^{} |f|^{p} \Big{)}^{\frac{1}{p}} \leq \lim_{p \to 0^{+}} \|f\|_{\infty} (\mu(A))^{\frac{1}{p}} = 0
\end{equation*}
But if we set $f = 1$ and $\mu(X) < \infty$, we know that $f \in L^{\infty}(X)$, if $\mu(A) > 1$, we have
\begin{equation*}
    \lim_{p \to 0^{+}} \Big{(} \int_{}^{} |f|^{p} \Big{)}^{\frac{1}{p}} = \lim_{p \to 0^{+}} (\mu(A))^{\frac{1}{p}} = \infty.
\end{equation*}
Thus the limit is not exist.

(iv) Since we have $A = \{x \in X: |f| > 0\}$, then
\begin{eqnarray*}
    \int_{X}^{} |f|^{p} \, d \mu & = & \int_{\{x \in X: |f| > 0\}}^{} |f|^{p} \, d \mu + \int_{\{x \in X: |f| = 0\}}^{} |f|^{p} \, d \mu \\
    & = & \int_{A}^{} |f|^{p} \, d \mu.
\end{eqnarray*}
And we denote that $F(p) = \log (\int_{A}^{} |f|^{p} \, d \mu)$, then we know that
\begin{equation*}
    \lim_{p \to 0^{+}} \Big{(} \int_{}^{} |f|^{p} \Big{)}^{\frac{1}{p}} = \lim_{p \to 0^{+}} e^{\frac{F(p)}{p}}.
\end{equation*}
As $F(0) = \log(\mu(A)) = 0$ and $e^{x}$ is continuous, then we have
\begin{eqnarray*}
    \lim_{p \to 0^{+}} \Big{(} \int_{}^{} |f|^{p} \Big{)}^{\frac{1}{p}} & = & \lim_{p \to 0^{+}} \exp \Big{\{} \frac{F(p) - F(0)}{p - 0} \Big{\}} \\
    & = & \exp \Big{\{} \lim_{p \to 0^{+}} \frac{F(p) - F(0)}{p - 0} \Big{\}} \\
    & = & e^{F^{'}(0)}.
\end{eqnarray*}
As $F(p) = \log (\int_{A}^{} |f|^{p} \, d \mu)$ and $\ln |f|$ is in $L^{1}(X)$, we have 
\begin{equation*}
    F^{'}(p) = \frac{\int_{A}^{} |f|^{p} \cdot \log |f| \, d \mu}{\int_{A}^{} |f|^{p} \, d \mu},
\end{equation*}
thus we have $F^{'}(0) = \frac{\int_{A}^{} \log |f| \, d \mu}{\mu(A)} = \int_{A}^{} \log |f| \, d \mu$ . Then we know that
\begin{eqnarray*}
    \lim_{p \to 0^{+}} \Big{(} \int_{}^{} f^{p} \Big{)}^{\frac{1}{p}} & = & e^{F^{'}(0)} \\
    & = & \exp (\int_{A}^{} \log |f| \, d \mu).
\end{eqnarray*}

\newpage

\section{GCE August, 2015}

$\textbf{Exercise 1:}$

Use the Fubini theorem to prove that
\begin{equation*}
    \int_{\mathbb{R}^{n}}^{} e^{- |\textbf{x}|^{2}} \, d \textbf{x} = \pi^{\frac{n}{2}}
\end{equation*}
Here $\textbf{x} = (x_{1}, x_{2}, \dots, x_{n})$. Hint: For $n = 2$, use polar coordinates.


\vspace{8pt}

$\textbf{Solution:}$

Firstly, we define 
\begin{equation*}
    I(a) = \int_{-a}^{a} e^{- x^{2}} \, d x,
\end{equation*}
then we have
\begin{equation*}
    I^{2}(a) = \int_{-a}^{a} e^{- x^{2}} \, d x \int_{-a}^{a} e^{- y^{2}} \, d y.
\end{equation*}
As $(-a, a)$ is an interval with finite measure and $|e^{-x^{2}}| \leq 1$, by the Fubini theorem, we have
\begin{equation*}
    I^{2}(a) = \int_{-a}^{a} \int_{-a}^{a} e^{- (x^{2}+y^{2})} \, d x \, d y.
\end{equation*}
Take the transformation as follows,
\begin{equation*}
\left\{
             \begin{array}{cl}
             x = r \cos \theta \\
             y = r \sin \theta 
             \end{array}
\right.
\end{equation*}
then we know that
\begin{equation*}
    \int_{0}^{2 \pi} \int_{0}^{a} r e^{-r^{2}} \, d r\, d \theta < I^{2}(a) < \int_{0}^{2 \pi} \int_{0}^{\sqrt{2} a} r e^{-r^{2}}  \, d r \, d \theta,
\end{equation*}
thus we can get
\begin{equation*}
    (1 - e^{- a^{2}}) \pi < I^{2}(a) < (1 - e^{- 2a^{2}}) \pi .
\end{equation*}
Let $a \to \infty$, we have \begin{equation*}
    \lim_{a \to \infty} I^{2}(a) = \int_{\mathbb{R}^{2}}^{} e^{- |\textbf{x}|^{2}} \, d \textbf{x} = \pi,
\end{equation*}
then we know that $\lim_{a \to \infty} I(a) = \int_{\mathbb{R}}^{} e^{- x^{2}} \, d x = \sqrt{\pi}$. For the $n$ dimensional domain, we have
\begin{eqnarray*}
    \int_{\mathbb{R}^{n}}^{} e^{- |\textbf{x}|^{2}} \, d \textbf{x} & = & \int_{\mathbb{R}}^{} \int_{\mathbb{R}}^{} \cdots \int_{\mathbb{R}}^{} e^{- (x_{1}^{2} + x_{2}^{2} + \cdots x_{n}^{2})} \, d x_{1} d x_{2} \cdots d x_{n} \\
    & = & \Big{(} \int_{\mathbb{R}}^{} e^{- x_{1}^{2} } \, d x_{1} \Big{)}^{n} = \pi^{\frac{n}{2}}.
\end{eqnarray*}

\newpage

$\textbf{Exercise 2:}$

Let $(X, \mathcal{A}, \mu)$ be a measure space, and $f$ be in $L^{1}(X)$. Let for all positive integers $n$ set $B_{n} = \{x \in X: n-1 \leq |f(x)| < n \}$.

(i) Show that $\mu(B_{n}) < \infty$ for all $n \geq 2$.

(ii) Show that $\sum_{n=2}^{\infty} n \mu(B_{n}) < \infty$.

(iii) Define $C_{n} = \{x \in X : n-1 \leq |f(x)| \leq n \}$. Is the sum $\sum_{n=2}^{\infty} n \mu(C_{n})$ finite?

(iv) Show that $$\sum_{n=2}^{\infty} \sum_{m=2}^{n} \frac{m^{2}}{n^{2}} \mu(B_{m}) < \infty.$$

(v) Show that for $n \geq 2$
\begin{equation*}
    \int_{}^{} |f|^{2} 1_{\{|f| < n\}} = \int_{}^{} |f|^{2} 1_{\{|f| < 1\}} + \sum_{m=2}^{n} \int_{}^{} |f|^{2} 1_{B_{m}}
\end{equation*}
and infer that
\begin{equation*}
    \sum_{n=1}^{\infty} \frac{1}{n^{2}} \int_{}^{} |f|^{2} 1_{\{|f| < n\}} < \infty 
\end{equation*}
 
 
 
\vspace{8pt}
$\textbf{Solution:}$

(i) Since $f \in L^{1}(X)$, we have $\int_{X}^{} |f| \, d \mu < \infty$. For $n \geq 2$, we know that
\begin{equation*}
    \int_{X}^{} |f| \, d \mu \geq \int_{B_{n}}^{} |f| \, d \mu \geq (n-1) \int_{B_{n}}^{} 1 \, d \mu = (n-1) \mu(B_{n}).
\end{equation*}
When $n \geq 2$, we have $n-1 \geq 1$, then we know that $\mu(B_{n}) < \infty$.  So, for any $n \geq 2$, we have $\mu(B_{n}) < \infty$.

(ii) Since $B_{n} = \{ x \in X: n-1 \leq |f(x)| < n \} = \{x \in X: n \leq |f(x)|+1 < n+1 \}$, we have
\begin{eqnarray*}
    \sum_{n=2}^{\infty} n \mu(B_{n}) & = & \sum_{n=2}^{\infty} \int_{B_{n}}^{} n \, d \mu \\
& \leq & \sum_{n=2}^{\infty}  \int_{B_{n}}^{} |f(x)| + 1 \, d \mu \\
& = & \sum_{n=2}^{\infty}  \int_{B_{n}}^{} |f(x)| \, d \mu + \sum_{n=2}^{\infty}  \int_{B_{n}}^{} 1 \, d \mu \\ 
& \leq & 2 \int_{\bigcup_{n=2}^{\infty} B_{n}}^{} |f(x)| \, d \mu \\
& \leq & 2 \int_{X}^{} |f(x)| \, d \mu < \infty.
\end{eqnarray*}

(iii) We claim that the sum $\sum_{n=2}^{\infty} n \mu(C_{n})$ is finite. As $C_{n} = \{x \in X : n-1 \leq |f(x)| \leq n \} \subset B_{n} \cup B_{n+1}$, then we have
\begin{equation*}
    \mu(C_{n}) \leq \mu(B_{n} \cup B_{n+1}) \leq \mu(B_{n}) + \mu(B_{n+1}),
\end{equation*}
therefore, we know that
\begin{equation*}
    \sum_{n=2}^{\infty} n \mu(C_{n}) \leq \sum_{n=2}^{\infty} n \mu(B_{n}) +  \sum_{n=2}^{\infty} n \mu (B_{n+1}).
\end{equation*}
Since $\int_{B_{n+1}}^{} |f| \, d \mu  \geq n \int_{B_{n+1}}^{} 1 \, d \mu = n \mu(B_{n+1})$, then we have
\begin{eqnarray*}
    \sum_{n=2}^{\infty} n \mu(B_{n+1}) & \leq & \sum_{n=2}^{\infty} \int_{B_{n+1}}^{} |f| \, d \mu \\
    & = & \int_{\bigcup_{n=2}^{\infty} B_{n+1}}^{} |f| \, d \mu \\
    & < & \int_{X}^{} |f| \, d \mu < \infty.
\end{eqnarray*}
As we showed $\sum_{n=2}^{\infty} n \mu(B_{n}) < \infty$ in (ii), hence we have
\begin{equation*}
    \sum_{n=2}^{\infty} n \mu(C_{n}) \leq \sum_{n=2}^{\infty} n \mu(B_{n}) +  \sum_{n=2}^{\infty} n \mu (B_{n+1}) < \infty.
\end{equation*}

(iv) We can rewrite the $\sum_{n=2}^{\infty} \sum_{m=2}^{n} \frac{m^{2}}{n^{2}} \mu(B_{m}) $ and then we have
\begin{eqnarray*}
    \sum_{n=2}^{\infty} \sum_{m=2}^{n} \frac{m^{2}}{n^{2}} \mu(B_{m}) & = & \sum_{m=2}^{\infty} \mu(B_{m}) m^{2} \sum_{n=m}^{\infty} \frac{1}{n^{2}}  \\
    & = & \sum_{m=2}^{\infty} m \mu(B_{m}) \sum_{n=m}^{\infty} \frac{m}{n^{2}}.
\end{eqnarray*}
Next we need to show that $\sum_{n=m}^{\infty} \frac{m}{n^{2}}$ is bounded. When $m \geq 2$, we have
\begin{equation*}
    \sum_{n=m}^{\infty} \frac{m}{n^{2}} < m \int_{m-1}^{\infty} \frac{1}{x^{2}} \, d x = \frac{m}{m-1} \leq 2,
\end{equation*}
then we know that
\begin{equation*}
    \sum_{n=2}^{\infty} \sum_{m=2}^{n} \frac{m^{2}}{n^{2}} \mu(B_{m}) < 2 \sum_{m=2}^{\infty} m \mu(B_{m}) < \infty.
\end{equation*}

(v) Firstly, we show that 
\begin{equation*}
    \int_{}^{} |f|^{2} 1_{\{|f| < n\}} \, d \mu = \int_{}^{} |f|^{2} 1_{\{|f| < 1\}} \, d \mu + \sum_{m=2}^{n} \int_{}^{} |f|^{2} 1_{B_{m}} \, d \mu.
\end{equation*}
By calculation, we have
\begin{eqnarray*}
    \int_{}^{} |f|^{2} 1_{\{|f| < n\}} \, d \mu & = & \int_{}^{} |f|^{2} 1_{\{|f| < 1\}} \, d \mu + \int_{}^{} |f|^{2} 1_{\{1 \leq |f| < n\}} \, d \mu \\
    & = & \int_{}^{} |f|^{2} 1_{\{|f| < 1\}} \, d \mu + \int_{}^{} |f|^{2} \sum_{m=2}^{n} 1_{\{m-1 \leq |f| < m\}} \, d \mu \\
    & = & \int_{}^{} |f|^{2} 1_{\{|f| < 1\}} \, d \mu + \sum_{m=2}^{n} \int_{}^{} |f|^{2}  1_{\{m-1 \leq |f| < m\}} \, d \mu \\
    & = & \int_{}^{} |f|^{2} 1_{\{|f| < 1\}} \, d \mu + \sum_{m=2}^{n} \int_{}^{} |f|^{2} 1_{B_{m}} \, d \mu,
\end{eqnarray*}
then we get the equation we wanted. Next we show that $\sum_{n=1}^{\infty} \frac{1}{n^{2}} \int_{}^{} |f|^{2} 1_{\{|f| < n\}} < \infty $. As
\begin{eqnarray*}
   &  & \sum_{n=1}^{\infty} \frac{1}{n^{2}} \int_{}^{} |f|^{2} 1_{\{|f| < n\}} \, d \mu  \\
    & = &  \sum_{n=1}^{\infty} \frac{1}{n^{2}} \Big{(} \int_{}^{} |f|^{2} 1_{\{|f| < 1\}} \, d \mu + \sum_{m=2}^{n} \int_{}^{} |f|^{2} 1_{B_{m}} \, d \mu \Big{)} \\
    & = & \sum_{n=1}^{\infty} \frac{1}{n^{2}} \int_{}^{} |f|^{2} 1_{\{|f| < 1\}} \, d \mu + \sum_{n=1}^{\infty} \frac{1}{n^{2}} \sum_{m=2}^{n} \int_{}^{} |f|^{2} 1_{B_{m}} \, d \mu.
\end{eqnarray*}
For the first term in the right hand side of the above equation, we have
\begin{equation*}
    \sum_{n=1}^{\infty} \frac{1}{n^{2}} \int_{}^{} |f|^{2} 1_{\{|f| < 1\}} \, d \mu < \sum_{n=1}^{\infty} \frac{1}{n^{2}} \int_{X}^{} |f| \, d \mu  < \infty.
\end{equation*}
And for the second term in the right hand side of the above equation, we have
\begin{eqnarray*}
    \sum_{n=1}^{\infty} \frac{1}{n^{2}} \sum_{m=2}^{n} \int_{}^{} |f|^{2} 1_{B_{m}} \, d \mu & = & \sum_{n=2}^{\infty} \frac{1}{n^{2}} \sum_{m=2}^{n} \int_{}^{} |f|^{2} 1_{B_{m}} \, d \mu \\
   & \leq & \sum_{n=2}^{\infty} \frac{1}{n^{2}} \sum_{m=2}^{n} \int_{}^{} m^{2} 1_{B_{m}} \, d \mu \\
   & = & \sum_{n=2}^{\infty} \sum_{m=2}^{n} \frac{m^{2}}{n^{2}} \mu(B_{m}) < \infty.
\end{eqnarray*}
Thus we can get
\begin{equation*}
    \sum_{n=1}^{\infty} \frac{1}{n^{2}} \int_{}^{} |f|^{2} 1_{\{|f| < n\}} \, d \mu  < \infty.
\end{equation*}

\newpage

$\textbf{Exercise 3:}$

Prove or disprove: suppose that $f, g: \mathbb{R} \rightarrow \mathbb{R}$, with $f$ being a measurable function, and $g$ being a continuous function. Then $f \circ g$ is measurable. By definition, $(f \circ g)(x) = f(g(x))$, that is, it is the composition of the two functions.


\vspace{8pt}
$\textbf{Solution:}$

No, the statement is not true and we can find a counter example as follows. Suppose that $C$ is the Cantor set and we define a mapping $\phi$: for any $x \in C$, let $0.c_{1}c_{2}c_{3} \cdots$ be its ternary expansion, where $c_{n} = 0$ or $c_{n} = 2$, $n = 1, 2, \cdots$ and let
\begin{equation*}
    \phi(x) = 0.\frac{c_{1}}{2}\frac{c_{2}}{2}\frac{c_{3}}{2}\cdots,
\end{equation*}
where the expansion on the right is now interpreted as a binary expansion in terms of digits $0$ and $1$. It is clear that the image of $C$, under $\phi$, is a subset of $[0, 1]$. And next we extend the domain to the entire unit interval $[0, 1]$. If $x \in [0, 1] \setminus C $, then $x$ is a member of one of the open intervals $(a, b)$ removed from $[0, 1]$ in the construction of $C$, and therefore $\phi(a) = \phi(b)$. And we define $\phi(x) = \phi(a) = \phi(b)$. Since $\phi(\cdot)$ is increasing on $[0, 1]$, and since the range of $\phi(\cdot)$ is the entire interval $[0, 1]$, $\phi(\cdot)$ has no jump discontinuities. Since a monotonic function can have no discontinuities other than jump discontinuities, we know that $\phi(\cdot)$ is continuous. Then we define
$$\varphi (x) = x + \phi(x), \, x \in [0, 1]$$
with range $[0, 2]$. Since $\phi(\cdot)$ is increasing on $[0, 1]$ and continuous there, $\varphi$ is strictly increasing and topological there (continuous and one-to-one with a continuous inverse on the range $\varphi$). Since each open interval removed from $[0,1]$ in the construction of the Cantor set $C$ is mapped by $\varphi$ onto an interval of $[0, 2]$ of the equal length, $\mu(\varphi(I \setminus C)) = \mu(I \setminus C) = 1$. Since $C$ is a set of measure zero, $\varphi$ is an example of a topological mapping that maps a set of measure zero onto a set of positive measure.

Now let $D$ is a non-measurable subset of $\varphi(C)$ and let $E = \varphi^{-1}(D)$. Then the characteristic function $f = 1_{E}(x)$ of the set $E$ is measurable and $g = \varphi^{-1}$ is continuous, but the composite function $f(g(x))$ is non-measurable characteristic function of the non-measurable set $D$.

$\textbf{Claim:}$ suppose that $f, g: \mathbb{R} \rightarrow \mathbb{R}$, with $f$ being a measurable function, and $g$ being a continuous function. Then $g \circ f$ is measurable.

$\textbf{Proof:}$ Since $f: (\mathbb{R}, \mathcal{B}_{\mathbb{R}}) \rightarrow (\mathbb{R}, \mathcal{B}_{\mathbb{R}})$ is Lebesgue-measurable and as $g: \mathbb{R} \rightarrow \mathbb{R}$ is continuous, it is Borel-measurable. Take any $B \in \mathcal{B}_{\mathbb{R}}$, we want to show that $(g \circ f)^{-1}(B) \in \mathcal{B}_{\mathbb{R}}$. By measurability of $g$, since $B \in \mathcal{B}_{\mathbb{R}}$, we have $B^{'} = g^{-1}(B) \in \mathcal{B}_{\mathbb{R}}$. By the measurability of $f$, this implies that $f^{-1}(B^{'}) \in \mathcal{B}_{\mathbb{R}}$. This shows that $g \circ f$ is measurable for the $\sigma$-algebras $\mathcal{B}_{\mathbb{R}}$.

\newpage

\section{GCE January, 2016}

$\textbf{Exercise 1:}$

Let $f_{n}$ be a sequence of continuous functions from $[0, 1]$ to $\mathbb{R}$ which is uniformly convergent. Let $x_{n}$ be in $[0, 1]$ such that $f_{n}(x_{n}) \geq f_{n}(x)$, for all $x$ in $[0, 1]$.

(i) Is the sequence $x_{n}$ convergent?

(ii) Show that the sequence $f_{n}(x_{n})$ is convergent.

\vspace{8pt}

$\textbf{Solution:}$

(i) No, the sequence $x_{n}$ may not convergent. We assume that $f_{n} (x) = 0$ for all $x \in [0, 1]$. And for any $k \in \mathbb{N}$ we set the sequence $x_{n}$ is
\begin{equation*}
x_{n} =
\left\{
             \begin{array}{cl}
             0, & n = 2k \\
             1, & n = 2k -1,
             \end{array}
\right.
\end{equation*}
Then we know that $x_{n} \in [0, 1]$ and $f_{n}(x_{n}) = 0 = f_{n}(x)$ for any $x \in [0, 1]$, but the sequence $x_{n}$ is not convergent.

(ii) We suppose $f_{n}$ is uniformly converges to $f$ on $[0, 1]$. Since $f_{n}$ is continuous, then $f$ is also a continuous function. For any $y \in [0, 1]$, there exist a $x$, such that $f(y) \leq f(x)$. And since $f_{n}$ is uniformly converges to $f$ on $[0, 1]$, for any $\epsilon > 0$, there exists a $N_{1} \in \mathbb{N}$, when $n > N_{1}$, for any $y \in [0, 1]$, we have
\begin{equation*}
    |f_{n}(y) - f(y)| < \epsilon,
\end{equation*}
which is equivalent to $f(y) - \epsilon < f_{n}(y) < f(y) + \epsilon$. We use the $x_{n}$ to substitute the $y$, then we have $f_{n}(x_{n}) \leq f(x_{n}) + \epsilon \leq f(x) + \epsilon$. 

On the other hand, for the above $x$, we have $f_{n}(x_{n}) \geq f_{n}(x)$. As $f_{n}$ is uniformly converges to $f$ on $[0, 1]$, for the above $\epsilon > 0$, there exists a $N_{2} \in \mathbb{N}$, when $n > N_{2}$, for the above $x$, we have $f_{n}(x) > f(x) - \epsilon$. And then we have $f_{n}(x_{n}) > f(x) - \epsilon$. Thus for the above $\epsilon$ and $x$, there exists a $N^{*}$, which is the biggest one we related, then when $n > N^{*}$, we have
\begin{equation*}
    f(x) - \epsilon < f_{n}(x_{n}) < f(x) + \epsilon .
\end{equation*}
So, we know that the sequence $f_{n}(x_{n})$ is convergent.

\vspace{8pt}

\noindent\rule[0.25\baselineskip]{\textwidth}{0.5pt}

$\textbf{Exercise 2:}$

Let $\mathbb{I}$ be the set of all irrational number $(\mathbb{I} \subset \mathbb{R})$.

(i) Using that $\mathbb{Q} = \mathbb{R} \setminus \mathbb{I}$ (the set of all rationals) is countable, show that given $\epsilon > 0$, there is a closed subset $F \subset \mathbb{I}$ such that $|\mathbb{I} \setminus F| < \epsilon$.

(ii) Is $F$ compact? Please explain why or why not.
 
\vspace{8pt}
$\textbf{Solution:}$

(i) We rearrange the rational number and denote it as $\{a_{n}\}_{n=1}^{\infty}$. It is a countable set. For $\epsilon > 0$, and for each $a_{n} \in \mathbb{Q}$, we can find an open set
\begin{equation*}
    a_{n} \in (a_{n} - \frac{\epsilon}{2^{n+1}}, a_{n} + \frac{\epsilon}{2^{n+1}}),
\end{equation*}
then we know that $\cup_{n=1}^{\infty} (a_{n} - \frac{\epsilon}{2^{n+1}}, a_{n} + \frac{\epsilon}{2^{n+1}}) $ is an open coverage of $\mathbb{Q}$, and
\begin{equation*}
    \Big{|} \cup_{n=1}^{\infty} (a_{n} - \frac{\epsilon}{2^{n+1}}, a_{n} + \frac{\epsilon}{2^{n+1}}) \Big{|} \leq \sum_{n=1}^{\infty} \frac{\epsilon}{2^{n}} = \epsilon.
\end{equation*}
We denote $S = \cup_{n=1}^{\infty} (a_{n} - \frac{\epsilon}{2^{n+1}}, a_{n} + \frac{\epsilon}{2^{n+1}})$, then $\mathbb{R} \setminus S \subset \mathbb{R} \setminus \mathbb{Q} = \mathbb{I}$. We set $F = \mathbb{R} \setminus S$, as $S$ is an open set, then $F$ is closed. And we have
\begin{equation*}
    |\mathbb{I} \setminus F| = |\mathbb{I}| - |\mathbb{R} \setminus S| = |\mathbb{I}| - |\mathbb{R} | + |S| < \epsilon.
\end{equation*}

(ii) No, $F$is not a compact set. Suppose $F$ is compact, then $F$ is closed and bounded, thus $F$ has finite measure. Since we have $(\mathbb{I} \setminus F) \cup F$, then there exists a $M > 0$ such that 
\begin{equation*}
    |\mathbb{I}| = |(\mathbb{I} \setminus F) \cup F| \leq |\mathbb{I} \setminus F| + |F| < \epsilon + M,
\end{equation*}
which is contradictory with $|\mathbb{I}| = \infty$. Thus $F$ is not compact.

\vspace{8pt}

\noindent\rule[0.25\baselineskip]{\textwidth}{0.5pt}

$\textbf{Exercise 3:}$

Find with proof:
\begin{equation*}
    \lim_{n \to \infty} \int_{0}^{1} \frac{1 + n x^{3}}{(1 + x^{2})^{n}} \, d x 
\end{equation*}

\vspace{8pt}
$\textbf{Solution:}$

For $x \in (0, 1)$, we denote $f_{n}(x) = \frac{1 + n x^{3}}{(1 + x^{2})^{n}}$. Firstly, for $x \in (0, 1)$, since $(1 + x^{2})^{n} \geq 1 + n x^{2}$, then we have
\begin{equation*}
    f_{n}(x) \leq \frac{1 + n x^{3}}{1 + n x^{2}} \leq 1 \in L^{1}((0, 1)).
\end{equation*}
And for $x \in (0, 1)$, since $(1 + x^{2})^{n} \geq \frac{1}{2}n(n-1) x^{4}$, we have
\begin{equation*}
    f_{n}(x) = \frac{1 + n x^{3}}{(1 + x^{2})^{n}} \leq \frac{2 + 2 n x^{3}}{n(n-1) x^{4}}  = \frac{\frac{2}{x^{4}}}{n(n - 1)} + \frac{\frac{1}{x}}{n-1},
\end{equation*}
so for any fixed $x \in (0, 1)$, we have $\lim_{n \to \infty} f_{n} (x) = 0$, thus we know that $f_{n}(x)$ converges to $0$ pointwise. By the dominate convergence theorem, we have
\begin{equation*}
    \lim_{n \to \infty} \int_{0}^{1} \frac{1 + n x^{3}}{(1 + x^{2})^{n}} \, d x = \int_{0}^{1} \lim_{n \to \infty} \frac{1 + n x^{3}}{(1 + x^{2})^{n}} \, d x = 0.
\end{equation*}

\vspace{8pt}

\noindent\rule[0.25\baselineskip]{\textwidth}{0.5pt}

$\textbf{Exercise 4:}$

Let $(X, \mathcal{A}, \mu)$ be a measure space such that $\mu(X) = 1$. Let $f$ be in $L^{1}(X)$ such that $f \geq 0$ almost everywhere.

(i) show that
\begin{equation*}
    \lim_{p \to 0^{+}} \int_{}^{} f^{p} = \mu(\{x \in X:  f(x) > 0\})
\end{equation*}

(ii) If $\mu(\{x \in X:  f(x) > 0\}) < 1$, find
\begin{equation*}
    \lim_{p \to 0^{+}} \Big{(} \int_{}^{} f^{p} \Big{)}^{\frac{1}{p}}. 
\end{equation*}


\vspace{8pt}
$\textbf{Solution:}$

(i) Since
\begin{eqnarray*}
    \int_{X}^{} f^{p} \, d \mu & = & \int_{\{x \in X: f > 0\}}^{} f^{p} \, d \mu + \int_{\{x \in X: f = 0\}}^{} f^{p} \, d \mu \\
    & = & \int_{\{x \in X: f > 0\}}^{} f^{p} \, d \mu,
\end{eqnarray*}
as $f$ be in $L^{1}(X)$ and $f \geq 0$ almost everywhere, by the Fatou's lemma,
\begin{equation*}
    \mu(\{x \in X:  f(x) > 0\}) = \int_{}^{} \mathbb{I}_{\{x \in X: f > 0\}} (x) \, d \mu \leq \liminf_{p \to 0^{+}} \int_{\{x \in X: f > 0\}}^{} f^{p} \, d \mu.
\end{equation*}
On the other hand, we know that
\begin{eqnarray*}
    \int_{\{x \in X: f > 0\}}^{} f^{p} \, d \mu & = & \int_{\{x \in X: 0 < f < n\}}^{} f^{p} \, d \mu  + \int_{\{x \in X: f \geq n\}}^{} f^{p} \, d \mu \\
    & \leq & \int_{\{x \in X: f \geq n\}}^{} f^{p} \, d \mu + n^{p} \mu(\{x \in X:  f(x) > 0\}).
\end{eqnarray*}
For $0 < p < 1$, when $x \in \{x \in X:  f(x) > n\}$, we have $f^{p} < f$, thus we have
\begin{eqnarray*}
    \limsup_{p \to 0^{+}} \int_{\{x \in X: f > 0\}}^{} f^{p} \, d \mu & \leq &  \mu(\{x \in X:  f(x) > 0\}) + \limsup_{p \to 0^{+}} \int_{\{x \in X: f \geq n\}}^{} f^{p} \, d \mu \\
    & \leq & \mu(\{x \in X:  f(x) > 0\}) + \int_{\{x \in X: f \geq n\}}^{} f \, d \mu \\
    & \leq & \mu(\{x \in X:  f(x) > 0\}) + \int_{X}^{} f \, \mathbb{I}_{\{x \in X: f \geq n\}}(x) \, d \mu
\end{eqnarray*}
Since $f \cdot \mathbb{I}_{\{x \in X: f \geq n\}}(x) \leq f \in L^{1}(X)$ and $\lim_{n \to \infty} f \, \mathbb{I}_{\{x \in X: f \geq n\}}(x) = 0 $, by the dominate convergence theorem, we have
\begin{equation*}
    \limsup_{p \to 0^{+}} \int_{\{x \in X: f > 0\}}^{} f^{p} \, d \mu \leq \mu(\{x \in X:  f(x) > 0\}),
\end{equation*}
thus we know that
\begin{equation*}
    \lim_{p \to 0^{+}} \int_{}^{} f^{p} = \mu(\{x \in X:  f(x) > 0\}).
\end{equation*}

(ii) 
\textbf{Method 1:}

As $\mu(X) = 1$ and $f \in L^{1}(X)$, we know that $f \in L^{\infty} (X)$. We denote $S = \{x \in X: f > 0\}$, then
\begin{eqnarray*}
    \int_{X}^{} f^{p} \, d \mu & = & \int_{\{x \in X: f > 0\}}^{} f^{p} \, d \mu + \int_{\{x \in X: f = 0\}}^{} f^{p} \, d \mu \\
    & = & \int_{S}^{} f^{p} \, d \mu \\
    & \leq & \int_{S}^{} \|f\|_{\infty}^{p} \, d \mu \\
    & = & \|f\|_{\infty}^{p} \mu(S),
\end{eqnarray*}
thus we have
\begin{equation*}
    \lim_{p \to 0^{+}} \Big{(} \int_{}^{} f^{p} \Big{)}^{\frac{1}{p}} \leq \lim_{p \to 0^{+}} \|f\|_{\infty} (\mu(S))^{\frac{1}{p}} = 0
\end{equation*}
as $\mu(S) < 1$. 

\textbf{Method 2:}

We denote $S = \{x \in X: f > 0\}$, then
\begin{eqnarray*}
    \int_{X}^{} f^{p} \, d \mu & = & \int_{\{x \in X: f > 0\}}^{} f^{p} \, d \mu + \int_{\{x \in X: f = 0\}}^{} f^{p} \, d \mu \\
    & = & \int_{S}^{} f^{p} \, d \mu.
\end{eqnarray*}
And we denote that $F(p) = \log (\int_{S}^{} f^{p} \, d \mu)$, then we know that
\begin{equation*}
    \lim_{p \to 0^{+}} \Big{(} \int_{}^{} f^{p} \Big{)}^{\frac{1}{p}} = \lim_{p \to 0^{+}} e^{\frac{F(p)}{p}}.
\end{equation*}
As $F(0) = \log(\mu(S))$, then we have
\begin{eqnarray*}
    \lim_{p \to 0^{+}} \Big{(} \int_{}^{} f^{p} \Big{)}^{\frac{1}{p}} & = & \lim_{p \to 0^{+}} \exp \Big{\{} \frac{F(p) - \log(\mu(S)) + \log(\mu(S))}{p} \Big{\}} \\
    & = & \lim_{p \to 0^{+}} (\mu(S))^{\frac{1}{p}} \exp \Big{\{} \frac{F(p) - \log(\mu(S))}{p - 0} \Big{\}}.
\end{eqnarray*}
As $F(p) = \log (\int_{S}^{} f^{p} \, d \mu)$, we have 
\begin{equation*}
    F^{'}(p) = \frac{\int_{S}^{} f^{p} \cdot \log f \, d \mu}{\int_{S}^{} f^{p} \, d \mu},
\end{equation*}
thus we have $F^{'}(0) = \frac{\int_{S}^{} \log f \, d \mu}{\mu(S)}$. Then we know that
\begin{eqnarray*}
    \lim_{p \to 0^{+}} \Big{(} \int_{}^{} f^{p} \Big{)}^{\frac{1}{p}} & = & \lim_{p \to 0^{+}} (\mu(S))^{\frac{1}{p}} \exp \Big{\{} \lim_{p \to 0^{+}} \frac{F(p) - F(0)}{p - 0} \Big{\}} \\
    & = & \lim_{p \to 0^{+}} (\mu(S))^{\frac{1}{p}} e^{F^{'}(0)} \\
    & = & 0
\end{eqnarray*}
as $\mu(S) < 1$.

\newpage

\section{GCE May, 2016}

$\textbf{Exercise 1:}$

A real-valued function $f$ is increasing on a closed interval $[a, b] \subset \mathbb{R}$ if and only if $f(x_{2}) \geq f(x_{1})$ whenever $x_{2} > x_{1}$.

(i) Using the definition of measurable, show that $f$ is measurable on $[a, b]$.

(ii) Show that $f$ is continuous, except perhaps a countable number of points.

\vspace{8pt}

$\textbf{Solution:}$

(i) For any $c \in \mathbb{R}$, we denote $S = f^{-1} ([c, +\infty])$, by the definition of $S$, we know that $S = \{x \in [a, b] | f(x) \geq c \}$. For any $x \in S$, if $y > x$ and $y \in [a, b]$, as $f$ is increasing, we have $f(y) \geq f(x) \geq c$. So, we have $y \in S$. It is equivalent to that if $x \in S$, for any $y \in [a, b]$ and $y \geq x$, we have $y \in S$. This means $S$ can only be $\empty$, $[a, b]$, $(a, b]$, $[\inf S, b]$ and $(\inf S, b]$, all of the sets are measurable, thus we know that $f$ is measurable.

\vspace{4pt}

(ii) Let $f(x^{-})$ and $f(x^{+})$ denote the left and the right hand limits of $f$ respectively. Let $A$
be the set of points where $f$ is not continuous. Then for any $x \in A \subset [a, b]$, we can find a rational number $f^{*}(x) \in \mathbb{Q}$, such that $f(x^{-}) < f(x^{*}) < f(x^{+})$. Since $f$ is increasing function, then for $x_{1}, x_{2} \in A$ and $x_{1} < x_{2}$, we have $f(x_{1}) \leq f(x_{2})$, also we have $f(x_{1}^{+}) \leq f(x_{2}^{-})$. Thus we have $f(x_{1}^{*}) < f(x_{1}^{+}) \leq f(x_{2}^{-}) < f(x_{2}^{*})$, then we know that $f(x_{1}^{*}) < f(x_{2}^{*})$. Then there exists a injection between $A$ and a subsets of rational number $\mathbb{Q}$. Since $\mathbb{Q}$ is countable, then we know that $A$ is also countable. Thus $f$ is continuous except perhaps a countable number of points.


\noindent\rule[0.25\baselineskip]{\textwidth}{0.5pt}

\vspace{8pt}
$\textbf{Exercise 2:}$

If $f$ is Lebesgue integrable on $\mathbb{R}$, define
\begin{equation*}
    F(x) = \int_{0}^{x} f \, d \mu
\end{equation*}
where $\mu(E)$ is the Lebesgue measurable set $E \subset \mathbb{R}$. Show that

(i) $F$ is continuous.

(ii) If $\mu(E) = 0$, then $\mu(F(E)) = 0$. 
 

\vspace{8pt}
$\textbf{Solution:}$

(i) Suppose $\{x_{n}\}$ is a sequence and $x_{n} \to x_{0}$ as $n$ goes to infinity. Then we need to show that $F(x_{n})$ converges to $F(x_{0})$, i.e.
\begin{equation*}
    \lim_{n \to + \infty} \int_{0}^{x_{n}} f \, d \mu = \int_{0}^{x_{0}} f \, d \mu.
\end{equation*}
Since we have
\begin{equation*}
    \lim_{n \to + \infty} \int_{0}^{x_{n}} f \, d \mu = \lim_{n \to + \infty} \int_{0}^{\infty} f \, \mathbb{I}_{[0, x_{n}]} (x) \, d \mu
\end{equation*}
and
\begin{equation*}
    |f \, \mathbb{I}_{[0, x_{n}]} (x) | \leq |f| \in L^{1}(\mathbb{R}),
\end{equation*}
by the dominate convergence theorem, we have
\begin{equation*}
     \lim_{n \to + \infty} \int_{0}^{\infty} f \, \mathbb{I}_{[0, x_{n}]} (x) \, d \mu =  \int_{0}^{\infty}  \lim_{n \to + \infty} f \, \mathbb{I}_{[0, x_{n}]} (x) \, d \mu.
\end{equation*}
Next we need to show that
\begin{equation*}
    \lim_{n \to + \infty} \, \mathbb{I}_{[0, x_{n}]} (x) = \mathbb{I}_{[0, x_{0}]} (x).
\end{equation*}
If $x_{n} \to x_{0}$, then for any $0 < t < x_{0}$, there exists a $N_{1} \in \mathbb{N}$, such that $t < x_{n}$ for any $n > N_{1}$, and hence we have $\mathbb{I}_{[0, x_{n}]} (t) = 1$ for all $n > N_{1}$. Similarly, for $t > x_{0}$, there exists a $N_{2} \in \mathbb{N}$ such that $\mathbb{I}_{[0, x_{n}]} (t) = 0$ for all $n > N_{2}$. Since $\{x_{0}\}$ is a singleton, which has zero measure,, thus we have 
\begin{equation*}
    \lim_{n \to + \infty} \, \mathbb{I}_{[0, x_{n}]} (x) = \mathbb{I}_{[0, x_{0}]} (x) \, a.e.
\end{equation*}
Then we have
\begin{equation*}
     \lim_{n \to + \infty} \int_{0}^{\infty} f \, \mathbb{I}_{[0, x_{n}]} (x) \, d \mu =  \int_{0}^{\infty} f \, \mathbb{I}_{[0, x_{0}]} (x) \, d \mu = \int_{0}^{x_{0}} f \, d \mu,
\end{equation*}
from which we know $F$ is continuous.

(ii) We need to show that the continuous image of a zero measure set is also a zero measure set. For $E \in \mathbb{R}$ and $\mu(E) = 0$, we can find a disjoint sequence ${E_{n}}$ such that $E \subset \cup_{n=1}^{\infty} E_{n}$ and for any $\epsilon > 0$ we have $\mu(\cup_{n=1}^{\infty} E_{n}) < \epsilon$. And then we have $F(E) \subset F(\cup_{n=1}^{\infty} E_{n}) $. Then we know that
\begin{equation*}
    \mu(F(E)) \leq \mu(F(\cup_{n=1}^{\infty} E_{n})) .
\end{equation*}
Since $F$ is continuous, if $f$ is lipchitz continuous or $f$ is absolutely continuous, then there exists a constant $K > 0$ and we have $\mu(F(\cup_{n=1}^{\infty} E_{n}))  \leq K \mu(\cup_{n=1}^{\infty} E_{n}) < K \epsilon$. So, we know that $\mu(F(E)) = 0$.
 
\noindent\rule[0.25\baselineskip]{\textwidth}{0.5pt}

\vspace{8pt}

$\textbf{Exercise 3:}$

Let $f$ be in $L^{1}(\mathbb{R})$ such that $f \geq 0$ almost everywhere and $\int_{\mathbb{R}}^{} f = 1$. Set $f_{n} (x) = n f(n x)$. Let $g$ be in $L^{\infty}(\mathbb{R})$.

(i) Let $x_{0}$ be in $\mathbb{R}$. Assume that $g$ is continuous at $x_{0}$. show that
\begin{equation*}
   \lim_{n \to \infty} \int_{\mathbb{R}}^{} f_{n}(x_{0} - y) g(y) \, d y = g(x_{0}).
\end{equation*}

(ii) If $g$ is uniformly continuous, is this limit uniformly in $x_{0}$?

(iii) If $h$ is in $L^{1}(\mathbb{R})$ show that the function in $x$
\begin{equation*}
    \int_{\mathbb{R}}^{} f_{n} (x - y) h(y) \, d y
\end{equation*}
converges to $h$ in $L^{1}(\mathbb{R})$.

\vspace{8pt}
$\textbf{Solution:}$

(i) We denote $z = x_{0} - y$, so we have
\begin{equation*}
    \int_{\mathbb{R}}^{} f_{n}(x_{0} - y) g(y) \, d y = \int_{\mathbb{R}}^{} f_{n}(z) g(x_{0} - z) \, d z = \int_{\mathbb{R}}^{} n f(n z) g(x_{0} - z) \, d z,
\end{equation*}
and then we denote $u = n z$,
\begin{equation*}
    \int_{\mathbb{R}}^{} n f(n z) g(x_{0} - z) \, d z = \int_{\mathbb{R}}^{}  f(u) g(x_{0} - \frac{u}{n}) \, d u.
\end{equation*}
Since $f \in L^{1} (\mathbb{R})$ and $g(x) \in L^{\infty}(\mathbb{R})$, there exists a $M > 0$ such that
\begin{equation*}
    |f(u) g(x_{0} - \frac{u}{n})| \leq M f(u) \in L^{1}(\mathbb{R}),
\end{equation*}
by the dominate convergence theorem, we have
\begin{eqnarray*}
    \lim_{n \to \infty} \int_{\mathbb{R}}^{} f_{n}(x_{0} - y) g(y) \, d y & = &  \lim_{n \to \infty} \int_{\mathbb{R}}^{}  f(u) g(x_{0} - \frac{u}{n}) \, d u \\
    & = & \int_{\mathbb{R}}^{} \lim_{n \to \infty} f(u) g(x_{0} - \frac{u}{n}) \, d u \\
    & = & \int_{\mathbb{R}}^{} f(u) g(x_{0}) \, d u \\
    & = & g(x_{0})
\end{eqnarray*}
as $g$ is continuous at $x_{0}$.

\vspace{4pt}

(ii) We need to show that $\int_{\mathbb{R}}^{} f_{n}(x - y) g(y) \, d y$ is uniformly converges to $g(x)$ when $g$ is uniformly continuous on $\mathbb{R}$. By the definition of $f_{n}(x)$, we have
\begin{equation*}
    \int_{\mathbb{R}}^{} f_{n}(x) \, d x = \int_{\mathbb{R}}^{} n f(n x) \, d x = \int_{\mathbb{R}}^{} f(n x) \, d (n x) = 1.
\end{equation*}
For any $x \in \mathbb{R}$,
\begin{eqnarray*}
    \Big{|} \int_{\mathbb{R}}^{} f_{n}(x - y) g(y) \, d y - g(x) \Big{|} & = &  \Big{|} \int_{\mathbb{R}}^{}  f_{n}(z) g(x - z) \, d z - g(x) \Big{|} \\
    & = & \Big{|} \int_{\mathbb{R}}^{}  f_{n}(z) g(x - z) \, d z - \int_{\mathbb{R}}^{}  f_{n}(z) g(x) \, d z \Big{|} \\
    & \leq &  \int_{\mathbb{R}}^{}  f_{n}(z) | g(x - z) - g(x) | \, d z \\
    & = & \int_{\mathbb{R}}^{} n f(n z) | g(x - z) - g(x) | \, d z,
\end{eqnarray*}
we denote $ u = n z$, then we have
\begin{equation*}
    \Big{|} \int_{\mathbb{R}}^{} f_{n}(x - y) g(y) \, d y - g(x) \Big{|} \leq \int_{\mathbb{R}}^{}  f(u) \Big{|} g \Big{(} x - \frac{u}{n} \Big{)} - g(x) \Big{|} \, d u.
\end{equation*}
As $f \in L^{1}(\mathbb{R})$ and $g \in L^{\infty}(\mathbb{R})$, there exists a $M > 0$ such that
\begin{equation*}
    \Big{|} f(u) \Big{(} g(x - \frac{u}{n}) - g(x)\Big{)} \Big{|} \leq 2M f(u) \in L^{1}(\mathbb{R}),
\end{equation*}
by the dominate convergence theorem, we have
\begin{equation*}
    \lim_{n \to \infty} \Big{|} \int_{\mathbb{R}}^{} f_{n}(x - y) g(y) \, d y - g(x) \Big{|} \leq \int_{\mathbb{R}}^{}  \lim_{n \to \infty} f(u) \Big{|} g \Big{(} x - \frac{u}{n} \Big{)} - g(x) \Big{|} \, d u.
\end{equation*}
Since $g$ is uniformly continuous on $\mathbb{R}$, for any $x \in \mathbb{R}$, and for any $\epsilon > 0$, there exists a $N \in \mathbb{N}$, which is independent of $x$, such that when $n > N$, we have $g ( x - \frac{u}{n} ) - g(x) < \epsilon $. So, for the above $\epsilon$ and $N$, when $n > N$ we have
\begin{equation*}
    \int_{\mathbb{R}}^{} f(u) \Big{|} g \Big{(} x - \frac{u}{n} \Big{)} - g(x) \Big{|} \, d u \leq \int_{\mathbb{R}}^{} f(u) \epsilon \, d u = \epsilon
\end{equation*}
thus we know that $\int_{\mathbb{R}}^{} f_{n}(x - y) g(y) \, d y$ is uniformly converges to $g(x)$.

\vspace{4pt}

(iii) As $h \in L^{1}(\mathbb{R})$ and $C_{c} (\mathbb{R})$ is dense in $L^{1} (\mathbb{R})$, for any $\epsilon > 0$, there exists  a function $g \in C_{c}(\mathbb{R})$, such that
\begin{equation*}
    \|g - h\|_{1} < \epsilon.
\end{equation*}
We denote $\int_{\mathbb{R}}^{} f_{n} (x - y) h(y) \, d y = h_{n}(x)$ and $\int_{\mathbb{R}}^{} f_{n} (x - y) g(y) \, d y = g(x)$, then we have
\begin{equation*}
    \|h(x) - h_{n}(x)\|_{1} \leq \|h(x) - g(x)\| + \|g(x) - g_{n}(x)\| + \|g_{n}(x) - h_{n}(x)\|.
\end{equation*}
For the above $\epsilon$, as $\|g - h\|_{1} < \epsilon$ and by the result we get from (ii), $g_{n}(x)$ is uniformly converges to $g(x)$, we have $\|g_{n}(x) - g(x)\| < \epsilon$, then we have
\begin{equation*}
    \lim_{n \to \infty} \|h(x) - h_{n}(x)\|_{1} = \lim_{n \to \infty} \|g_{n}(x) - h_{n}(x)\|.
\end{equation*}
Next we need to verify the term $\|g_{n}(x) - h_{n}(x)\|$, since
\begin{eqnarray*}
    \|g_{n}(x) - h_{n}(x)\| & = & \Big{\|} \int_{\mathbb{R}}^{} f_{n} (x - y) h(y) \, d y - \int_{\mathbb{R}}^{} f_{n} (x - y) g(y) \, d y \Big{\|} \\
    & = & \int_{\mathbb{R}}^{} \Big{|}  \int_{\mathbb{R}}^{} f_{n} (x - y) ( h(y) - g(y)) \, d y \Big{|} \, d x \\
    & \leq & \int_{\mathbb{R}}^{} \int_{\mathbb{R}}^{} f_{n} (x - y) | h(y) - g(y)| \, d y \, d x \\
    & = & \int_{\mathbb{R}}^{} \int_{\mathbb{R}}^{} f (u) \Big{|} h \Big{(} x - \frac{u}{n} \Big{)} - g \Big{(} x - \frac{u}{n} \Big{)} \Big{|} \, d u \, d x,
\end{eqnarray*}
by Fubini's theorem, we have
\begin{equation*}
    \|g_{n}(x) - h_{n}(x)\| \leq \int_{\mathbb{R}}^{} f (u)  \int_{\mathbb{R}}^{}  \Big{|} h \Big{(} x - \frac{u}{n} \Big{)} - g \Big{(} x - \frac{u}{n} \Big{)} \Big{|} \, d x \, d u.
\end{equation*}
Since $f \in L^{1}(\mathbb{R})$, $h \in L^{1}(\mathbb{R})$ and $g \in C_{c}(\mathbb{R})$, there exists a $M > 0$ such that 
\begin{equation*}
    f (u) \Big{|} h \Big{(} x - \frac{u}{n} \Big{)} - g \Big{(} x - \frac{u}{n} \Big{)} \Big{|} \leq 2 M f(u) \in L^{1}(\mathbb{R}),
\end{equation*}
by the dominate convergence theorem, we have
\begin{eqnarray*}
    \lim_{n\to \infty} \|g_{n}(x) - h_{n}(x)\| & \leq & \int_{\mathbb{R}}^{} f (u) \lim_{n\to \infty}  \int_{\mathbb{R}}^{}  \Big{|} h \Big{(} x - \frac{u}{n} \Big{)} - g \Big{(} x - \frac{u}{n} \Big{)} \Big{|} \, d x \, d u \\ & = & \int_{\mathbb{R}}^{} f (u) \lim_{n\to \infty}  \Big{\|} h \Big{(} x - \frac{u}{n} \Big{)} - g \Big{(} x - \frac{u}{n} \Big{)} \Big{\|} \, d u = 0.
\end{eqnarray*}
Thus we know that
\begin{equation*}
    \lim_{n \to \infty} \|h(x) - h_{n}(x)\|_{1} = 0,
\end{equation*}
which means $\int_{\mathbb{R}}^{} f_{n} (x - y) h(y) \, d y$ converges to $h$ in $L^{1}(\mathbb{R})$.

\newpage

\section{GCE August, 2016}

$\textbf{Exercise 1:}$

Suppose that $u$ is a real-valued function defined on $[0, 1]$, that $u \geq 0$ and that $u \in L^{1}([0, 1])$. Define $E_{n} : = \{x \in [0, 1]: n - 1 \leq u(x) \leq n \}$ for each positive integer $n$. Show that
\begin{equation*}
    \sum_{n = 1}^{\infty} n |E_{n}| < + \infty.
\end{equation*}

\vspace{8pt}

$\textbf{Solution:}$

As $u \in L^{1}([0, 1])$ and $u(x) \geq 0$, we have
\begin{equation*}
    \int_{0}^{1} |u (x)| \, d x = \int_{0}^{1} u (x) \, d x < + \infty.
\end{equation*}
And since
\begin{eqnarray*}
\int_{0}^{1} u (x) \, d x  & = & \sum_{n = 1}^{\infty} \int_{E_{n}}^{} u(x) \, d x \\
& \geq & \sum_{n = 1}^{\infty} (n -1) |E_{n}| \\
& = & \sum_{n = 1}^{\infty} n |E_{n}| + \sum_{n = 1}^{\infty} |E_{n}|,
\end{eqnarray*}
and $\sum_{n = 1}^{\infty} |E_{n}| < + \infty$, then we have 
\begin{equation*}
    \sum_{n = 1}^{\infty} n |E_{n}| < + \infty.
\end{equation*}


\noindent\rule[0.25\baselineskip]{\textwidth}{0.5pt}

\vspace{8pt}
$\textbf{Exercise 2:}$

Show that a subset $E$ of a metric space $X$ is open if and only if there is a continuous real-valued function $f$ on $X$ such that $E = \{x \in x : f(x) > 0 \}$.
 
\vspace{8pt}
$\textbf{Solution:}$

If there is a continuous real-valued function $f$ on $X$ such that $E = \{x \in x : f(x) > 0 \}$, we want to show that $E$ is an open set. Since $(0, + \infty)$ is an open set, $E = \{x \in x : f(x) > 0 \} = f^{-1} ((0, + \infty))$ is also an open set as $f$ is continuous on $X$. We can also verify the statement by definition. Suppose $y \in E$, since $E = \{x \in x : f(x) > 0 \}$, we have $f(y) > 0$. Since $f$ in continuous on $X$, we know that there exists a $\delta$ such that when $d(x,y) < \delta$, then $|f(x) - f(y)| < f(y)$, which implies $- f(y) < f(x) - f(y) < f(y)$, hence we have $f(x) > 0$. Then we know that there exists a $\delta > 0$, when $x \in B_{\delta} (y)$, we have $f(x) > 0$. Thus for any $y \in E$, there exists a $\delta$, and $B_{\delta} (y) \subset E$. So we know that $E$ is an open set.

On the other direction, we want to show that if $E \subset X$ is open, there exists a continuous function $f$ on $X$ such that $E = \{x \in x : f(x) > 0 \}$. For $E \in X$, we denote
\begin{equation*}
    f(x) = d(x, E^{c}) = \min_{y \in E^{c}} d(x, y).
\end{equation*}
Then we have when $x \in E^{c}$, $f(x) = 0$ and when $x \in E$, $f(x) > 0$, so we have $E = \{x \in x : f(x) > 0 \}$. Next we need to show $f$ is continuous on $X$ . Let $x, y \in X$ and $p$ is the any point in $E^{c}$, then
\begin{equation*}
    d(x, p) \leq d(x, y) + d(y, p),
\end{equation*}
and so
\begin{equation*}
    d(x, E^{c}) \leq d(x, y) + d(y, p)
\end{equation*}
as $d(x, A)$ is the minimum. Then we have $d(y, p) \geq d(x, E^{c}) - d(x, y)$ for all $p \in E^{c}$, thus we can get that $d(y, E^{c}) \geq d(x, E^{c}) - d(x, y)$, which is equivalent to
\begin{equation*}
    d(x, E^{c}) - d(y, E^{c}) \leq d(x, y).
\end{equation*}
Similarly, we can change the position of $x$ and $y$ then get
\begin{equation*}
    d(y, E^{c}) - d(x, E^{c}) \leq d(x, y),
\end{equation*}
so we have for any $x, y \in X$,
\begin{equation*}
    | d(x, E^{c}) - d(y, E^{c})| \leq d(x, y).
\end{equation*}
Then for any $\epsilon > 0$, there exists a $\delta = \epsilon$, such that when $d(x, y) < \delta$, we have $|d(x, E^{c}) - d(y, E^{c})| < d(x, y) = \epsilon$. So, we have showed that $f$ is a continuous function on $X$.

\noindent\rule[0.25\baselineskip]{\textwidth}{0.5pt}

\vspace{8pt}

$\textbf{Exercise 3:}$

Consider the sequence of functions $\{f_{n}\}$ defined on the non-negative reals: $[0, + \infty)$ where $f_{n}(x) = 2 n x e^{-n x^{2}}$. Let $g$ be a continuous and bounded function on $[0, + \infty)$ valued in $\mathbb{R}$.

(i) Find with proof
\begin{equation*}
   \lim_{n \to \infty} \int_{0}^{\infty} f_{n}(t) g(t) \, d t.
\end{equation*}

(ii) Define for $x$ in $[0, + \infty)$,
\begin{equation*}
   g_{n}(x) = \int_{0}^{\infty} f_{n} (t) g(x + t) \, d t.
\end{equation*}
Assuming $g$ is zero outside the interval $[0, M]$, where $M > 0$, does the sequence $g_{n}$ have a limit in $L^{1}([0, + \infty))$?

(iii) If $h$ is in $L^{1}([0, + \infty))$, define for $x$ in $[0, + \infty)$,
\begin{equation*}
   h_{n}(x) = \int_{0}^{\infty} f_{n} (t) h(x + t) \, d t.
\end{equation*}
Show that $h_{n}$ is measurable on $[0, + \infty)$ and is in $L^{1} ([0, + \infty))$.

(iv) Find, if it exists, with proof, the limit of $h_{n}$ in $L^{1} ([0, + \infty))$.


\vspace{8pt}
$\textbf{Solution:}$

(i) We denote $y = n t^{2}$, then we have
\begin{equation*}
    \int_{0}^{\infty} 2 n t e^{- n t^{2}} g(t) \, d t = \int_{0}^{\infty} e^{-y} g \Big{(} \sqrt{\frac{y}{n}} \Big{)} \, d y.
\end{equation*}
Since $g(x)$ is a continuous and bounded function on $[0, + \infty)$, we suppose that $| g(x) | \leq C$ for any $x \in [0, + \infty)$. Then we know that $| e^{-y} g(\sqrt{\frac{y}{n}}) | \leq C e^{-y}$ and $C e^{-y} \in L^{1}([0, + \infty))$ as $\int_{0}^{\infty} | C e^{-y}| \, d y = C < + \infty$. And for any fixed $y \in [0, + \infty)$, when $n \to \infty$, $g(\sqrt{\frac{y}{n}}) \to g(0)$ and then $e^{-y} g(\sqrt{\frac{y}{n}}) \to e^{-y} g(0)$. By the dominate convergence theorem, we have
\begin{eqnarray*}
\lim_{n \to \infty} \int_{0}^{\infty} f_{n}(t) g(t) \, d t & = &  \int_{0}^{\infty} \lim_{n \to \infty} e^{-y} g \Big{(} \sqrt{\frac{y}{n}} \Big{)} \, d y \\
& = &  \int_{0}^{\infty} e^{-y} g(0) \, d y  \\
& = & g(0).
\end{eqnarray*}

\vspace{4pt}

(ii) Since $f_{n} (x) = 2 n x e^{-n x^{2}}$, we denote $y = n t^{2}$, then we have
\begin{equation*}
    g_{n} (x) = \int_{0}^{\infty} f_{n} (t) g(x + t) \, d t = \int_{0}^{\infty} e^{-y} g \Big{(} x + \sqrt{\frac{y}{n}} \Big{)} \, d y.
\end{equation*}
Next we want to show that $g_{n}$ converges to $g$ in $L^{1}([0, + \infty))$. Since
\begin{eqnarray*}
\int_{0}^{\infty} |g_{n}(x) - g(x)| \, d x & = &  \int_{0}^{\infty} \Big{|} \int_{0}^{\infty}  e^{-y} g \Big{(} x + \sqrt{\frac{y}{n}} \Big{)} \, d y  - g(x) \Big{|} \, d x \\
& = &  \int_{0}^{\infty} \Big{|} \int_{0}^{\infty}  e^{-y} g \Big{(} x + \sqrt{\frac{y}{n}} \Big{)} \, d y  - \int_{0}^{\infty} g(x) e^{-y} \, d y  \Big{|} \, d x  \\
& = &  \int_{0}^{\infty} \Big{|} \int_{0}^{\infty}  e^{-y} \Big{(} g \Big{(} x + \sqrt{\frac{y}{n}} \Big{)}  - g(x) \Big{)} \, d y \Big{|} \, d x \\
& \leq & \int_{0}^{\infty} \int_{0}^{\infty}  e^{-y} \Big{|} g \Big{(} x + \sqrt{\frac{y}{n}} \Big{)}  - g(x) \Big{|} \, d y \, d x,
\end{eqnarray*}
and by Fubini theorem,
\begin{eqnarray*}
\int_{0}^{\infty} \int_{0}^{\infty}  e^{-y} \Big{|} g \Big{(} x + \sqrt{\frac{y}{n}} \Big{)}  - g(x) \Big{|} \, d y \, d x & = &  \int_{0}^{\infty} \int_{0}^{M - \sqrt{\frac{y}{n}}}  e^{-y} \Big{|} g \Big{(} x + \sqrt{\frac{y}{n}} \Big{)}  - g(x) \Big{|} \, d x \, d y \\
& + & \int_{0}^{\infty} \int_{M - \sqrt{\frac{y}{n}}}^{M}  e^{-y} | g(x) | \, d x \, d y ,
\end{eqnarray*}
when $n \to \infty$, we have
\begin{equation*}
    \int_{0}^{\infty} \int_{M - \sqrt{\frac{y}{n}}}^{M}  e^{-y} | g(x) | \, d x \, d y \to 0,
\end{equation*}
thus we know that
\begin{eqnarray*}
 \lim_{n \to \infty} \int_{0}^{\infty} |g_{n}(x) - g(x)| \, d x  & \leq & \lim_{n \to \infty}  \int_{0}^{\infty} \int_{0}^{M - \sqrt{\frac{y}{n}}}  e^{-y} \Big{|} g \Big{(} x + \sqrt{\frac{y}{n}} \Big{)}  - g(x) \Big{|} \, d x \, d y \\
 & \leq & \lim_{n \to \infty}  \int_{0}^{\infty} \int_{0}^{M}  e^{-y} \Big{|} g \Big{(} x + \sqrt{\frac{y}{n}} \Big{)}  - g(x) \Big{|} \, d x \, d y.
\end{eqnarray*}
Since $e^{-y} | g ( x + \sqrt{\frac{y}{n}} )  - g(x) | \leq 2 C e^{-y}$ and $2 C e^{-y} \in L^{1}([0, + \infty))$, then by the dominate convergence theorem we have
\begin{equation*}
    \lim_{n \to \infty}  \int_{0}^{\infty} \int_{0}^{M}  e^{-y} \Big{|} g \Big{(} x + \sqrt{\frac{y}{n}} \Big{)}  - g(x) \Big{|} \, d x \, d y = 0,
\end{equation*}
thus we know that 
\begin{equation*}
    \lim_{n \to \infty} \int_{0}^{\infty} |g_{n}(x) - g(x)| \, d x  = 0.
\end{equation*}
So, we have showed that $g_{n}$ converges to $g$ in $L^{1}([0, + \infty))$.

\vspace{8pt}

(iii) Since $C_{c}([0, + \infty))$ is dense in $L^{1}([0, + \infty))$ and $h(x) \in L^{1}([0, + \infty))$, we can find a sequence $\{h^{k}\}_{k = 1}^{\infty}$ such that $h^{k} \to h$ in $L^{1}([0, + \infty))$. We want to show $h_{n}$ is measurable by showing it is the limit of a sequence of measurable functions. By the result we got from (ii), for any $k \in \mathbb{N}$, we have $h_{n}^{k} = \int_{0}^{\infty} f_{n}(t) g(t) \, d t$ converges to $h^{k}(x)$ in $L^{1}([0, + \infty))$. Firstly we show that $h_{n}^{k} (x)$ converges to $h_{n} (x)$ almost everywhere. For any $x \in [0, + \infty)$, we have
\begin{eqnarray*}
|h_{n}(x) - h_{n}^{k} (x)|  & = & \Big{|} \int_{0}^{\infty} f_{n} (t) h(x + t) \, d t - \int_{0}^{\infty} f_{n} (t) h^{k}(x + t) \, d t \Big{|} \\
 & = & \Big{|} \int_{0}^{\infty} f_{n} (t) ( h(x + t) - h^{k}(x + t)) \, d t \Big{|} \\
 & \leq & \int_{0}^{\infty} f_{n} (t) | h(x + t) - h^{k}(x + t)| \, d t,
\end{eqnarray*}
we denote $z = x + t$, then
\begin{equation*}
    |h_{n}(x) - h_{n}^{k} (x)| \leq \int_{x}^{\infty} f_{n} (z - x) | h(z) - h^{k}(z)| \, d z.
\end{equation*}
Since $f_{n}(x) = 2 n x e^{-n x^{2}}$, when $x = \frac{1}{\sqrt{2n}}$, the $f_{n}(x)$ gets the maximum value as $\sqrt{2n} e^{-\frac{1}{2}}$, thus we have
\begin{eqnarray*}
|h_{n}(x) - h_{n}^{k} (x)|  & \leq & \int_{x}^{\infty} f_{n} (z - x) | h(z) - h^{k}(z)| \, d z \\
 & \leq  & \|f_{n}\|_{\infty} \int_{x}^{\infty} |h(z) - h^{k} (z)| \, d z \\
 & \leq & \|f_{n}\|_{\infty} \int_{0}^{\infty} |h(z) - h^{k} (z)| \, d z \\
 & = & \|f_{n}\|_{\infty}  \|h - h^{k}\|_{1} \to 0
\end{eqnarray*}
as $k \to + \infty$. Then we show that $h_{n}^{k}$ is continuous. This means we want to show that for $x \in [0, + \infty)$, let $x_{j} \to x$, then $h_{n}^{k} (x_{j}) \to h_{n}^{k} (x)$. By the definition of $h_{n}^{k} (x_{j})$, we have
\begin{equation*}
    h_{n}^{k} (x_{j}) = \int_{0}^{\infty} f_{n}(t) h^{k}(x_{j} + t) \, d t = \int_{0}^{\infty} e^{-y} h^{k} \Big{(} x_{j} + \sqrt{\frac{y}{n}} \Big{)} \, d y.
\end{equation*}
And since $h^{k} \in C_{c}([0, + \infty))$, $| e^{-y} h^{k} ( x_{j} + \sqrt{\frac{y}{n}} | \leq \|h^{k}\|_{\infty} e^{-y} \in L^{1}([0, + \infty))$, by the dominate convergence theorem, we have
\begin{equation*}
    \lim_{j \to \infty} h_{n}^{k} (x_{j}) = \int_{0}^{\infty} \lim_{j \to \infty} e^{-y} h^{k} \Big{(} x_{j} + \sqrt{\frac{y}{n}} \Big{)} \, d y = \int_{0}^{\infty} e^{-y} h^{k} \Big{(} x + \sqrt{\frac{y}{n}} \Big{)} \, d y = h_{n}^{k} (x),
\end{equation*}
thus we know that $h_{n}^{k}$ is uniformly continuous. From above, we have $h_{n}^{k} \to h_{n}$ almost everywhere and $h_{n}^{k}$ is uniformly continuous, then we have $h_{n}$ is the limit of a sequence of measurable functions. So, we get that $h_{n}$ is measurable on $[0, + \infty)$.

Next we show that $h_{n}$ is in $L^{1}([0, + \infty))$. Since
\begin{eqnarray*}
\|h_{n}\|_{1}  & = & \int_{0}^{\infty} |h_{n} (x)| \, d x  \\
 & = & \int_{0}^{\infty} \Big{|} \int_{0}^{\infty} f_{n} (t) h(x + t) \, d t  \Big{|} \, d x \\
 & \leq & \int_{0}^{\infty} \int_{0}^{\infty} | f_{n} (t) h(x + t) | \, d t \, d x, 
\end{eqnarray*}
by Fubini theorem, we have
\begin{eqnarray*}
\|h_{n}\|_{1}  & \leq & \int_{0}^{\infty} \int_{0}^{\infty} | f_{n} (t) h(x + t) | \, d x \, d t \\
& = & \int_{0}^{\infty} f_{n} (t) \Big{(} \int_{0}^{\infty}  | h(x + t) | \, d x \Big{)} \, d t \\
& = & \int_{0}^{\infty} f_{n} (t) \Big{(} \int_{t}^{\infty}  | h(z) | \, d z \Big{)} \, d t \\
& \leq & \int_{0}^{\infty} f_{n} (t) \Big{(} \int_{0}^{\infty}  | h(z) | \, d z \Big{)} \, d t \\
& = & \|h\|_{1} \int_{0}^{\infty} f_{n} (t) \, d t \\ 
& = & \|h\|_{1} < + \infty.
\end{eqnarray*}
Thus we know that $h_{n}$ is in $L^{1}([0, + \infty))$.

(iv) We want to show that $h_{n}$ converges to $h$ in $L^{1}([0, + \infty))$. Let $\epsilon > 0$, since $C_{c}([0, + \infty))$ is dense in $L^{1}([0, + \infty))$, then there exists a $g \in C_{c} ([0, + \infty))$ such that $\|h - g\|_{1} < \epsilon$. So we have
\begin{eqnarray*}
\|h_{n} - h \|_{1}  & = & \| h_{n} - g_{n} + g_{n} - g + g - f \|_{1} \\
& \leq & \|h_{n} - g_{n}\|_{1} + \|g_{n} - g\|_{1} + \| g - f \|_{1},
\end{eqnarray*}
where the definition of $g_{n}$ is as question (ii). By the result we get form (ii), for the $\epsilon$ above, we have $\|g_{n} - g\| < \epsilon$, then we know that
\begin{equation*}
    \|h_{n} - h \|_{1} < \|h_{n} - g_{n}\|_{1} + 2 \epsilon.
\end{equation*}
Next we need to deal with $\|h_{n} - g_{n}\|_{1}$. Since
\begin{eqnarray*}
\|h_{n} - g_{n}\|_{1} & = & \int_{0}^{\infty} |h_{n} (x)- g_{n} (x)| \, d x \\
& \leq & \int_{0}^{\infty} \int_{0}^{\infty} f_{n} (t) |h(x + t) - g(x + t)| \, d t \, d x,
\end{eqnarray*}
we denote $z = x + t$ and by Fubini theorem we have
\begin{eqnarray*}
\|h_{n} - g_{n}\|_{1}  & \leq & \int_{0}^{\infty} \int_{0}^{\infty} f_{n} (t) |h(x + t) - g(x + t)| \, d t \, d x \\
& = & \int_{0}^{\infty}  f_{n} (t) \int_{t}^{\infty} |h(z) - g(z)| \, d z \, d t \\
& \leq & \int_{0}^{\infty}  f_{n} (t) \int_{0}^{\infty} |h(z) - g(z)| \, d z \, d t \\
& = & \int_{0}^{\infty}  f_{n} (t) \|h - g\|_{1} d t \\
& = & \|h - g\|_{1} \int_{0}^{\infty} f_{n} (t) d t \\
& = & \|h - g\|_{1} < \epsilon.
\end{eqnarray*}
Thus we know that 
\begin{equation*}
    \|h_{n} - h\|_{1} < \|h_{n} - g_{n}\|_{1} + 2 \epsilon < 3 \epsilon
\end{equation*}
for any $\epsilon > 0$. So, we have showed that $h_{n}$ converges to $h$ in $L^{1}([0, + \infty))$.

\noindent\rule[0.25\baselineskip]{\textwidth}{0.5pt}

\vspace{8pt}

$\textbf{Exercise 4:}$

Show that a set $E \subset \mathbb{R}$ is Lebesgue measurable if and only if $E = H \cup Z$ where $H$ is a countable union of closed sets and $Z$ has measure zero. You may use the following property: for any Lebesgue measurable subset $A$ of $\mathbb{R}$ and any $\epsilon > 0$, there is a closed subset $F$ of $\mathbb{R}$ such that $F \subset A$ and the measure of $A \setminus F$ is less than $\epsilon$.

\vspace{8pt}
$\textbf{Solution:}$

If $E \subset \mathbb{R}$ is Lebesgue measurable, then we know that $\forall \epsilon > 0$, there is a closed subset $H$ of $\mathbb{R}$ such that $H \subset E$ and the measure of $E \setminus H$ is less than $\epsilon$. We denote $Z = E \setminus H$, then we have $m(Z) = 0$ and $Z \cup H = (E \setminus H) \cup H = E$. 

Since $H$ is a countable union of closed sets, then $H$ is a $\mathcal{F}_{\sigma}$ set and it is measurable. And as $Z$ is a zero measure set, it is also Lebesgue measurable. Thus we know that $E = H \cup Z$ is Lebesgue measurable.

\noindent\rule[0.25\baselineskip]{\textwidth}{0.5pt}

$\textbf{Exercise 5:}$

Give an example of a sequence $f_{n}$ in $L^{1} ((0, 1))$ such that $f_{n} \to 0$ in $L^{1}((0, 1))$ but $f_{n}$ does not converge to zero almost everywhere.

\vspace{8pt}
$\textbf{Solution:}$

We suppose that
\begin{equation*}
    f_{n} (x) = \mathbb{I}_{[\frac{n - 2^{k}}{2^{k}}, \frac{n - 2^{k} + 1}{2^{k}}]} (x),
\end{equation*}
whenever $k \geq 0, 2^{k} \leq n < 2^{k + 1}$. For any $n \in \mathbb{N}$, we have
\begin{equation*}
    \int_{0}^{1} | f_{n} (x) | \, d x = \int_{0}^{1} \mathbb{I}_{[\frac{n - 2^{k}}{2^{k}}, \frac{n - 2^{k} + 1}{2^{k}}]} (x) \, d x  = \frac{1}{2^{k}} < +\infty,
\end{equation*}
so we know that $f_{n} \in L^{1}((0, 1))$. And similarly we have
\begin{equation*}
    \int_{0}^{1} | f_{n} (x) - 0 | \, d x = \int_{0}^{1} \mathbb{I}_{[\frac{n - 2^{k}}{2^{k}}, \frac{n - 2^{k} + 1}{2^{k}}]} (x) \, d x  = \frac{1}{2^{k}} < \frac{2}{n},
\end{equation*}
then when $n \to + \infty$, $\int_{0}^{1} | f_{n} (x) - 0 | \, d x \to 0$, thus we get $f_{n} \to 0$ in $L^{1}((0, 1))$. But for any $x \in (0, 1)$, and for any $N \in \mathbb{N}$, we can find a $n > N$ with $f_{n} (x) = 1$. Thus $f_{n}$ can not converges to $0$ anywhere for $x \in (0, 1)$.

\newpage

\section{GCE January, 2017}

$\textbf{Exercise 1:}$

Consider the sequence of functions $f_{n}$ defined on the non-negative reals by $f_{n} (x) = 2 n x P(x) e^{-n x^{2}}$, where $P$ is a polynomial function.

(i) Is $f_{n}$ pointwise convergent on $[0, + \infty)$? Is $f_{n}$ uniformly convergent on $[0, + \infty)$? Explain your answers to both questions.

(ii) Let $g_{n}$ be a sequence of continuous functions defined on $[0, + \infty)$ and valued in $\mathbb{R}$. Assume that each $g_{n}$ is in $L^{1}([0, + \infty))$ and that sequence $g_{n}$ is uniformly convergent to zero. Prove or disprove: $\lim_{n \to \infty} \int_{0}^{\infty} g_{n} = 0$. 

(iii) Determine (with proof) $\lim_{n \to \infty} \int_{0}^{\infty}  f_{n}$.
  
\vspace{8pt}

$\textbf{Solution:}$

(i) For any $x \in [0, \infty)$, as $P(x)$ is a polynomial function, by the L'Hospital's Rule, we have
\begin{equation*}
    \lim_{n \to \infty} f_{n} (x) = \lim_{n \to \infty} 2 n x P(x) e^{-n x^{2}} = \lim_{n \to \infty} \frac{2 n x P(x)}{e^{n x^{2}}} = 0,
\end{equation*}
then for any $\epsilon > 0$, there exist a $N \in \mathbb{N}$, such that $n > N$ we have 
\begin{equation*}
    |2 n x P(x) e^{- n x^{2}} - 0| < \epsilon,
\end{equation*}
thus we know that $f_{n}$ converges to $f(x) = 0$ pointwise on $[0, \infty)$. But $f_{n}$ is not uniformly convergent to $f(x) = 0$. We suppose $P(x) = 1$, then we have $f_{n}(x) = 2 n x e^{-n x^{2}} $. When $x = \frac{1}{\sqrt{n}}$, 
\begin{equation*}
    f_{n} (x) = 2 n \frac{1}{\sqrt{n}} e^{- n \frac{1}{n}} = 2 \sqrt{n} e^{-1},
\end{equation*}
so we have
\begin{equation*}
    \sup_{x \in [0, \infty)} |f_{n} (x) - 0| \geq 2 \sqrt{n} e^{-1} \to \infty
\end{equation*}
when $n$ goes to $+ \infty$. Thus we know that $f_{n}$ is not uniformly converges on $[0, + \infty)$. 

(ii) The statement is not true. We suppose 
\begin{equation*}
g_{n}(x) =
\left\{
             \begin{array}{cl}
             \frac{4}{n^{2}} x, & x \in [0, \frac{n}{2}) \\
             \frac{4}{n} - \frac{4}{n^{2}} x, & x \in [\frac{n}{2}, n]  \\
             0, & x \in (n, + \infty),
             \end{array}
\right.
\end{equation*}
then we know that for any $n \in \mathbb{N}$,
\begin{equation*}
    \int_{[0, \infty)}^{} g_{n} (x) \, d x = \int_{0}^{\frac{n}{2}} \frac{4}{n^{2}} x \, d x + \int_{\frac{n}{2}}^{n} \frac{4}{n} - \frac{4}{n^{2}} x \, d x = 1,
\end{equation*}
so we know that $g_{n} (x) \in L^{1}([0, \infty))$. When $x \in [0, \frac{n}{2})$, $g_{n}(x) = \frac{4}{n^{2}} x \leq \frac{2}{n}$ and when $x \in [\frac{n}{2}, n]$, $g_{n} (x) = \frac{4}{n} - \frac{4}{n^{2}} x \leq \frac{2}{n}$, so we know that $g_{n}$ uniformly converges to $0$. But since for any $n \in \mathbb{N}$, $\int_{0}^{\infty} g_{n}(x) \, d x  = 1$, then we have
\begin{equation*}
    \lim_{n \to + \infty} \int_{[0, \infty)}^{} g_{n} (x) \, d x = \lim_{n \to \infty} 1 = 1.
\end{equation*}
Thus $\lim_{n \to \infty} \int_{0}^{\infty} g_{n} = 0$ can not hold.

(iii) We denote $y = n x^{2}$, then we have
\begin{equation*}
    \int_{0}^{\infty} 2 n x P(x) e^{- n x^{2}} \, d x = \int_{0}^{\infty} e^{-y} P \Big{(} \sqrt{\frac{y}{n}} \Big{)} \, d y.
\end{equation*}
Since $P(x)$ is a polynomial function, for any fixed $y$, when $n \to \infty$, $P(\sqrt{\frac{y}{n}}) \to P(0)$ and then $e^{-y} P(\sqrt{\frac{y}{n}}) \to e^{-y} P(0)$. Since $P(x)$ is a polynomial function, there exist a $M > 0$, such that when $y \in [M, \infty)$, $e^{-y} P(\sqrt{\frac{y}{n}}) < \frac{1}{y^{2}}$, then we have
\begin{eqnarray*}
\lim_{n \to \infty} \int_{0}^{\infty} f_{n} & = & \lim_{n \to \infty} \int_{0}^{M} f_{n} + \lim_{n \to \infty} \int_{M}^{\infty} f_{n} \\
& = & \lim_{n \to \infty} \int_{0}^{M} e^{-y} P \Big{(} \sqrt{\frac{y}{n}} \Big{)} \, d y + \lim_{n \to \infty} \int_{M}^{\infty} e^{-y} P \Big{(} \sqrt{\frac{y}{n}} \Big{)} \, d y .
\end{eqnarray*}
Since $P(x)$ is a polynomial function, then $P(\sqrt{\frac{y}{n}})$ is continuous on $y \in [0,M]$, then we have when $y \in [0, M]$,
\begin{equation*}
    \Big{|} e^{-y} P \Big{(} \sqrt{\frac{y}{n}} \Big{)} \Big{|} \leq e^{-y} \|P\|_{\infty}.
\end{equation*}
Since $e^{-y} \|P\|_{\infty} \in L^{1}([0, M])$ and $\frac{1}{y^{2}} \in L^{1}([M, + \infty))$, by the dominate convergence theorem, we have
\begin{equation*}
    \lim_{n \to \infty} \int_{0}^{M} e^{-y} P \Big{(} \sqrt{\frac{y}{n}} \Big{)} \, d y = \int_{0}^{M} e^{-y} P(0) \, d y = P(0) (1 - e^{- M}),
\end{equation*}
and
\begin{equation*}
    \lim_{n \to \infty} \int_{M}^{\infty} e^{-y} P \Big{(} \sqrt{\frac{y}{n}} \Big{)} \, d y = \int_{M}^{\infty} e^{-y} P(0) \, d y = P(0) e^{- M}.
\end{equation*}
Thus we know that
\begin{eqnarray*}
\lim_{n \to \infty} \int_{0}^{\infty} f_{n} & = &  \lim_{n \to \infty} \int_{0}^{M} e^{-y} P \Big{(} \sqrt{\frac{y}{n}} \Big{)} \, d y + \lim_{n \to \infty} \int_{M}^{\infty} e^{-y} P \Big{(} \sqrt{\frac{y}{n}} \Big{)} \, d y \\
& = & P(0) (1 - e^{- M}) + P(0) e^{- M} \\
& = & P(0).
\end{eqnarray*}

\newpage


$\textbf{Exercise 2}  (\text{all answers require proofs: })$

Let $f_{n}$ be the sequence in $L^{2}(\mathbb{R})$ defined by $f_{n} = \mathbb{I}_{[n, n+1]}$.

(i) Let $g$ be in $L^{2}(\mathbb{R})$. Does $\int_{}^{} f_{n} g$ have a limit as $n$ tends to infinity?

(ii) Does the sequence $f_{n}$ converge in $L^{2}(\mathbb{R})$?

\vspace{8pt}
$\textbf{Solution:}$

(i) Firstly we show that $f_{n} = \mathbb{I}_{[n, n+1]} (x)$ converges to $f(x) = 0$ pointwise on $\mathbb{R}$. Since 
\begin{equation*}
    |f_{n} - f| = |\mathbb{I}_{[n, n+1]} (x) - 0| = \mathbb{I}_{[n, n+1]} (x), 
\end{equation*}
for any fixed $x \in \mathbb{R}$, $\forall \epsilon > 0$,we can find a $N = [x] + 1$, such that $n > N$, we have 
\begin{equation*}
    |f_{n} - f| = \mathbb{I}_{[n, n+1]} (x) = 0 < \epsilon.
\end{equation*}
Thus we know that $f_{n}$ converges to $f(x) = 0$ pointwisely on $\mathbb{R}$. Since  
\begin{equation*}
    \Big{|} \int_{\mathbb{R}}^{} f_{n} g \, d x \Big{|} \leq \int_{\mathbb{R}}^{} | f_{n} g | \, d x = \int_{n}^{n + 1} | g (x) | \, d x,
\end{equation*}
by Cauchy-Schwarz inequality, we have
\begin{eqnarray*}
\Big{|} \int_{\mathbb{R}}^{} f_{n} g \, d x \Big{|} & \leq & \int_{n}^{n + 1} | g (x) | \, d x \\
& \leq & \Big{(} \int_{n}^{n+1} |g|^{2} \, d x \Big{)}^{\frac{1}{2}} \Big{(} \int_{n}^{n+1} 1^{2} \, d x \Big{)}^{\frac{1}{2}} \\
& = & \Big{(} \int_{n}^{n+1} |g|^{2} \, d x \Big{)}^{\frac{1}{2}} \\ 
& = & \Big{(} \int_{\mathbb{R}}^{} |g|^{2} \mathbb{I}_{[n, n+1]} (x) \, d x \Big{)}^{\frac{1}{2}}.
\end{eqnarray*}
Since $|g|^{2} \mathbb{I}_{[n, n+1]} (x) \leq |g(x)|^{2}$ and since $g \in L^{2}(\mathbb{R})$, we have $\int_{\mathbb{R}}^{} |g(x)|^{2} \, d x < + \infty$, then we know that $|g(x)|^{2} \in L^{1}(\mathbb{R})$, by the dominate convergence theorem, we have
\begin{eqnarray*}
\lim_{n \to \infty} \Big{|} \int_{\mathbb{R}}^{} f_{n} g \, d x \Big{|}^{2} & \leq & \lim_{n \to \infty} \Big{(} \int_{\mathbb{R}}^{} |g|^{2} \mathbb{I}_{[n, n+1]} (x) \, d x \Big{)} \\
& = & \int_{\mathbb{R}}^{} \lim_{n \to \infty} \big{(} |g|^{2} \mathbb{I}_{[n, n+1]} (x)  \big{)} \, d x.
\end{eqnarray*}
Since $f_{n} = \mathbb{I}_{[n, n+1]} (x)$ converges to $f(x) = 0$ pointwise on $\mathbb{R}$, we can show that $|g|^{2} \mathbb{I}_{[n, n+1]} (x)$ also converges to $f(x) = 0$ pointwise on $\mathbb{R}$, then we have
\begin{equation*}
    \lim_{n \to \infty} \Big{|} \int_{\mathbb{R}}^{} f_{n} g \, d x \Big{|}^{2} \leq \int_{\mathbb{R}}^{} \lim_{n \to \infty} \big{(} |g|^{2} \mathbb{I}_{[n, n+1]} (x)  \big{)} \, d x = 0.
\end{equation*}
Thus we know that $\lim_{n \to \infty} \int_{\mathbb{R}}^{} f_{n}(x) g(x) \, d x = 0$.

(ii) Since $f_{n} = \mathbb{I}_{[n, n+1]} (x)$ converges to $f(x) = 0$ pointwise on $\mathbb{R}$, but 
\begin{equation*}
    \int_{\mathbb{R}}^{} |f_{n}(x) - 0|^{2} \, d x = \int_{\mathbb{R}}^{} f_{n}^{2} \, d x = \int_{n}^{n + 1} 1 \, d x = 1,
\end{equation*}
we know that $f_{n}$ does not converges to $f(x) = 0$ in $L^{2}(\mathbb{R})$, then we can get that the sequence $f_{n}$ does not converge in $L^{2}(\mathbb{R})$.

\vspace{8pt}

\noindent\rule[0.25\baselineskip]{\textwidth}{0.5pt}

$\textbf{Exercise 3:}$

Let $X$ be a matrix space. For any subset $A$ of $X$, we denote by $\bar{A}$ the closure of $A$ and $\mathring{A}$ the union of all open subsets contained in $A$. We set $\partial A = \bar{A} \setminus \mathring{A} $.

(i) Show that $A$ is closed if and only if $\partial A \subset A$.

(ii) Show that $A$ is open if and only if $\partial A \cap A = \emptyset$.

(iii) Is the identity $\partial (\partial B) = \partial B$ valid for all subsets $B$ of $X$?
  
(iv) Show that if $A$ is closed then $\partial(\partial A) = \partial A$.

\vspace{8pt}
$\textbf{Solution:}$

(i) When $A$ is closed, we have $A = \bar{A}$, since $\mathring{A}$ the union of all open subsets contained in $A$, then $\mathring{A} \subset A$. Thus we have $\partial A = \bar{A} \setminus \mathring{A} = A \setminus \mathring{A}  \subset A$ as $\mathring{A} \subset A$.

When $\partial A \subset A$, we have $\partial A \cup A \subset A \cup A = A$, then we know that $\bar{A} \subset A$. Since $A \subset \bar{A}$, we can get $\bar{A} = A$, thus $A$ is closed.

(ii) When $A$ is open, since $\mathring{A}$ the union of all open subsets contained in $A$, then we have $A \subset \mathring {A}$. And we know that $\mathring{A} \subset A$, then we can get $A = \mathring{A}$. As $\partial A = \bar{A} \setminus \mathring{A} $, we have $\partial A = \bar{A} \setminus A $, then it is obviously that $\partial A \cap  A  = \emptyset$.

When $\partial A \cap A = \emptyset$, we suppose $A$ is not an open set, then there exists a element $x \in A$ such that no open set containing $x$ is a subset of $A$. Since $\mathring{A}$ the union of all open subsets contained in $A$, we have $x \notin \mathring{A}$. And as $x \in A$, we know that $x \in \bar{A}$, then we have $x \in \bar{A} \setminus \mathring{A} = \partial A$. Then we can get $x \in  \partial A \cap A $, it is contradict with the condition we have. So, the statement that $A$ is not an open set is wrong. Thus we have $A$ is an open set.

(iii) No, the statement is not true. We suppose $B = \mathbb{Q} \cap [0, 1]$, which represents the rational number in the interval $[0, 1]$. Then we have $\partial B = [0, 1]$ and $\partial (\partial B) = \{0, 1\}$, which is not equal to $\partial B$.

(iv) Since $\bar{A}$ is closed and $\mathring{A}$ is open, we have $\partial A = \bar{A} \setminus \mathring{A} $ is closed, then we can get $\overline{\partial A} = \partial A$. By the definition of $\partial A$, we have $\partial (\partial A) = \overline{\partial A} \setminus \mathring{\partial A} = \partial A \setminus \mathring{\partial A} \subset \partial A$. Next we need to show that $\partial A \subset \partial (\partial A) = \partial A \setminus \mathring{\partial A}$, then we just need to prove that $\mathring{\partial A} = \emptyset$ when $A$ is closed. 

When $A$ is closed, since $\partial A = \bar{A} \setminus \mathring{A} = A \setminus \mathring{A}$. As $A \setminus \mathring{A} \subset A$, then we have $\mathring{\partial A} \subset \mathring{A}$. And since the union of subsets in $(A \setminus \mathring{A})$ is the subset of $ A \setminus \mathring{A}$, we have $\mathring{\partial A} \subset A \setminus \mathring{A}$. Then we know that $\mathring{\partial A} \subset \mathring{A}$ and $\mathring{\partial A} \subset A \setminus \mathring{A}$. Thus we can get $\mathring{\partial A} \subset \mathring{A} \cap (A \setminus \mathring{A}) = \emptyset$. So, we have showed that $\mathring{\partial A} = \emptyset$. In conclusion, we have $\partial(\partial A) = \partial A$ when $A$ is closed.

\vspace{8pt}

\noindent\rule[0.25\baselineskip]{\textwidth}{0.5pt}

$\textbf{Exercise 4:}$

Let $X$ be a measure space, $f_{n}$ a sequence in $L^{1} (X)$ and $f$ an element of $L^{1}(X)$ such that $f_{n}$ converges to $f$ almost everywhere and $\lim_{n \to \infty} \int_{}^{} |f_{n}| = \int_{}^{} |f|$. Show that $\lim_{n \to \infty} \int_{}^{} |f_{n} - f| = 0$.

\vspace{8pt}
$\textbf{Solution:}$

Since $|f_{n} - f| \leq |f_{n}| + |f|$ holds on $X$, we know that $|f_{n}| + |f| - |f_{n} - f|$ is a non-negative function. By the Fatou's lemma, we have
\begin{equation*}
    \int_{}^{} \lim_{n \to \infty} (|f_{n}| + |f| - |f_{n} - f|)  \leq \liminf_{n \to \infty} \int_{}^{} (|f_{n}| + |f| - |f_{n} - f|).
\end{equation*}
Since $f_{n}$ converges to $f$ almost everywhere, then we know that $|f_{n}|$ converges to $|f|$ almost everywhere. Thus we have
\begin{equation*}
    \lim_{n \to \infty} (|f_{n}| + |f| - |f_{n} - f|) = 2 |f|.
\end{equation*}
Then we can get that
\begin{eqnarray*}
\int_{}^{} 2 |f| & \leq & \liminf_{n \to \infty} \int_{}^{} (|f_{n}| + |f| - |f_{n} - f|) \\
& \leq & \liminf_{n \to \infty} \int_{}^{} (|f_{n}| + |f|) - \limsup_{n \to \infty}  \int_{}^{} (|f_{n} - f|) \\
& = & \int_{}^{} 2 |f| - \limsup_{n \to \infty}  \int_{}^{} (|f_{n} - f|),
\end{eqnarray*}
as $f \in L^{1}(X)$, then $\int_{}^{} |f| < + \infty$, we have
\begin{equation*}
    \limsup_{n \to \infty}  \int_{}^{} (|f_{n} - f|) \leq 0.
\end{equation*}
On the other hand, we have
\begin{equation*}
    0 \leq \liminf_{n \to \infty}  \int_{}^{} (|f_{n} - f|)
\end{equation*}
as $|f_{n} - f| \geq 0$. Thus we know that
\begin{equation*}
    \limsup_{n \to \infty}  \int_{}^{} (|f_{n} - f|) \leq 0 \leq \liminf_{n \to \infty}  \int_{}^{} (|f_{n} - f|),
\end{equation*}
which is equivalent to
\begin{equation*}
    \limsup_{n \to \infty}  \int_{}^{} (|f_{n} - f|) = \liminf_{n \to \infty}  \int_{}^{} (|f_{n} - f|) = 0.
\end{equation*}
So we have
\begin{equation*}
    \lim_{n \to \infty} \int_{}^{} |f_{n} - f| = 0.
\end{equation*}



\newpage

\section{GCE May, 2017}

$\textbf{Exercise 1:}$

Let $(X, \mathcal{A}, \mu)$ be a measure space. Let $A_{n}$ be a sequence in $\mathcal{A}$ such that $\mu(A_{n})$ converges to zero.

(i) Prove or disprove: if $f : X \rightarrow [0, + \infty)$ is a measurable function and $\mu(X) < + \infty$, then $\int_{A_{n}}^{} f$ converges to zero.

(ii) Let $g$ be in $L^{1}(X)$. Show that $\int_{A_{n}}^{} g$ converges to zero.

\vspace{8pt}

$\textbf{Solution:}$

(i) The statement is not true. We suppose $X = (0, 1]$ and $f(x) = \frac{1}{x^{2}}$, then we know that $\mu(X) < + \infty$ and $f(x)$ is measurable on $X$. We set $A_{n} = [\frac{1}{n^{2}}, \frac{1}{n}], n \in \mathbb{N}$. Thus we have for all $n \in \mathbb{N}$, $A_{n} \subset X$. And
\begin{equation*}
   \mu(A_{n}) = \frac{1}{n} - \frac{1}{n^{2}} = \frac{n - 1}{n^{2}} \to 0
\end{equation*}
as $n$ goes to infinity. But for the $\int_{A_{n}}^{} f$, we have
\begin{equation*}
   \int_{A_{n}}^{} f \, d \mu = \int_{\frac{1}{n^{2}}}^{\frac{1}{n}} \frac{1}{x^{2}} \, d x = n^{2} - n \to + \infty
\end{equation*}
as $n \to + \infty$. So, we know that $\int_{A_{n}}^{} f$ does not converges to zero.

\vspace{8pt}

(ii) We denote
\begin{equation*}
   g_{n}(x) = g(x) \mathbb{I}_{A_{n}} (x),
\end{equation*}
where $\mathbb{I}_{A_{n}} (\cdot)$ is a indicator function on $A_{n}$. Since $A_{n}$ is a sequence in $\mathcal{A}$ such that $\mu(A_{n}) \to 0$ as $n \to + \infty$, then we know that $g_{n}(x)$ converges to $0$ almost everywhere. As
\begin{equation*}
   |g_{n}(x)| = |g(x) \mathbb{I}_{A_{n}} (x)| \leq |g(x)|
\end{equation*}
and $g \in L^{1}(X)$, we know that $g$ is a dominate function of $g_{n}$. By the dominate convergence theorem, we have
\begin{equation*}
   \lim_{n \to \infty} \int_{X}^{} g_{n}(x) \, d \mu = \int_{X}^{} 0 \, d \mu = 0,
\end{equation*}
thus we have
\begin{equation*}
   \lim_{n \to \infty} \int_{X}^{} g_{n}(x) \, d \mu = \lim_{n \to \infty} \int_{A_{n}}^{} g \, d \mu = 0.
\end{equation*}
So, we know that $\int_{A_{n}}^{} g$ converges to zero.

\newpage


$\textbf{Exercise 2:}$

Let $(x, d)$ be a bounded metric space. For any non empty subset $S$ of $X$ and $x$ in $X$ we define:
\begin{equation*}
   d(x, S) = \inf \{d(x, s): s \in S\}.
\end{equation*}
If $A$ and $B$ are two non empty subsets of $X$ we define:
\begin{equation*}
   d_{H}(A, B) = \max \{\sup_{x \in A} d(x, B), \sup_{x \in B} d(x, A) \}.
\end{equation*}

(i) Prove or disprove: If $d_{H}(A, B) = 0$, are $A$ and $B$ necessarily equal?

(ii) Let $\mathcal{C}$ be the set of all non empty closed subsets of $X$. Show that $d_{H}$ defines a metric on $\mathcal{C}$.
  
\vspace{8pt}
$\textbf{Solution:}$

(i) The statement is not true. By the definition of $d_{H}(A, B)$, since $d_{H}(A, B) = 0$, we have
$$\max \{\sup_{x \in A} d(x, B), \sup_{x \in B} d(x, A) \} = 0,$$
then we have $\sup_{x \in A} d(x, B) =  \sup_{x \in B} d(x, A) = 0$, so we know that $\forall x \in A, d(x, B) = 0$ and $\forall x \in B, d(x, A) = 0$. For any $x \in A$
, since $d(x, B) = \inf \{d(x, y): y \in B \} = 0$, we can find a sequence $\{y_{n}\}$, and for any $x \in A$ this sequence converges to $x$. So we have $B \subset \bar{A}$, where $\bar{A}$ is the closure of $A$. Similarly, we have $A \subset \bar{B}$.

We suppose $A = [0, 1)$ and $B = [0, 1]$, thus $A \neq B$. Since $A \subset B$, $\forall x \in A$, $\exists y \in B$ such that $x = y$ and $d(x, y) = 0$, we have $\sup_{x \in A} d(x, B) = 0$. On the other hand, when $x \in B$ and $x \in [0, 1)$, since $A = [0, 1)$, we know that foe any $x \in [0, 1)$, there exists a $y \in A$ such that $x = y$ and then $d(x, y) = 0$. And when $x \in B$ and $x = \{1\}$, since $y \in A = [0, 1)$, we have $d(x, A) = \inf \{d(x, y): y \in A \} = 0$. Thus it is also holds that $\sup_{x \in B} d(x, A) = 0$. Then we know that $d_{H}(A, B) = 0$ but $A \neq B$. So, $A$ and $B$ is not necessarily equal.

\vspace{8pt}

(ii) Since $\mathcal{C}$ is the set of all non empty closed subsets of $X$, for $A \in \mathcal{C}$ and $B \in \mathcal{C}$, $A, B$ are both closed sets. Next we need to verify the definition of the metric.

(a) $d_{H}(A, B) \geq 0$: since $(X, d)$ is a metric space, then $d(x, B) \geq 0$ and $d(x, A) \geq 0$, thus we have $d_{H}(A, B) = \max \{\sup_{x \in A} d(x, B), \sup_{x \in B} d(x, A) \} \geq 0$.

(b) $d_{H}(A, B) = 0 \iff A = B$: if A = B, then we have $d(x, B) = 0$ for any $x \in A$ and $d(x, A) = 0$ for any $x \in B$, thus we know that $d_{H}(A, B) = 0$. If $d_{H}(A, B) = 0$, by the result we get from (i), we know that $A \subset \bar{B}$ and $B \subset \bar{A}$. Since $A$ and $B$ are both closed sets, then we have $A \subset B$ and $B \subset A$, thus we can get $A = B$.

(c) $d_{H}(A, B) = d_{H}(B, A)$: since $d_{H}(A, B) = \max \{\sup_{x \in A} d(x, B), \sup_{x \in B} d(x, A) \}$ and $d_{H}(B, A) = \max \{\sup_{x \in B} d(x, A), \sup_{x \in A} d(x, B) \}$, thus we have $d_{H}(A, B) = d_{H}(B, A)$.

(d) For $A, B, C \in \mathcal{C}$, $d_{H}(A, B) \leq d_{H}(A, C) + d_{H}(C, B)$: since $d_{H}(A, C) + d_{H}(C, B) \geq $ $\sup_{x \in A} d(x, C) + \sup_{x \in C} d(x, B)$, then we know that $d_{H}(A, C) + d_{H}(C, B) \geq \sup_{x \in A} d(x, B)$. Similarly, we have $d_{H}(A, C) + d_{H}(C, B) \geq \sup_{x \in B} d(x, A)$, thus we can get $d_{H}(A, C) + d_{H}(C, B) \geq \max \{\sup_{x \in A} d(x, B), \sup_{x \in B} d(x, A) \} = d_{H}(A, B)$.

\vspace{8pt}

\noindent\rule[0.25\baselineskip]{\textwidth}{0.5pt}

$\textbf{Exercise 3:}$

Let $(X, \mathcal{A}, \mu)$ be a measure space and $\{f_{k}\}$ a sequence in $L^{p}(X)$ where $1 \leq p \leq + \infty$. Suppose that $\{f_{k}\}$ converges in $L^{p}(X)$ to $f$. Show that $f_{k}$ converges in measure to $f$ on $X$.

$\textbf{Hint: }$ According to the definition f convergence in measure, you need to show that for any positive $\epsilon$, $\mu(\{x \in X: |f_{k}(x) - f(x)| \geq \epsilon \})$ converges to zero as $k$ tends to infinity.

\vspace{8pt}
$\textbf{Solution:}$

When $p = + \infty$, since the sequence $\{f_{k}\}$ converges to $f$ in $L^{\infty} (X)$, then $\forall \epsilon > 0$, $\exists N \in \mathbb{N}$, when $n > N$, we have $\|f_{n} - f\|_{\infty} < \epsilon$. It means that $|f_{n} - f|$ is less than $\epsilon$ almost everywhere. Thus we have $\mu(|f_{n} - f| > \epsilon) = 0$ when $n \to \infty$. So we get that $\mu(\{x \in X: |f_{n}(x) - f(x)| \geq \epsilon \})$ converges to zero as $n$ tends to infinity.

When $1 \leq p < \infty$, for any $\epsilon > 0$, we have
\begin{eqnarray*}
\|f_{n} - f\|_{p}^{p} & = & \int_{X}^{} |f_{n} - f|^{p} \, d \mu \\
& \geq & \int_{\{x \in X: |f_{n} - f|^{p} \geq \epsilon^{p}\}}^{} |f_{n} - f|^{p}  \, d \mu \\
& \geq & \epsilon^{p} \mu(\{x \in X: |f_{n} - f|^{p} \geq \epsilon^{p} \})  \\
& = & \epsilon^{p} \mu(\{x \in X: |f_{n} - f| \geq \epsilon \}),
\end{eqnarray*}
so we know that
\begin{equation*}
   \mu(\{x \in X: |f_{n} - f| \geq \epsilon \}) \leq \frac{1}{\epsilon^{p}} \| f_{n} - f \|_{p}^{p}.
\end{equation*}
Since $\{f_{n}\}$ converges in $L^{p}(X)$ to $f$, we have $\|f_{n} - f \|_{p}^{p} \to 0$ as $n \to \infty$. So, for any $\epsilon > 0$, $\mu(\{x \in X: |f_{n}(x) - f(x)| \geq \epsilon \})$ converges to zero as $n$ tends to infinity.


\vspace{8pt}

\noindent\rule[0.25\baselineskip]{\textwidth}{0.5pt}

$\textbf{Exercise 4:}$

Suppose $g_{n}, g \in L^{1}(\mathbb{R})$, $g_{n}$ converges to $g$ almost everywhere, and $\int_{}^{} g_{n} $ converges to $\int_{}^{} g$. Define $f_{n}(x) := g_{n}(x + n)$.

(i) Prove or disprove: there exists an $f$ in $L^{1}(\mathbb{R})$ such that $f_{n}$ converges to $f$ almost everywhere.

(ii) Prove or disprove: if there is an $f$ as in (i), then $\int_{}^{} f_{n}$ converges to $\int_{}^{} f$.

\vspace{8pt}
$\textbf{Solution:}$

(i) The statement is not true. We suppose $g_{n}(x) = (x + \frac{1}{n}) \mathbb{I}_{[0, 1]} (x)$ and $g(x) = x \mathbb{I}_{[0, 1]} (x)$, then we have
\begin{equation*}
   |g_{n} (x) - g(x) | = | (x + \frac{1}{n}) \mathbb{I}_{[0, 1]} (x) - x \mathbb{I}_{[0, 1]} (x) | = \frac{1}{n} \to 0
\end{equation*}
when $n$ tends to infinity. So, $g_{n}$ converges to $g$ almost everywhere. Since 
\begin{equation*}
   \int_{\mathbb{R}}^{} g_{n}(x) \, d x = \int_{0}^{1} (x + \frac{1}{n}) \, d x = \frac{1}{2} + \frac{1}{n} \to \frac{1}{2} 
\end{equation*}
as $n \to + \infty$ and
\begin{equation*}
   \int_{\mathbb{R}}^{} g (x) \, d x = \int_{0}^{1} x \, d x = \frac{1}{2},
\end{equation*}
we know that $\int_{}^{} g_{n} $ converges to $\int_{}^{} g$. As $f_{n}(x) := g_{n}(x + n)$, then $f_{n} (x) = (x + n + \frac{1}{n}) \mathbb{I}_{[0, 1]} (x)$, it is diverges as $f_{n} (x) > n$ for any $x \in [0, 1]$.

(ii) The statement is not true. We set $g_{n} (x) = \frac{1}{\sqrt{2 \pi}} e^{- \frac{x^{2}}{2}}$ for all $n \in \mathbb{N}$ and $g (x) = \frac{1}{\sqrt{2 \pi}} e^{- \frac{x^{2}}{2}}$ too. Since $\int_{- \infty}^{+ \infty} \frac{1}{\sqrt{2 \pi}} e^{- \frac{x^{2}}{2}} \, d x = 1$, then we have $g_{n} (x) \in L^{1}(\mathbb{R})$ and $g(x) \in L^{1}(\mathbb{R})$. As $g_{n} (x) = g(x)$, we know that $g_{n}$ converges to $g$ almost everywhere. By the definition of $f_{n}(x)$, we know that $f_{n}(x) = g_{n} (x + n) = \frac{1}{\sqrt{2 \pi}} e^{- \frac{(x+n)^{2}}{2}}$ and when $f(x) = 0$, for any fix $x \in \mathbb{R}$ we have,
\begin{equation*}
   |f_{n}(x) - f(x)| = |\frac{1}{\sqrt{2 \pi}} e^{- \frac{(x+n)^{2}}{2}} - 0| \to 0 
\end{equation*}
as $n \to + \infty$. So, we know that $f$ is in $L^{1}(\mathbb{R})$ and $f_{n}$ converges to $f$ almost everywhere. But for any $n \in \mathbb{N}$, we have
\begin{equation*}
   \int_{\mathbb{R}}^{} f_{n} (x) \, d x = \int_{\mathbb{R}}^{} \frac{1}{\sqrt{2 \pi}} e^{- \frac{(x+n)^{2}}{2}} \, d x = 1,
\end{equation*}
and we know that
\begin{equation*}
   \int_{\mathbb{R}}^{} f (x) \, d x = \int_{\mathbb{R}}^{} 0 \, d x = 0,
\end{equation*}
thus $\int_{}^{} f_{n}$ does not converges to $\int_{}^{} f$. 

\newpage

\section{GCE August, 2017}

$\textbf{Exercise 1:}$

Let $h_{n}$ be a sequence of non-negative, borel measurable functions on the interval $(0, 1)$ such that $h_{n} \rightarrow 0$ in $L^{1}((0, 1))$.

(i) Show $\sqrt{h_{n}} \rightarrow 0$ in $L^{1}((0, 1))$.

(ii) Given an example to show that $h_{n}^{2}$ need not converge to zero in $L^{1}((0, 1))$.

(iii) If $g_{n}$ is in $L^{1}(\mathbb{R})$ such that $|g_{n}|^{\frac{1}{2}}$ is in $L^{1}(\mathbb{R})$, and $g_{n}$ converges to zero in $L^{1}(\mathbb{R})$ as $n$ tends to infinity, does $|g_{n}|^{\frac{1}{2}}$ converges to zero in $L^{1}(\mathbb{R})$?

\vspace{8pt}

$\textbf{Solution:}$

(i) We want to show that $\int_{0}^{1} |\sqrt{h_{n}} - 0| \, d \mu \rightarrow 0$ as $n \rightarrow \infty$. Since $h_{n} \rightarrow 0$ in $L^{1}((0, 1))$ and by the holder inequality, we have
\begin{eqnarray*}
\int_{0}^{1} |\sqrt{h_{n}} - 0| \, d \mu & \leq & \Big{(} \int_{0}^{1} |(\sqrt{h_{n}})^{2}| \, d \mu \Big{)}^{\frac{1}{2}} \Big{(} \int_{0}^{1} 1^{2} \, d \mu \Big{)}^{\frac{1}{2}} \\
& = & \Big{(} \int_{0}^{1} h_{n} \, d \mu \Big{)}^{\frac{1}{2}} \Big{(} \int_{0}^{1} 1 \, d \mu \Big{)}^{\frac{1}{2}} \\
& = & \Big{(} \int_{0}^{1} |h_{n}-0| \, d \mu \Big{)}^{\frac{1}{2}}.
\end{eqnarray*}
So when $n$ goes to infinity, we have $\int_{0}^{1} |\sqrt{h_{n}} - 0| \, d \mu \rightarrow 0$. Thus we know that $\sqrt{h_{n}} \rightarrow 0$ in $L^{1}((0, 1))$.

(ii) We suppose for $n \in \mathbb{N}$,
\begin{equation*}
   h_{n} (x) = n^{\frac{3}{2}} x \mathbb{I}_{[\frac{1}{n^{2}}, \frac{1}{n})} (x).
\end{equation*}
Then we have
\begin{equation*}
   \int_{0}^{1} n^{\frac{3}{2}} x \mathbb{I}_{[\frac{1}{n^{2}}, \frac{1}{n})} (x) \, d x  = n^{\frac{3}{2}} \int_{\frac{1}{n^{2}}}^{\frac{1}{n}} x \, d x  = \frac{1}{2} (\frac{1}{\sqrt{n}} - \frac{1}{n^{\frac{5}{2}}}),
\end{equation*}
when $n \to + \infty$, we get $\|h_{n} \|_{1} \to 0$, so we know that $h_{n} \to 0$ in $L^{1}((0, 1))$. But for the $h_{n}^{2}(x)$, we have
\begin{equation*}
   \int_{0}^{1} n^{3} x^{2} \mathbb{I}_{[\frac{1}{n^{2}}, \frac{1}{n})} (x) \, d x  = n^{3} \int_{\frac{1}{n^{2}}}^{\frac{1}{n}} x^{2} \, d x  = \frac{1}{3} n^{3} (\frac{1}{n^{3}} - \frac{1}{n^{6}}) = \frac{1}{3} - \frac{1}{3 n^{3}}.
\end{equation*}
When $n$ tends to infinity, $\int_{0}^{1} n^{3} x^{2} \mathbb{I}_{[\frac{1}{n^{2}}, \frac{1}{n})} (x) \, d x \rightarrow \frac{1}{3}$, which is not goes to $0$. So, we know that $h_{n}^{2}(x)$ don't converge to zero in $L^{1}((0, 1))$. 

(iii) No, $|g_{n}|^{\frac{1}{2}}$ need not converge to zero in $L^{1}(\mathbb{R})$. We can give a counter example. Suppose $g_{n}(x) = \frac{1}{x^{2}} \mathbb{I}_{[n, n^{2}]} (x)$, then we have
\begin{equation*}
   \int_{\mathbb{R}}^{} |g_{n}(x)| \, d x  = \int_{n}^{n^{2}} \frac{1}{x^{2}} \, d x  = \frac{1}{n} - \frac{1}{n^{2}}.
\end{equation*}
When $n$ goes to infinity, we have $\|g_{n}(x) \|_{1} \to 0$, so $g_{n} (x)$ is in $L^{1}(\mathbb{R})$ and $g_{n}$ converges to zero in $L^{1}(\mathbb{R})$. For the $|g_{n}|^{\frac{1}{2}} = \frac{1}{x} \mathbb{I}_{[n, n^{2}]} (x)$, for any $n \in \mathbb{N}$ we have 
\begin{equation*}
   \int_{\mathbb{R}}^{} |g_{n}(x)|^{\frac{1}{2}} \, d x  = \int_{n}^{n^{2}} \frac{1}{x} \, d x  = \ln n.
\end{equation*}
When $n$ goes to infinity, we have $\int_{\mathbb{R}}^{} |g_{n}(x)|^{\frac{1}{2}} \, d x \to + \infty$, so $|g_{n}|^{\frac{1}{2}}$ is in $L^{1}(\mathbb{R})$ but $g_{n}$ don't converges to zero in $L^{1}(\mathbb{R})$.
 



\noindent\rule[0.25\baselineskip]{\textwidth}{0.5pt}

\vspace{8pt}
$\textbf{Exercise 2:}$

Let $f$ be in $L^{\infty} ((0, 1))$. Show that $\|f \|_{p} \rightarrow \|f \|_{\infty}$ as $p \rightarrow \infty$.

\vspace{8pt}
$\textbf{Solution:}$

Since $f \in L^{\infty} ((0,1))$ and $\mu((0, 1)) = 1 < \infty$, then we know that for any $p \geq 1$, $f \in L^{p}((0, 1))$. We denote $t \in [0, \|f \|_{\infty})$, then the set 
\begin{equation*}
   A = \{x \in (0, 1): |f(x)| \geq t \}
\end{equation*}
has positive and bounded measure. Since
\begin{eqnarray*}
\|f\|_{p} & = & \Big{(} \int_{(0, 1)}^{} |f|^{p} \, d \mu \Big{)}^{\frac{1}{p}} \geq \Big{(} \int_{A}^{} |f|^{p} \, d \mu \Big{)}^{\frac{1}{p}} \\
& \geq & \Big{(} t^{p} \mu(A)\Big{)}^{\frac{1}{p}} = t (\mu(A))^{\frac{1}{p}},
\end{eqnarray*}
and $\mu(A)$ is finite, then when $p \to + \infty$, we have $(\mu(A))^{\frac{1}{p}} \to 1$ and
\begin{equation*}
   \liminf_{p \to + \infty} \|f\|_{p} \geq t.
\end{equation*}
Since $t$ is arbitrary and $t \in [0, \|f \|_{\infty})$, we have
\begin{equation*}
   \liminf_{p \to + \infty} \|f\|_{p} \geq \|f \|_{\infty} .
\end{equation*}
On the other hand, as $|f(x)| \leq \|f\|_{\infty}$ for almost every $x \in (0, 1)$, then for $1 \leq q < p$, since $f(x)$ is in $L^{p}((0, 1))$ and $f(x)$ is in $L^{q}((0, 1))$, we have
\begin{eqnarray*}
\|f\|_{p} & = & \Big{(} \int_{(0, 1)}^{} |f|^{p} \, d \mu \Big{)}^{\frac{1}{p}} \\
& = & \Big{(} \int_{(0, 1)}^{} |f|^{q} |f|^{p - q} \, d \mu \Big{)}^{\frac{1}{p}} \\
& \leq & (\|f\|_{\infty})^{\frac{p - q}{p}} (\|f\|_{q})^{\frac{q}{p}}.
\end{eqnarray*}
Since $\|f\|_{q} < + \infty$, then when $p \to + \infty$, we know that
\begin{equation*}
   \limsup_{p \to + \infty} \|f\|_{p} \leq \|f \|_{\infty} .
\end{equation*}
Thus we have
\begin{equation*}
   \limsup_{p \to + \infty} \|f\|_{p} \leq \|f \|_{\infty} \leq \liminf_{p \to + \infty} \|f\|_{p},
\end{equation*}
then we know that $\|f \|_{p} \rightarrow \|f \|_{\infty}$ as $p \rightarrow \infty$.


\noindent\rule[0.25\baselineskip]{\textwidth}{0.5pt}

\vspace{8pt}

$\textbf{Exercise 3:}$

Let $a_{n}$ be a sequence in $[0, 1]$ such that the set $S = \{a_{n}: n = 1, 2, \dots \}$ is dense in $[0, 1]$. Set
\begin{equation*}
   f(x) = \sum_{n = 1}^{\infty} \frac{|x - a_{n}|^{- \frac{1}{2}}}{n^{2}}.
\end{equation*}

(i) Show that $f$ is in $L^{1}([0, 1])$.

(ii) Is $f$ in $L^{2} ([0, 1])$?

(iii) Is there a continuous function
\begin{equation*}
   g : [0, 1] \setminus S \rightarrow \mathbb{R}
\end{equation*}
such that $f = g$ almost everywhere?

\vspace{8pt}
$\textbf{Solution:}$

(i) We check $f \in L^{1}([0,1])$ by definition, since
\begin{eqnarray*}
\int_{0}^{1} \sum_{n = 1}^{\infty} \frac{|x - a_{n}|^{- \frac{1}{2}}}{n^{2}} \, d x  & = & \sum_{n = 1}^{\infty} \frac{1}{n^{2}} \int_{0}^{1} |x - a_{n}|^{- \frac{1}{2}} \, d x \\
& = & \sum_{n = 1}^{\infty} \frac{1}{n^{2}} \Big{[}  \int_{0}^{a_{n}} (a_{n} - x)^{- \frac{1}{2}} \, d x + \int_{a_{n}}^{1} (x - a_{n})^{- \frac{1}{2}} \, d x  \Big{]} \\
& = & \sum_{n = 1}^{\infty} \frac{1}{n^{2}} \Big{[} 2 (a_{n})^{\frac{1}{2}} + 2 (1 - a_{n})^{\frac{1}{2}}  \Big{]}
\end{eqnarray*}
and $a_{n} \in [0, 1]$, then we know that
\begin{equation*}
   \int_{0}^{1} \sum_{n = 1}^{\infty} \frac{|x - a_{n}|^{- \frac{1}{2}}}{n^{2}} \, d x \leq 4 \sum_{n = 1}^{\infty} \frac{1}{n^{2}} < + \infty
\end{equation*}
as $\sum_{n = 1}^{\infty} \frac{1}{n^{2}} = \frac{\pi^{2}}{6}$. Thus we know that $f \in L^{1}([0,1])$.

(ii) No, we can show that $f \notin L^{2}([0,1])$. For $x \in [0, 1]$, we have
\begin{eqnarray*}
\|f\|_{2} & =& \int_{0}^{1} \big{(} \sum_{n = 1}^{\infty} \frac{|x - a_{n}|^{- \frac{1}{2}}}{n^{2}}\big{)}^{2}  \, d x  \\
& \geq & \int_{0}^{1} \sum_{n = 1}^{\infty} \big{(} \frac{|x - a_{n}|^{- \frac{1}{2}}}{n^{2}}\big{)}^{2}  \, d x  \\
& = & \sum_{n = 1}^{\infty} \frac{1}{n^{4}} \int_{0}^{1} |x - a_{n}|^{-1} \, d x .
\end{eqnarray*}
To show $f \notin L^{2}([0,1])$, we just need to prove that $\int_{0}^{1} |x - a_{n}|^{-1} \, d x = + \infty$. We denote $y = x - a_{n}$, then we have
\begin{equation*}
   \int_{0}^{1} |x - a_{n}|^{-1} \, d x = \int_{-a_{n}}^{1 - a_{n}} |y|^{-1} \, d y.
\end{equation*}
Since there exists $k > 0$ such that $\frac{1}{k} < a_{n}$, then we have $- \frac{1}{k} < 0 < 1 - a_{n}$ and
\begin{equation*}
   \int_{0}^{1} |x - a_{n}|^{-1} \, d x \geq \int_{-a_{n}}^{- \frac{1}{k}} |y|^{-1} \, d y = \int_{\frac{1}{k}}^{a_{n}} y^{-1} \, d y = \ln a_{n} + \ln k.
\end{equation*}
When $k \to + \infty$, we have $\ln k+ \ln a_{n} \to \infty$. So, we know that $\int_{0}^{1} |x - a_{n}|^{-1} \, d x = + \infty$. Thus $\|f\|_{2} = + \infty$, then we have $f \notin L^{2}([0,1])$.


(iii) To show that there is a continuous function $g : [0, 1] \setminus S \rightarrow \mathbb{R}$ such that $f = g$ almost everywhere, we just need to prove that $f$ is continuous in $[0, 1] \setminus S$. So for $x \in [0, 1] \setminus S$, we want to show that: $\forall \epsilon > 0$, $\exists \delta > 0$, $\forall y \in [0, 1] \setminus S$ such that $|x - y| < \delta$, we have $|f(x) - f(y)| < \epsilon$. Firstly, we deal with $f(x) - f(y)$, and then we can get
\begin{eqnarray*}
|f(x) - f(y)| & =& \Big{|} \sum_{n = 1}^{\infty} \frac{|x - a_{n}|^{- \frac{1}{2}}}{n^{2}} - \sum_{n = 1}^{\infty} \frac{|y - a_{n}|^{- \frac{1}{2}}}{n^{2}} \Big{|} \\
& = &  \Big{|} \sum_{n = 1}^{\infty} \frac{1}{n^{2}} (|x - a_{n}|^{- \frac{1}{2}} - |y - a_{n}|^{- \frac{1}{2}}) \Big{|} \\
& \leq & \sum_{n = 1}^{\infty} \frac{1}{n^{2}}  \Big{|} |x - a_{n}|^{- \frac{1}{2}} - |y - a_{n}|^{- \frac{1}{2}} \Big{|}.
\end{eqnarray*}
Since $g(x) = |x - a_{n}|^{- \frac{1}{2}}$ is continuous on $(0, 1]$, then $\forall \epsilon > 0$, $\exists \delta > 0$, $\forall y \in (0, 1]$ such that $|x - y| < \delta$, we have 
\begin{equation*}
   \Big{|} |x - a_{n}|^{- \frac{1}{2}} - |y - a_{n}|^{- \frac{1}{2}} \Big{|} < \frac{6}{\pi^{2}} \epsilon.
\end{equation*}
Since $S$ is countable and dense in $[0, 1]$, then for the above $\epsilon$ and $\delta$, $\forall y \in [0, 1] \setminus S$ such that $|x - y| < \delta$, we have
\begin{equation*}
   |f(x) - f(y)| \leq \sum_{n = 1}^{\infty} \frac{1}{n^{2}}  \Big{|} |x - a_{n}|^{- \frac{1}{2}} - |y - a_{n}|^{- \frac{1}{2}} \Big{|} < \frac{\pi^{2}}{6} \times \frac{6}{\pi^{2}} \epsilon = \epsilon.
\end{equation*}
Thus we know that $f(x)$ is continuous in $[0, 1] \setminus S$, which is equivalent to $f(x)$ is continuous almost everywhere in $[0, 1]$. So, there exists a continuous function $g : [0, 1] \setminus S \rightarrow \mathbb{R}$ such that $f = g$ almost everywhere.


\noindent\rule[0.25\baselineskip]{\textwidth}{0.5pt}

\vspace{8pt}

$\textbf{Exercise 4:}$

Let $\mathcal{R}$ be the set of all rectangles $(a_{1}, b_{1}) \times (a_{2}, b_{2})$ in $\mathbb{R}^{2}$ such that $a_{1}, b_{1}, a_{2}, b_{2}$ are rational numbers.

(i) Let $V$ be an open set in $\mathbb{R}^{2}$. Show that 
\begin{equation*}
   V = \bigcup_{R \in \mathcal{R}, R \subset V} R.
\end{equation*}

(ii) Recall that the Borel sets of $\mathbb{R}^{2}$ are the sets in the smallest sigma algebra of $\mathbb{R}^{2}$ containing all open sets. Show that the smallest sigma algebra of $\mathbb{R}^{2}$ containing $\mathcal{R}$ is equal to the set set of Borel sets of $\mathbb{R}^{2}$. 

\vspace{8pt}
$\textbf{Solution:}$

(i) Since $\bigcup_{R \in \mathcal{R}, R \subset V} R \subset V$, to prove $V = \bigcup_{R \in \mathcal{R}, R \subset V} R$, we just need to show that $V \subset \bigcup_{R \in \mathcal{R}, R \subset V} R$. Suppose that $\vec{x} = (x_{1}, x_{2}) \in V$, since $V$ is an open set, then there exists an open ball such that $B(\vec{x}, r) \subset V$, where $r$ is a positive constant and it is called the radius of the ball. So we can find a rectangle $ R = (a_{1}, b_{1}) \times (a_{2}, b_{2})$, whose center is exactly $\vec{x}$. We denote $d((a_{1}, b_{1}), (a_{2}, b_{2}))$ is the distance between $(a_{1}, b_{1})$ and $(a_{2}, b_{2})$. Suppose $d((a_{1}, b_{1}), (a_{2}, b_{2})) < r$, then when know that $\vec{x} \in R$, $R \subset V$ and $R \in \mathcal{R}$. For any $x \in V$ we can do same thing, so we have $V \subset \bigcup_{R \in \mathcal{R}, R \subset V} R$. Thus we know that $V = \bigcup_{R \in \mathcal{R}, R \subset V} R$.

(ii) We denote $\sigma(\mathcal{R})$ is the sigma algebra on $\mathbb{R}^{2}$ generated by sets in $\mathcal{R}$. And we denote $\mathcal{B}(\mathbb{R}^{2})$ as the Borel sets of $\mathbb{R}^{2}$. Since $R$ is open rectangle in $\mathbb{R}^{2}$ and $\mathcal{R} = \{(a_{1}, b_{1}) \times (a_{2}, b_{2}) | a_{i}, b_{i} \in \mathbb{Q}, i = 1, 2\}$, so $\mathcal{R}$ is the open set in $\mathbb{R}^{2}$. Then we know that $\sigma(\mathbb{R}) \subset \mathcal{B}(\mathbb{R}^{2})$. On the other hand, $V$ is open set and by the result we get in (i), we have $V = \bigcup_{R \in \mathcal{R}, R \subset V} R$. Since the number of set $R$ is countable, then we have $V \in \sigma(\mathcal{R})$. Thus the open sets in $\mathbb{R}^{2}$ is subset of $\sigma(\mathcal{R})$. Since $\mathcal{B}(\mathbb{R}^{2})$ is generated by the open sets in $\mathbb{R}^{2}$, then we have $\mathcal{B}(\mathbb{R}^{2}) \subset \sigma(\mathcal{R})$. So we can get $\mathcal{B}(\mathbb{R}^{2}) = \sigma(\mathcal{R})$. Then we know that the smallest sigma algebra of $\mathbb{R}^{2}$ containing $\mathcal{R}$ is equal to the set set of Borel sets of $\mathbb{R}^{2}$.


\newpage

\section{GCE May, 2018}

$\textbf{Exercise 1:}$

Let $(X, \rho)$ be a metric space and $K_{n}$ a sequence of compact subsets of $X$ such that $K_{n+1} \subset K_{n}$. Set
\begin{equation*}
   d_{n} = \sup \{ \rho (x, y): x \in K_{n}, y \in K_{n} \}
\end{equation*}
Assuming that $d_{n}$ converges to zero show that $\bigcap_{n = 1}^{\infty} K_{n}$ is a singleton.

\vspace{8pt}

$\textbf{Solution:}$

Since $\lim_{n \to + \infty} d_{n} = 0$, it means the diameter of the intersection of the $K_{n}$ is zero. So, $\bigcap_{n = 1}^{\infty} K_{n}$ is either empty or consists of a single point. For any $n \in \mathbb{N}$, we pick an element $a_{n} \in K_{n}$. So we can get a point sequence $\{a_{n}\}$, and we have $\{a_{n}: n \in \mathbb{N}\} \in K_{1}$. Since $K_{1}$ is compact, then we know there exists a sub-sequence of $a_{n}$, which is denoted as $a_{n_{k}}$, converges to a point $a$. For any $n \in \mathbb{N}$, since each $K_{n}$ is compact, and $a$ is the limit of a sequence in $K_{n}$, we have $a \in K_{n}$. Thus $a \in \bigcap_{n = 1}^{\infty} K_{n}$. So we know that $\bigcap_{n = 1}^{\infty} K_{n}$ is a singleton.

\noindent\rule[0.25\baselineskip]{\textwidth}{0.5pt}

\vspace{8pt}
$\textbf{Exercise 2:}$

(i) Let $[a, b]$ be an interval in $\mathbb{R}$. If $\tilde{f}$ is continuous on $[a, b]$, $g$ is differentiable on $[a, b]$ and monotonic, and $g'$ is continuous on $[a, b]$, show that there is a $c$ in $[a, b]$, such that
\begin{equation*}
   \int_{a}^{b} \tilde{f}g = g(a) \int_{a}^{c} \tilde{f} + g(b) \int_{c}^{b} \tilde{f}.
\end{equation*}
$\textbf{Hint:}$ Introduce $F(x) = \int_{a}^{x} \tilde{f}$ and integral by parts.

(ii) Show that if $g$ is as specified above and $f$ is in $L^{1} ([a, b])$, there is a $c$ in $[a, b]$ such that 
\begin{equation*}
   \int_{a}^{b} f g = g(a) \int_{a}^{c} f + g(b) \int_{c}^{b} f.
\end{equation*}

\vspace{8pt}
$\textbf{Solution:}$

(i) Since $\tilde{f}$ is continuous on $[a, b]$, we can introduce $F(x) = \int_{a}^{x} \tilde{f}$, so we know that $F'(x) = \tilde{f}(x)$. Then through integral by parts, we have 
\begin{eqnarray*}
\int_{a}^{b} \tilde{f(x)} g(x) \, d x & = & \int_{a}^{b} g(x) \, d F(x) \\
& = & g(b) F(b) - g(a) F(a) - \int_{a}^{b} g'(x) F(x) \, d x  \\
& = & g(b) \int_{a}^{b} \tilde{f} (x) \, d x - g(a) \int_{a}^{a} \tilde{f} (x) \, d x - \int_{a}^{b} g'(x) F(x) \, d x  \\
& = &  g(b) \int_{a}^{b} \tilde{f} (x) \, d x - \int_{a}^{b} g'(x) F(x) \, d x.
\end{eqnarray*}

Since $g$ is differentiable on $[a, b]$ and monotonic, and $g'$ is continuous on $[a, b]$, we know that $g'$ is integrable in $[a, b]$ and $g'(x) \geq 0$ for all $x \in [a, b]$. By the mean value theorem for integral, there exists $c \in [a, b]$, and
\begin{equation*}
   \int_{a}^{b} g'(x) F(x) \, d x = F(c) \int_{a}^{b} g'(x) \, d x = F(c) (g(b) - g(a)).
\end{equation*}
Thus for this $c \in [a, b]$, we have
\begin{eqnarray*}
\int_{a}^{b} \tilde{f(x)} g(x) \, d x & = & g(b) \int_{a}^{b} \tilde{f} (x) \, d x - F(c) (g(b) - g(a)) \\
& = & g(b) \int_{a}^{b} \tilde{f} (x) \, d x - (g(b) - g(a)) \int_{a}^{c} \tilde{f} (x) \, d x \\
& = & g(b) \int_{a}^{b} \tilde{f} (x) \, d x - g(b) \int_{a}^{c} \tilde{f} (x) \, d x + g(a) \int_{a}^{c} \tilde{f} (x) \, d x  \\
& = &  g(b) \int_{c}^{b} \tilde{f} (x) \, d x + g(a) \int_{a}^{c} \tilde{f} (x) \, d x.
\end{eqnarray*}

(ii) Since $C_{c}([a, b])$ is dense in $L^{1}([a, b])$, then we know that for any $f \in L^{1}([0, 1])$, there exists a function sequence $\{f_{n}\} \subset C_{c}([a, b])$ and $\int_{a}^{b} |f_{n} - f| \to 0$ as $n \to + \infty$.
Since $g$ is differentiable on $[a,b]$ and monotonic, we know there exists $K > 0$, and $\forall x \in [a, b]$, we have $|g(x)| \leq K$. So, we have
\begin{equation*}
   \lim_{n \to + \infty} \int_{a}^{b} |g f - g f_{n}| \leq K \lim_{n \to + \infty} \int_{a}^{b}|f - f_{n}| = 0,
\end{equation*}
then by the conclusion we get from (i) we have
\begin{equation*}
   \int_{a}^{b} f g = \lim_{n \to + \infty} \int_{a}^{b} f_{n} g = \lim_{n \to + \infty} \Big{(} g(a) \int_{a}^{c_{n}} f_{n} + g(b) \int_{c_{n}}^{b} f_{n} \Big{)},
\end{equation*}
where $c_{n}$ is depends on $f_{n}$ for each n.

Since $\{c_{n}\} \subset [a, b]$ and $[a, b]$ is compact, there exists a subsequence of $\{c_{n}\}$, which is denoted as $\{c_{n_{k}}\}$, converges to $c$ and $c \in [a, b]$. Thus we have
\begin{eqnarray*}
\int_{a}^{b} f g & = & \lim_{k \to + \infty} \Big{(} g(a) \int_{a}^{c_{n_{k}}} f_{n_{k}} + g(b) \int_{c_{n_{k}}}^{b} f_{n_{k}} \Big{)} \\
& = & \lim_{k \to + \infty} \Big{(} g(a) \int_{a}^{c} f_{n_{k}} + g(a) \int_{c}^{c_{n_{k}}} f_{n_{k}} + g(b) \int_{c_{n_{k}}}^{c} f_{n_{k}} + g(b) \int_{c}^{b} f_{n_{k}} \Big{)} \\
& = &  g(a) \int_{a}^{c} f + g(b) \int_{c}^{b} f + \lim_{k \to + \infty} \Big{(} g(a) \int_{c}^{c_{n_{k}}} f_{n_{k}} +  g(b) \int_{c_{n_{k}}}^{c} f_{n_{k}} \Big{)} \\
& = & g(a) \int_{a}^{c} f + g(b) \int_{c}^{b} f.
\end{eqnarray*}

\noindent\rule[0.25\baselineskip]{\textwidth}{0.5pt}

\vspace{8pt}

$\textbf{Exercise 3:}$

Let $\{f_{n}\}$ be a sequence of functions $f_{n}: [0, 1] \rightarrow \mathbb{R}$.

(i) Define equicontinuity for this sequence.

(ii) Show that if each $f_{n}$ is differentiable on $[0, 1]$ and $|f'_{n}(x)| \leq 1$ for all $x$ in $[0, 1]$ and $n \in \mathbb{N}$, the sequence is equicontinuous.

(iii) Suppose the sequence is uniformly bounded and that (ii) holds. Show that $f_{n}$ has a subsequence which converges uniformly to a continuous function.

(iv) Show through an example that the limit may not be differentiable.

\vspace{8pt}
$\textbf{Solution:}$

(i) The definition of equicontinuity of sequence $\{f_{n}\}$ at point $x$ is as follows: $\forall \epsilon > 0, \exists \delta > 0$, such that $|x - y| < \delta$ and $\forall n \in \mathbb{N}$, we have $|f_{n}(x) - f_{n}(y)| < \epsilon$. And the definition of uniformly equicontinuity of sequence $\{f_{n}\}$ is as follows:
 $\forall x \in [0, 1], \forall \epsilon > 0, \exists \delta > 0$, such that $|x - y| < \delta$ and $\forall n \in \mathbb{N}$, we have $|f_{n}(x) - f_{n}(y)| < \epsilon$.

(ii) Since $f_{n}$ is differentiable on $[0, 1]$, by the mean value theorem, we know that $\forall x, y \in [0, 1]$, there exists a $c \in [x, y]$ and we have
\begin{equation*}
   |f_{n}(y) - f_{n}(x)| = |f'_{n}(c)| |y - x|.
\end{equation*}
Since $|f'_{n}(x)| \leq 1$ for all $x \in [0, 1]$ and $n \in \mathbb{N}$, then we have
\begin{equation*}
   |f_{n}(y) - f_{n}(x)| \leq |y - x|.
\end{equation*}
We set $\delta = \epsilon$, then for all $\epsilon > 0$, there exists $\delta = \epsilon$, such that when $|y - x| < \delta$, $\forall n \in \mathbb{N}$, we have $|f_{n}(y) - f_{n}(x)| < \epsilon$. So we know the sequence $\{f_{n}\}$ is equicontinuous.

(iii) By the Arzel$\grave{a}$-Ascoli theorem, we can get $f_{n}$ has a subsequence which converges uniformly to a continuous function directly. Next we can show the proof of Arzel$\grave{a}$-Ascoli theorem.

We enumerate $\{x_{i}\}_{i \in \mathbb{N}}$ as the rational number in $[0, 1]$. Since the sequence $\{f_{n}\}$ is uniformly bounded, then the set of points $\{f_{n}(x_{1})\}$ is bounded, by the Bolzano-Weierstrass theorem, there is a subsequence $\{f_{n1}(x_{1})\}$ converges. Repeating the same argument for the sequence points $\{f_{n1}(x_{2})\}$, there is a subsequence $\{f_{n2}\}$ of $\{f_{n1}\}$ such that $\{f_{n2}(x_{2})\}$ converges. By induction this process can be continued forever, and so there is a chain of subsequences
\begin{equation*}
   \{f_{n}\} \supset \{f_{n1}\} \supset \{f_{n2}\} \supset \cdots
\end{equation*}
Such that for each $k \in \mathbb{N}$, the subsequence$\{f_{nk}\}$ converges at point $x_{k}$. We choose the diagonal subsequence $\{f_{kk}\}$. Except for the first $n$ functions, $\{f_{kk}\}$ is a subsequence of the $n$th row $\{f_{nk}\}$. Therefore, the sequence $\{f_{kk}\}$ converges simultaneously on all $x_{n}$.

Next we need to show that $\{f_{kk}\}$ is converges uniformly on $[a, b]$. We just need to prove the uniform Cauchy criterion holds. Given any $\epsilon > 0$ and rational $x_{k} \in [0, 1]$, there is an integer $N(\epsilon, x_{k})$ such that when $n, m > N$, we have
\begin{equation*}
   |f_{nn}(x_{k}) - f_{mm}(x_{k})| < \frac{\epsilon}{3}.
\end{equation*}
Since $\bigcap (x_{k} - \frac{1}{n}, x_{k} + \frac{1}{n})$ covers the compact interval [0, 1], then by the Heine-Borel theorem there is a finite subcover, we denote the finite subcover as $U_{1}, \dots, U_{J}$. There exists an integer $K$ such that each open interval $U_{j}$, $1 \leq j \leq J$, contains a rational number $x_{k}$ with $1 \leq k \leq K$. Finally, for any $x \in [0, 1]$, there are $j$ and $k$ so that $x$ and $x_{k}$ belong to the same interval $U_{j}$. For this k, we have
\begin{eqnarray*}
|f_{nn}(x) - f_{mm}(x)| & \leq & |f_{nn}(x) - f_{nn}(x_{k})| + |f_{nn}(x_{k}) - f_{mm}(x_{k})| + |f_{mm}(x_{k}) - f_{mm}(x)| \\
& \leq & \frac{\epsilon}{3} + \frac{\epsilon}{3} + \frac{\epsilon}{3} = \epsilon
\end{eqnarray*}
for all $ N = \max\{N(\epsilon,x_{1}), \dots, N(\epsilon,x_{K})\}$ as $f_{n}$ is equicontinuous. So, for the subsequence $\{f_{kk}\}$, the uniform Cauchy criterion holds. Thus we know that $\{f_{kk}\}$ converges to a continuous function.



(iv) We denote $f_{n} (x) = \sqrt{(x - \frac{1}{2})^{2} + \frac{1}{n}}, \, x \in [0, 1]$. Since for all $n \in \mathbb{N}$ and $x \in [0, 1]$,
\begin{equation*}
   |f'_{n}(x)| = \Big{|} \frac{x - \frac{1}{2}}{\sqrt{(x - \frac{1}{2})^{2} + \frac{1}{n}}} \Big{|} < 1
\end{equation*}
and $f_{n} (x) = \sqrt{(x - \frac{1}{2})^{2} + \frac{1}{n}} < 2$, by the conclusion we get from (ii) and (iii), we know that the sequence $\{f_{n}\}$ is equicontinuous and it has a subsequence which converges uniformly to a continuous function. When $n \to + \infty$, we have $f_{n}(x) \to f(x) = |x - \frac{1}{2}|$, which is not differentiable. So, we know that the limit of this type sequence may not be differentiable.

\noindent\rule[0.25\baselineskip]{\textwidth}{0.5pt}

\vspace{8pt}

$\textbf{Exercise 4:}$

Let $f$ be a lebesgue measurable function such that 
\begin{equation*}
   \int_{0}^{1} f(x) e^{Kx} \, d x = 0
\end{equation*}
for all $K = 1, 2, 3, \dots$. Show that necessarily $f(x) = 0$ for almost every $0 \leq x \leq 1$. 



\newpage


\section{ GCE August, 2018}

$\textbf{Exercise 1:}$

Let $X$ and $Y$ be two metric spaces and $f$ a mapping from $X$ to $Y$.

(i) Show that $f$ is continuous if and only if for every subset $A$ of $X$, $f(\overline{A}) \subset  \overline{f(A)}$.

(ii) Prove or disprove: assume that $f$ is injective. Then $f$ is continuous if and only if for every subset $A$ of $X$, $f(\overline{A}) = \overline{f(A)}$. 

(iii) Prove or disprove: assume that $X$ is compact. Then $f$ is continuous if and only if for every subset $A$ of $X$, $f(\overline{A}) = \overline{f(A)}$. 

\vspace{8pt}

$\textbf{Solution:}$

(i) Firstly, we show that if $f$ is continuous, then for every subset $A$ of $X$, $f(\overline{A}) \subset  \overline{f(A)}$. Since $\overline{f(A)}$ is closed, $f^{-1}(\overline{f(A)})$ is closed as $f$ is continuous, where $f^{-1}(\overline{f(A)})$ is the inverse image of $\overline{f(A)}$. Since $A \subset f^{-1} (f(A))$, then we have $A \subset f^{-1}( \overline{f(A)})$. Since the closure of $A$ is contained in any closed set containing $A$, so we have $\overline{A} \subset f^{-1} (\overline{f(A)})$. Thus we know that for any $x \in \overline{A}$, we have $f(x) \in \overline{f(A)}$, then we get $f(\overline{A}) \subset  \overline{f(A)}$.

Secondly, we show that if for every subset $A$ of $X$, $f(\overline{A}) \subset  \overline{f(A)}$, we have $f$ is continuous. To verify that $f$ is continuous, we just need to show that for any closed set $C \subset Y$, the inverse image of the $C$ under the function $f$ is also a closed set. We denote $D = f^{-1}(C)$, then we want to show $D$ is closed in $X$. Since $f(\overline{D}) \subset \overline{f(D)} = \overline{f(f^{-1}(C))} = \overline{C} = C$, we know that $f(\overline{D}) \subset C$
. Thus we have $\overline{D} \subset f^{-1}(C) = D$, then we know that $D$ is a closed set in $X$. So, $f$ is continuous.

(ii) The statement is not true. We can give a counter example as following. We suppose $X = \mathbb{R}^{+}, Y = \mathbb{R}^{+}$ and $\forall x \in X, f(x) = \frac{1}{x}$. Then $f(x)$ is continuous in $X$. We set $A = [1, + \infty)$, and we have $A \subset X$. So, $\overline{A} = [1, + \infty) = A$, and we know that $f(\overline{A}) = (0, 1]$. Since $f(A) = (0, 1]$, we have $\overline{f(A)} = [0, 1]$. Thus $f(\overline{A}) \subsetneqq \overline{f(A)}$, and we can not say $f(\overline{A}) = \overline{f(A)}$.

(iii) From the question (i), we know that if for every subset $A$ of $X$, $f(\overline{A}) \subset  \overline{f(A)}$, we have $f$ is continuous. Then, if for every subset $A$ of $X$, $f(\overline{A}) = \overline{f(A)}$, we have $f$ is continuous. 

Next we should verify if $f$ is continuous, then for every subset $A$ of $X$, $f(\overline{A}) = \overline{f(A)}$. By the result we get from question (i), we know that if $f$ is continuous, then for every subset $A$ of $X$, $f(\overline{A}) \subset  \overline{f(A)}$. We just need to verify $\overline{f(A)} \subset  f(\overline{A})$. Since $A \subset \overline{A}$, then $f(A) \subset f(\overline{A})$ and $\overline{f(A)} \subset \overline{f(\overline{A})}$. As $A \subset X$ and $X$ is compact, then $\overline{A}$ is compact. As $f$ is continuous, we have $\overline{f(\overline{A})} = f(\overline{A})$. So we can get $\overline{f(A)} \subset  f(\overline{A})$. In summary, when $f$ is continuous, we have $f(\overline{A}) \subset  \overline{f(A)}$ and $\overline{f(A)} \subset  f(\overline{A})$. Thus if $f$ is continuous, for every subset $A$ of $X$, we have $f(\overline{A}) = \overline{f(A)}$.

To sum up, we showed that $f$ is continuous if and only if for every subset $A$ of $X$, $f(\overline{A}) = \overline{f(A)}$.


\noindent\rule[0.25\baselineskip]{\textwidth}{0.5pt}

\vspace{8pt}
$\textbf{Exercise 2:}$

Let $K \subset \mathbb{R}$ have finite measure and let $f \in L^{\infty} (\mathbb{R})$. Show that the function $F$ defined by 
\begin{equation*}
   F(x):= \int_{K}^{} f(x + t) \, d t
\end{equation*}
is uniformly continuous on $\mathbb{R}$.

\vspace{8pt}
$\textbf{Solution:}$

We want to show that $\forall \epsilon > 0$, there exists a $\delta > 0$, such that when $|x - y| < \delta$, we have $|F(x) - F(y)| < \epsilon$. We verify the result by definition. Since
\begin{equation*}
   |F(x) - F(y)|  =  \Big{|} \int_{K}^{} f(x + t) \, d t - \int_{K}^{} f(y + t) \, d t \Big{|},
\end{equation*}
we change the variable and denote $K_{1} = \{k + x| k \in K \}$ and $K_{2} = \{ k + y| k \in K \}$, then we have
\begin{equation*}
   |F(x) - F(y)|  =  \Big{|} \int_{K_{1}}^{} f(t) \, d t - \int_{K_{2}}^{} f(t) \, d t \Big{|}.
\end{equation*}
We denote $\text{ess} \sup_{x \in \mathbb{R}} |f(x)| = C$. Since $f \in L^{\infty} (\mathbb{R})$, then $\forall \epsilon > 0$, there exist a positive number $M$ such that
\begin{equation*}
    \int_{K_{1} \bigcap [-M, M]^{c}}^{} |f(t)| \, d t < \epsilon.
\end{equation*}
Otherwise, $\exists \epsilon > 0$, and $\forall M > 0$, we have $\int_{K_{1} \bigcap [-M, M]^{c}}^{} |f(t)| \, d t \geq \epsilon$. We set $M \to + \infty$, then $\int_{K_{1} \bigcap [-M, M]^{c}}^{} f(t) \, d t  < C \mu \{K_{1} \bigcap [-M, M]^{c} \} \to 0$. It is contradiction. So, for all $\epsilon > 0$, there exist a $M$, such that
\begin{eqnarray*}
|F(x) - F(y)| & = & \Big{|} \int_{K_{1}}^{} f(t) \, d t - \int_{K_{2}}^{} f(t) \, d t \Big{|} \\
& = & \Big{|} \int_{K_{1} \bigcap [-M, M]}^{} f(t) \, d t  + \int_{K_{1} \bigcap [-M, M]^{c}}^{} f(t) \, d t  \\
& & - \int_{K_{2} \bigcap [-M, M]}^{} f(t) \, d t - \int_{K_{2} \bigcap [-M, M]^{c}}^{} f(t) \, d t \Big{|}   \\
&\leq&  \Big{|} \int_{K_{1} \bigcap [-M, M]}^{} f(t) \, d t - \int_{K_{2} \bigcap [-M, M]}^{} f(t) \, d t \Big{|} + 2 \epsilon.
\end{eqnarray*}
We denote $S = (K_{1} \bigcap [-M, M]) \Delta (K_{2} \bigcap [-M, M])$, then we have
\begin{equation*}
    |F(x) - F(y)|  \leq  \int_{S}^{} |f(t)| \, d t + 2 \epsilon \leq C \mu \{S \} + 2 \epsilon.
\end{equation*}
As $K_{1} \bigcap [-M, M]$ and $K_{2} \bigcap [-M, M]$ are finite, and $K_{1} = \{k + x| k \in K \}$, $K_{2} = \{ k + y| k \in K \}$, we can cover the set $S$ by several open sets whose measure is $|y -x|$, then we have
\begin{equation*}
    |F(x) - F(y)| \leq C m |y - x| + 2 \epsilon,
\end{equation*}
where C is a positive number. We set $\delta = \frac{\epsilon}{Cm}$, then we have
\begin{equation*}
    |F(x) - F(y)| \leq  3 \epsilon,
\end{equation*}
so, $F(x)$ is uniformly continuous on $\mathbb{R}$.

\noindent\rule[0.25\baselineskip]{\textwidth}{0.5pt}

\vspace{8pt}

$\textbf{Exercise 3:}$

Let $\{f_{n}\}$ be a sequence in $L^{1}(\mathbb{R})$ such that $f_{n} \to 0$ a.e.

(i) Show that if $\{f_{2n}\}$ is increasing and $\{f_{2n + 1} \}$ is decreasing, then
\begin{equation*}
    \int_{}^{} f_{n} \to 0.
\end{equation*}

(ii) Prove or disprove: if $\{f_{kn} \}$ is decreasing for every prime number $k$, then 
\begin{equation*}
    \int_{}^{} f_{n} \to 0.
\end{equation*}
(Note on notation: e.g., if $k = 2$, then $\{f_{kn}\} = \{f_{2n}\}$. Note also that 1 is not prime).

\vspace{8pt}
$\textbf{Solution:}$

(i) Firstly, we consider the sequence $\{f_{2n} - f_{2}\}$. Since $\{f_{2n}\}$ is increasing, $f_{2n} \to 0$ and $\{f_{n}\} \in L^{1}(\mathbb{R})$ for all n, then $\{f_{2n} - f_{2}\}$ is increasing and $f_{2n} - f_{2} \to -f_{2}$ a.e., then by the monotone convergence theorem, we have
\begin{equation*}
    \lim_{n \to + \infty} \int_{}^{} (f_{2n} - f_{2}) =  \int_{}^{} \lim_{n \to + \infty} (f_{2n} - f_{2}) = \int_{}^{}  - f_{2},
\end{equation*}
then we have
\begin{equation*}
    \lim_{n \to + \infty} \int_{}^{} f_{2n} =  0.
\end{equation*}
Similarly, as $\{f_{2n + 1} \}$ is decreasing, we know that  $\{f_{1} - f_{2n -1} \}$ is a increasing sequence and $f_{1} - f_{2n-1} \to f_{1}$ a.e., by the monotone convergence theorem, we have
\begin{equation*}
    \lim_{n \to + \infty} \int_{}^{} (f_{1} - f_{2n-1}) =  \int_{}^{} \lim_{n \to + \infty} (f_{1} - f_{2n-1}) = \int_{}^{}  f_{1},
\end{equation*}
then we have
\begin{equation*}
    \lim_{n \to + \infty} \int_{}^{} f_{2n-1} =  0.
\end{equation*}
Then we show that for any subsequence of $\{\int f_{n}\}$, which denoted as $\{\int f_{n_{k}}\}$, we can find a subsequence of $\{\int f_{n_{k}}\}$, which is denoted as $\{\int f_{n_{k_{l}}}\}$, and we have
\begin{equation*}
    \lim_{n \to + \infty} \int_{}^{} f_{n_{k_{l}}} =  0.
\end{equation*}
For the subsequence $\{\int f_{n_{k}}\}$, we take the even number in the indicator set ${n_{k}}$ if it is infinite, or we can take the odd number in the indicator set ${n_{k}}$ if it is infinite, then we can get the subsequence of $\{\int f_{n_{k}}\}$, which is denoted as $\{\int f_{n_{k_{l}}}\}$. Since we have showed that $\lim_{n \to + \infty} \int_{}^{} f_{2n} =  0$ and $\lim_{n \to + \infty} \int_{}^{} f_{2n-1} =  0$, then we know that $\lim_{n \to + \infty} \int_{}^{} f_{n_{k_{l}}} =  0$. So, we know that
\begin{equation*}
    \int_{}^{} f_{n} \to 0.
\end{equation*}

(ii) The statement is not true. We can find a counter example as follows. We define
\begin{equation*}
    f_{p} (x) = p \, \mathbb{I}_{[0, \frac{1}{p}]} (x),
\end{equation*}
where $p$ is a prime number and
\begin{equation*}
    f_{m} (x) = 2 \, \mathbb{I}_{[0, \frac{1}{m}]} (x),
\end{equation*}
where $m$ is a not prime number. Then we know that $\{f_{np}\}$ is decreasing for every prime number $p$. But we can find a subsequence of $\{f_{n}\}$, which is denoted as $\{f_{p}\}$, $p$ is the prime number, and $\lim_{n \to + \infty} \int_{}^{} f_{p} \neq 0$ as
\begin{equation*}
    \lim_{p \to + \infty} \int_{}^{} f_{p} = \lim_{p \to + \infty} \int_{}^{} p \, \mathbb{I}_{[0, \frac{1}{p}]} (x) \, d x = 1.
\end{equation*}

\newpage

\section{GCE January, 2019}

$\textbf{Exercise 1:}$

Let $E:= [0, 1] - S_{\mathbb{Q}} = [0, 1] \bigcap (S_{\mathbb{Q}})^{c}$ where $S_{\mathbb{Q}} := \{x \in [0, 1] | x = \frac{\sqrt{p}}{q}   \,  \text{for some} \, p, q \in \mathbb{Z}^{+} \}$. Prove or disprove: There exists a closed, uncountable subset $F \subset E$.

\vspace{8pt}

$\textbf{Solution:}$

This proposition is true. Since $S_{\mathbb{Q}}$ is a countable set, there exists a bijection between $S_{\mathbb{Q}}$ and the positive rational number in the interval $[0, 1]$, so we can enumerate the set $S_{\mathbb{Q}}$ as $\{a_{n} | n \in \mathbb{N} \}$. That is to say we have $S_{\mathbb{Q}} = \{a_{n} | n \in \mathbb{N} \}$. And then we consider the union 
$$\bigcup_{n = 1}^{+ \infty}(a_{n} - \frac{\epsilon}{2^{n}}, a_{n} + \frac{\epsilon}{2^{n}}),$$
it is an open set, we denote it as $A$, then $A = \bigcup_{n = 1}^{+ \infty}(a_{n} - \frac{\epsilon}{2^{n}}, a_{n} + \frac{\epsilon}{2^{n}})$. And when $\epsilon \to 0$, we know that $A \subset [0, 1]$ and $S_{\mathbb{Q}} \subset A$.

Since $A$ is an open set, then $[0, 1] \bigcap (A)^{c}$ is a closed set. We denote $F = [0, 1] \bigcap (A)^{c}$, since the measure of set $A$ is
\begin{equation*}
    m(A) = 2 \sum_{n = 1}^{+ \infty}  \frac{\epsilon}{2^{n}} = 2 \epsilon,
\end{equation*}
then we have $m(F) = 1 - 2 \epsilon > 0$, so, the set $F$ is uncountable. Since $F \subset E$ and it is both closed and uncountable, then the proposition is true.

For any countable set $S$, $S \subset [0, 1]$, let $E = [0, 1] - S$, we can find a closed, uncountable subset $F \subset E$, and we have the supremum of the measure of $F$ is 1.

\noindent\rule[0.25\baselineskip]{\textwidth}{0.5pt}

\vspace{8pt}
$\textbf{Exercise 2:}$

For $x$ in $[-1, 1]$ set $P_{n} (x) = c_{n} (1 - x^{2})^{n}$ where $c_{n}$ is such that $\int_{-1}^{1} P_{n} = 1.$

(i) Show that there is a positive constant $C$ such that $c_{n} \leq C \sqrt{n}$.

(ii) Let $f$ be a real valued continuous function on $[0, 1]$ such that $f(0) = f(1) = 0$. Set for $x$ in $[0, 1]$
\begin{equation*}
    f_{n}(x) = \int_{0}^{1} P_{n}(x-t) f(t) \, d t
\end{equation*}
Show that $f_{n}$ is uniformly convergence to $f$.

(iii) Let $g$ be in $L^{1}((0, 1))$. Defining $g_{n}(x) = \int_{0}^{1} P_{n} (x- t) g(t) \, d t$, is $g_{n}$ uniformly convergence to $g$ in $(0, 1)$? Does $g_{n}$ converge to $g$ in $L^{1}((0, 1))$?

\vspace{8pt}
$\textbf{Solution:}$

(i) Method 1:

Since $\int_{-1}^{1} c_{n} (1-x^{2})^{n}\, d x = 1$, then we have
\begin{equation*}
   c_{n} = \frac{1}{2 \int_{0}^{1}(1-x^{2})^{n} \, d x }.
\end{equation*}
Next we need to find a lower bound of the integral term $\int_{0}^{1}(1-x^{2})^{n} \, d x$. Since for $n > 1$,
\begin{eqnarray*}
\int_{0}^{1}(1-x^{2})^{n} \, d x &\geq& \int_{0}^{\frac{1}{\sqrt{n}}}(1-x^{2})^{n} \, d x  \\
            &\geq& \frac{1}{\sqrt{n}} (1 - \frac{1}{n})^{n},
\end{eqnarray*}
then we have $c_{n} \leq \frac{\sqrt{n}}{2 (1-\frac{1}{n})^{n}}$. We just need to find a lower bound of $(1 - \frac{1}{n})^{n}$. Since $(1 - \frac{1}{n})^{n} = 1 - C_{n}^{1} \frac{1}{n} + C_{n}^{2} \frac{1}{n^{2}} + \cdots + (-\frac{1}{n})^{n} > \frac{1}{3} - \frac{2}{6n^{2}}  > \frac{1}{4}$ as $n > 1$, then we set $C = 2$, we have $c_{n} \leq C \sqrt{n}$ for $n>1$. For $n = 1$, we get $c_{1} = \frac{3}{4} < 2$, then when $C = 2$, we have $c_{n} \leq C \sqrt{n}$ holds.

Method 2:

We change the element and define $x = \sin y$, then we have $\int_{0}^{\frac{\pi}{2}}  c_{n} \cos^{2n+1}y \, d y = \frac{1}{2}$. Since 
\begin{equation*}
   \int_{0}^{\frac{\pi}{2}} \cos^{2n+1}y \, d y = 2n \int_{0}^{\frac{\pi}{2}} \cos^{2n-1}y \, d y - 2n \int_{0}^{\frac{\pi}{2}} \cos^{2n+1}y \, d y,
\end{equation*}
we denote $I_{2n + 1} = \int_{0}^{\frac{\pi}{2}} \cos^{2n+1}y \, d y$, then we have $(2n + 1)I_{2n+1} = 2n I_{2n-1}$. Since $I_{1} = \int_{0}^{\frac{\pi}{2}} \cos y \, d y = 1$, we have $\int_{0}^{\frac{\pi}{2}} \cos^{2n+1}y \, d y = \frac{(2n)!!}{(2n+1)!!}$. And since
\begin{eqnarray*}
\frac{(2n)!!}{(2n+1)!!} &=&  \frac{2n (2n-2) \cdots 2}{(2n+1) (2n-1) \cdots 3}  \\
            &\geq& \frac{\sqrt{2n+1}\sqrt{2n-1}\sqrt{2n-1}\sqrt{2n-3} \cdots \sqrt{3}\sqrt{1}}{(2n+1) (2n-1) \cdots 3} \\
            &=& \frac{1}{\sqrt{2n+1}},
\end{eqnarray*}
then we have $c_{n} \leq \frac{\sqrt{2n+1}}{2}$. We set $C = 1$, then we have $c_{n} \leq C \sqrt{n}$.


(ii) Firstly we extend $f(x)$ to a function from $\mathbb{R}$ to $\mathbb{R}$ by zero. Then we have
\begin{equation*}
    f_{n}(x) = \int_{0}^{1} P_{n}(x-t) f(t) \, d t = \int_{\mathbb{R}}^{} P_{n}(x-t) f(t) \, d t,
\end{equation*}
then we change the element as $x - t = y$, we have
\begin{equation*}
    f_{n}(x) = \int_{\mathbb{R}}^{} P_{n}(y) f(x - y) \, d y.
\end{equation*}
Then we know that
\begin{eqnarray*}
|f_{n}(x) - f(x)| &=& \Big{|}\int_{\mathbb{R}}^{} P_{n}(y) f(x - y) \, d y - \int_{-1}^{1} P_{n}(y) f(x) \, d y \Big{|}  \\
&=& \Big{|} \int_{-1}^{1} P_{n}(y) (f(x - y) - f(x)) \, d y + \int_{([-1,1])^{c}}^{} P_{n}(y) f(x-y) \, d y \Big{|} \\
&\leq& \int_{-1}^{1} P_{n}(y) | (f(x - y) - f(x))| \, d y + \int_{([-1,1])^{c}}^{} |P_{n}(y) f(x-y)| \, d y.
\end{eqnarray*}
Since when $x \in [0, 1]$ and $y \in ([-1, 1])^{c}$, we have $x - y > 1$ or $x - y < 0$, then we have $f(x -y) = 0$, so we have
\begin{equation*}
    |f_{n}(x) - f(x)| \leq \int_{-1}^{1} P_{n}(y) | (f(x - y) - f(x))| \, d y.
\end{equation*}
And by the definition of continuous, we have $\forall \epsilon > 0$, there $\exists \delta$, when $|x - y -x| < \delta$, we have $|f(x-y) - f(x)| < \epsilon$. We denote $S = [-1,1] \bigcap [-\delta, \delta]$, since $f(x)$ is continuous in $\mathbb{R}$, we denote $\text{sup}_{x \in [0, 1]} f(x) = M$, then we have $M < + \infty$ and
\begin{eqnarray*}
|f_{n}(x) - f(x)| &\leq& \int_{-\delta}^{\delta} P_{n}(y) | (f(x - y) - f(x))| \, d y  + \int_{S}^{} P_{n}(y) | (f(x - y) - f(x))| \, d y  \\
&\leq& \epsilon \int_{-\delta}^{\delta} P_{n}(y) \, d y  + 2M \int_{S}^{} P_{n}(y) \, d y   \\
&\leq& \epsilon + 2M \int_{S}^{} c_{n} (1 - y^{2})^{n} \, d y   \\
&\leq& \epsilon + 4M C \sqrt{n} \int_{\delta}^{1} (1 - y^{2})^{n} \, d y   \\
&\leq& \epsilon + 4M C \sqrt{n} (1 - \delta)(1 - \delta^{2})^{n}.
\end{eqnarray*}
Since $\lim_{n \to + \infty} 4M C \sqrt{n} (1 - \delta)(1 - \delta^{2})^{n} = 0 $, then we can say that there exists a $N \in \mathbb{N}$, when $n > N$, we have $4M C \sqrt{n} (1 - \delta)(1 - \delta^{2})^{n} < \epsilon$. Overall, we know that $\forall x \in [0, 1], \forall \epsilon > 0$, there exists a $N \in \mathbb{N}$, when $n > N$, we have $|f_{n}(x) - f(x)| < 2 \epsilon$, so that $f_{n}$ is uniformly converges to $f$.


(iii) Firstly, the $g_{n}(x)$ is not uniformly convergent to $g$ in $(0, 1)$, we can give an counter example as following. We define 
\begin{equation*}
g(x) =
\left\{
             \begin{array}{cl}
             1, & x = \frac{1}{2} \\
             0, & x \in (0, \frac{1}{2}) \bigcup (\frac{1}{2}, 1),
             \end{array}
\right.
\end{equation*}
obviously $g(x)$ is not continuous in $(0, 1)$, but we have $g_{n} (x) = \int_{0}^{1} P_{n}(x -t)g(t) \, d t = 0, \forall x \in (0, 1)$. Then $g_{n} (x)$ is continuous in $[0, 1]$. Since $g(x)$ is not continuous in $(0, 1)$, we can say that $g_{n}(x)$ is not uniformly convergent to $g(x)$ in $(0, 1)$.

Secondly, we can show that $g_{n}(x)$ convergent to $g(x)$ in $L^{1}((0, 1))$. Since the continuous functions with compact support are dense in $L^{1}$ space, then for all $\epsilon > 0$, there exist a continuous function $f(x) \in C_{c}([0, 1])$, such that $\|f- g \|_{1} < \epsilon$. We define the $f_{n}(x)$ as the section (ii), then we have
\begin{equation*}
\|g- g_{n} \|_{1} \leq  \|g- f \|_{1} + \|f- f_{n} \|_{1} + \|f_{n}- g_{n} \|_{1}.
\end{equation*}
Since $f_{n}$ is uniformly converges to $f$, for all $\epsilon > 0$, there exists a $N \in \mathbb{N}$, when $n > N$, we have $\|f- f_{n} \|_{1} < \epsilon$. And for the same $\epsilon$, by the property that continuous function is dense in $L^{1}$ space, we have $\|f- g \|_{1} < \epsilon$. Next we verify that $\|f_{n}- g_{n} \|_{1} < \epsilon$. Since
\begin{eqnarray*}
\|f_{n}- g_{n} \|_{1} &=& \int_{0}^{1} \Big{|} \int_{0}^{1}  P_{n}(x-t) g(t) - \int_{0}^{1}  P_{n}(x-t) f(t) \, d t \Big{|} d x  \\
&=& \int_{0}^{1} \Big{|} \int_{0}^{1}  P_{n}(x-t) (g(t) - f(t)) \, d t \Big{|} d x  \\
&\leq& \int_{0}^{1}  \int_{0}^{1}  P_{n}(x-t) |g(t) - f(t)|\, d t  d x,
\end{eqnarray*}
and $P_{n}(x-t)$ is continuous for $t \in [0, 1]$, then we can find the upper bound for  $P_{n}(x-t)$, we denote it as $C$, then we have
\begin{eqnarray*}
\|f_{n}- g_{n} \|_{1} &\leq&  \int_{0}^{1}  \int_{0}^{1}  P_{n}(x-t) |g(t) - f(t)|\, d t  d x \\
&\leq&  C \int_{0}^{1}  \int_{0}^{1} |g(t) - f(t)|\, d t  d x \\
&=&  C \int_{0}^{1} |g(t) - f(t)|\, d t \\
&=& C \|g-f \|_{1}.
\end{eqnarray*}
Since $\|g-f \|_{1} < \epsilon$, we have $\|g-g_{n} \|_{1} < (2 + \frac{1}{C}) \epsilon$ for all $\epsilon > 0$. So, we know that $g_{n}(x)$ convergent to $g(x)$ in $L^{1}((0, 1))$.



\noindent\rule[0.25\baselineskip]{\textwidth}{0.5pt}

\vspace{8pt}

$\textbf{Exercise 3:}$

Give an example of $f_n, f : \mathbb{R} \mapsto [0, \infty)$ such that $f_{n} \in L^{1}(\mathbb{R})$ for every $n \in \mathbb{N}$, $f \in L^{2}(\mathbb{R})$, $f_{n} \leq f$ for every $n \in \mathbb{N}$, $f_{n} \to 0$ a.e., and $\int_{}^{} f_{n} \nrightarrow 0$.

\vspace{8pt}
$\textbf{Solution:}$

We define the $f(x) = \frac{1}{x} \mathbb{I}_{[1, +\infty)}$ and $f_{n}(x) = \frac{1}{x} \mathbb{I}_{[n, n^{2}]}$. For a fixed $n$, we have
\begin{equation*}
\int_{\mathbb{R}}^{} |f_{n}(x)| \, d x = \int_{n}^{n^{2}} \frac{1}{x} \, d x = \ln{n},
\end{equation*}
so we have $f_{n} \in L^{1}(\mathbb{R})$ for every $n \in \mathbb{N}$. And since
\begin{equation*}
\int_{\mathbb{R}}^{} |f(x)|^{2} \, d x = \int_{1}^{+ \infty} \frac{1}{x^{2}} \, d x = 1,
\end{equation*}
so we know that $f \in L^{2}(\mathbb{R})$. Since for all n, $f_{n}$ is just a part of $f$ and $f > 0$, then we have $f_{n} \leq f$ for every $n \in \mathbb{N}$. When $n \to +\infty$, we have $f_{n}(x) \leq \frac{1}{n}$, so that $f_{n} \to 0$ almost everywhere. And we calculate the integral of $f_{n}$, we have
\begin{equation*}
\int_{\mathbb{R}}^{} f_{n}(x) \, d x = \int_{n}^{n^{2}} \frac{1}{x} \, d x = \ln{n},
\end{equation*}
when $n \to + \infty$, $\ln{n} \to + \infty$, so we can get $\int_{}^{} f_{n} \nrightarrow 0$.

\newpage


\section{GCE May, 2019}

$\textbf{Exercise1:}$

Let $V$ be a normed vector space and $S$ a subset of $V$. Let $S^{c}$ be the complement of $S$. Let $x$ be in $S$ and $y$ be in $S^{c}$. The line segment $[x, y]$ is by definition the set
\begin{equation*}
    \{(1-t)x + t y : t \in [0, 1] \}.
\end{equation*}

Show that the intersection of $[x, y]$ and $\partial S$ is non empty, where $\partial S$ is the boundary of $S$ (by definition the boundary of $S$ is the set of points that are in the closure of $S$ and that are not in the interior of $S$).

\vspace{8pt}
$\textbf{Solution:}$

We want to prove that the intersection of $[x, y]$ and $\partial S$ is non empty, then we need to find a $t^{*} \in [0, 1]$, $\forall \delta >0$, $B((1-t^{*})x + t^{*}y, \delta) \bigcap S \neq \varnothing$ and $B((1-t^{*})x + t^{*}y, \delta) \bigcap S^{c} \neq \varnothing$, where $B((1-t^{*})x + t^{*}y, \delta) = \{(1-t)x + t y: |t - t^{*} | < \delta, t \in [0, 1] \}.$ Then we need to find that $t^{*}$. We define
\begin{equation*}
    Z = \{t: (1-t) x + t y \in S, t \in [0,1]\},
\end{equation*}
and we denote $t^{*} = \text{sup} {Z}$. And we denote $B((1-t^{*})x + t^{*}y, \delta) = B_{t^{*}, \delta}$. 

Firstly, we show that $B_{t^{*}, \delta} \bigcap S \neq \varnothing$. Since $t^{*} = \text{sup} Z$, by the definition of $t^{*}$ then we have $\forall \delta > 0, \exists \epsilon = \frac{\delta}{2}, \big(1 - (t^{*} - \epsilon)x \big) + (t^{*} - \epsilon) y \in S$. And since $|t^{*} - \epsilon - t^{*}| = \epsilon < \delta$, then $\big(1 - (t^{*} - \epsilon)x \big) + (t^{*} - \epsilon) y \in B_{t^{*}, \delta}$, such that $B_{t^{*}, \delta} \bigcap S \neq \varnothing$.

Secondly, we need verify that $B_{t^{*}, \delta} \bigcap S^{c} \neq \varnothing$. Suppose $B_{t^{*}, \delta} \bigcap S^{c} = \varnothing$, then we have that $B_{t^{*}, \delta} \subset S$. Since $t^{*} = \text{sup} Z$, by the definition of $t^{*}$ then we have $\forall \delta > 0, \exists \epsilon = \frac{\delta}{2}, \big(1 - (t^{*} + \epsilon)x \big) + (t^{*} + \epsilon) y \notin S$. And since $|t^{*} - \epsilon - t^{*}| = \epsilon < \delta$, then $\big(1 - (t^{*} + \epsilon)x \big) + (t^{*} + \epsilon) y \in B_{t^{*}, \delta}$. It is contradict with $B_{t^{*}, \delta} \subset S$, then we know that $B_{t^{*}, \delta} \bigcap S^{c} \neq \varnothing$.

Overall, we find $t^{*} \in [0, 1]$, $(1- t^{*})x + t^{*} y \in [x, y]$, $\forall \delta > 0$, we have $B_{t^{*}, \delta} \bigcap S \neq \varnothing$ and $B_{t^{*}, \delta} \bigcap S^{c} \neq \varnothing$, such that $(1- t^{*})x + t^{*} y \in \partial{S}$. So, we conclude that the intersection of $[x, y]$ and $\partial S$ is non empty.

\vspace{8pt}

\noindent\rule[0.25\baselineskip]{\textwidth}{0.5pt}


$\textbf{Exercise2:}$

Let $(X, \mathcal{A}, \mu)$ be a measure space. Let $g$ be a measurable function defined on $X$. Set 
\begin{equation*}
    p_{g} (t) = \mu ({x \in X : |g(x)| > t}).
\end{equation*}

(i) If $f$ is in $L^{1}(X)$ show that there is a constant $C > 0$ such that $p_{f}(t) \leq \frac{C}{t}$.

(ii) Find a measurable function $h$ defined almost everywhere on $\mathbb{R}$ such that $\exists C > 0$, $p_{h} (t) \leq \frac{C}{t}$ and $h$ is not in $L^{1}(\mathbb{R})$.

\vspace{8pt}
$\textbf{Solution:}$

(i) Since $f \in L^{1}(X)$, then $\exists C > 0$, $\int_{X}^{} |f| \, d \mu \leq C < + \infty$. We can decompose the integral as following:
\begin{eqnarray*}
\int_{X}^{} |f| \, d \mu &=& \int_{X}^{} |f| \mathbb{I}_{\{|f| > t\}} \, d \mu + |f| \mathbb{I}_{\{|f| \leq t\}} \, d \mu  \\
            &=& \int_{X}^{} |f| \mathbb{I}_{\{|f| > t\}} \, d \mu + \int_{X}^{} |f| \mathbb{I}_{\{|f| \leq t\}} \, d \mu  \\
            &\geq & \int_{X}^{} |f| \mathbb{I}_{\{|f| > t\}} \, d \mu  \\
            & \geq & t \int_{X}^{} \mathbb{I}_{\{|f| > t\}} \, d \mu \\
            & = & t p_{f}(t).
\end{eqnarray*}
Then we have $t p_{f}(t) \leq C$, such that $p_{f}(t) \leq \frac{C}{t}$.

(ii) We suppose that 
\begin{equation*}
h(x) =
\left\{
             \begin{array}{cl}
             0, & x = 0 \\
             \frac{1}{|x|}, & x \neq 0,
             \end{array}
\right.
\end{equation*}
then $h(x) \notin L^{1}(\mathbb{R})$ since $\frac{1}{x} \notin L^{1}([0, + \infty))$. Since
\begin{equation*}
p_{t}(t) = \int_{\mathbb{R}}^{} \mathbb{I}_{\{|h| > t\}} \, d \mu  = \int_{\mathbb{R}}^{} \mathbb{I}_{\{|x| < \frac{1}{t} \}} \, d \mu  =  \int_{\{|x| < \frac{1}{t} \}}^{} \, d \mu,
\end{equation*}
hence we can set $C = 2$ and $p_{h} (t) \leq \frac{C}{t}$ and $h$ is not in $L^{1}(\mathbb{R})$.

\vspace{8pt}

\noindent\rule[0.25\baselineskip]{\textwidth}{0.5pt}


$\textbf{Exercise3:}$

Let $\{f_{n}\} : [0, 1] \mapsto [0, \infty) $ be a sequence of functions, each of which is non-decreasing on the interval $[0, 1]$. Suppose the sequence is uniformly bounded in $L^{2}([0, 1])$. Show that there exists a sub sequence that converges in $L^{1}([0, 1])$.

\vspace{8pt}
$\textbf{Solution:}$

Since $f_{n}$ is non-decreasing, then for $x \in [0, 1]$, we have $\int_{x}^{1} f_{n}(y) \, d y \geq (1-x) f_{n}(x)$. On the other hand, since the sequence is uniformly bounded in $L^{2}([0, 1])$, we have $\forall n \in \mathbb{N}$, $\exists C > 0$, and $\| f_{n} \|_{2} \leq C$. And then we have
\begin{eqnarray*}
\int_{x}^{1} f_{n}(y) \, d y &=& \int_{0}^{1} f_{n}(y) \mathbb{I}_{[x, 1]} (y) \, d y  \\
            &\leq & \Big( \int_{0}^{1} f^{2}_{n}(y) \, d y \Big)^{\frac{1}{2}} \Big( \int_{0}^{1} \mathbb{I}^{2}_{[x, 1]} (y) \, d y \Big)^{\frac{1}{2}} \\
            &\leq & C (1-x)^{\frac{1}{2}}.
\end{eqnarray*}
Such that we have $(1-x) f_{n}(x) \leq C (1-x)^{\frac{1}{2}}$, then $f_{n}(x) \leq C (1-x)^{- \frac{1}{2}}$. Until now we find a type of function $f(x) = C (1-x)^{- \frac{1}{2}}$ that can control the sequence $f_{n}$, where $C$ is from the bound of $f_{n}$ in the $L^{2}([0, 1])$.

\vspace{8pt}

\noindent\rule[0.25\baselineskip]{\textwidth}{0.5pt}

$\textbf{Exercise4:}$

Consider the sequence of functions $f_{n}: [0, 1] \mapsto \mathbb{R}$ where $f_{1}(x) = \sqrt{x}, f_{2}(x) = \sqrt{x + \sqrt{x}}, f_{3}(x) = \sqrt{x + \sqrt{x + \sqrt{x}}}$, and in general $f_{n}(x) = \sqrt{x + \sqrt{x + \sqrt{\dots + \sqrt{x}}}}$ with $n$ roots.

(i) Show that this sequence converges pointwise on $[0, 1]$ and find the limit function $f$ such that $f_{n} \rightarrow f$.

(ii) Does this sequence converge uniformly on $[0, 1]$? Prove or disprove uniform convergence.

\vspace{8pt}
$\textbf{Solution:}$

(i) Firstly, we show that the sequence $f_{n}(x)$ is non-decreasing for the fixed $x$. We use the mathematical induction. For the fixed $x \in [0, 1]$, when $k = 1$, since $f_{k} (x) = \sqrt{x}$ and $f_{k+1} (x) = \sqrt{x + \sqrt{x}}$, then $f_{k}(x) \leq f_{k+1}(x)$. We suppose when $k = n-1$, the formula $f_{k}(x) \leq f_{k+1}(x)$ holds, which is equivalent to $f_{n-1}(x) \leq f_{n}(x)$ . We want to verify $f_{n}(x) \leq f_{n+1}(x)$. Since $f_{n}(x) = \sqrt{x + f_{n-1}(x)}$ and $f_{n+1}(x) = \sqrt{x + f_{n}(x)}$, when $f_{n-1}(x) \leq f_{n}(x)$, we have $\sqrt{x + f_{n-1}(x)} \leq \sqrt{x + f_{n}(x)}$, such that $f_{n}(x) \leq f_{n+1}(x)$. So when $k = n$, the formula $f_{k}(x) \leq f_{k+1}(x)$ can also hold. Thus we know that the sequence $f_{n}(x)$ is non-decreasing for the fixed $x$.

Then, we show that the sequence $f_{n}(x)$ is uniformly bounded. We also use the mathematical induction. When $k =1$, $f_{k}(x) = \sqrt{x} < \sqrt{3}$. We suppose that when $k = n-1$, we have $f_{k}(x) < \sqrt{3}$. When $k =n$, $f_{n}(x) = \sqrt{f_{n-1}(x) + x} < \sqrt{\sqrt{3}+1} < \sqrt{3}$. Such that we get a uniform bound of sequence $f_{n}$.

Overall, since the sequence $f_{n}(x)$ is non-decreasing for the fixed $x$, and the sequence $f_{n}(x)$ has uniformly bound $\sqrt{3}$, then this sequence converges pointwise on $[0, 1]$. We suppose the sequence $f_{n}(x)$ converges pointwise on $[0, 1]$ to $f(x)$. Since $f_{n+1}(x) = \sqrt{x + f_{n}(x)}$, when $n \to \infty$, we have $f(x) = \sqrt{x + f(x)}$. So we can get $f^{2}(x) - f(x) - x = 0$, such that $f(x) = \frac{1 + \sqrt{1 + 4 x}}{2}$ as $f(x) \geq 0$. When $x =  0$, we have $f_{n} (x) = 0, \forall n$. Then we have
\begin{equation*}
f(x) =
\left\{
             \begin{array}{cl}
             0, & x = 0 \\
             \frac{1 + \sqrt{1 + 4 x}}{2}, & x \in (0, 1].
             \end{array}
\right.
\end{equation*}

(ii) Since for all $n \in \mathbb{N}$, $ f_{n}(x)$ is continuous, if the sequence $f_{n}(x)$ converge uniformly on $[0, 1]$ to $f(x)$, then $f(x)$ should be continuous. Since the $f(x)$ we get in (i) is not a continuous function, then this sequence $f_{n}(x)$ is not converge uniformly on $[0, 1]$.

\vspace{8pt}

\noindent\rule[0.25\baselineskip]{\textwidth}{0.5pt}

$\textbf{Exercise5:}$

$S$ is a normed space, and we define $B_{1} = \{ x \in S: \|x\| \leq 1 \}$. Prove or disprove: $B_{1}$ is compact.

\vspace{8pt}
$\textbf{Solution:}$

The $B_{1}$ is not compact, we can find a counter example. We consider $S = l^{2}$ and $B_{1} = \{ x \in l^{2}: \|x\| = 1 \}$. 

Firstly, we can show that $B_{1}$ is bounded and closed. By the definition of $B_{1}$, we know that $B_{1}$ is bounded by 1. $\forall x, y \in B_{1}$, since $\|x\| \leq \|x - y \| + \|y\|$ and $\|y\| \leq \|x - y \| + \|x\|$, we have $| \|x\| - \|y\| | \leq \|x - y\|$, such that the norm is continuous from $l^{2}$ to $\mathbb{R}$. Since the image set $\{1\}$ is closed, then we know the inverse image of $\{1\}$ is also closed, which is actually $B_{1}$. So, $B_{1}$ is bounded and closed.

Next, we verify that $\exists \epsilon > 0$, $B_{1}$ cannot be covered by finitely many balls with radius $\epsilon$. We define $e_{i}$ as follow:
\begin{equation*}
e_{i,m} =
\left\{
             \begin{array}{cl}
             1, & m = i \\
             0, & m \neq i
             \end{array},
\right.
\end{equation*}
such that $e_{i} \in l^{2}$. Clearly, we have $\forall i, j$, if $i \neq j$,  we have  $\|e_{i} - e_{j} \| = \sqrt{2}$. Suppose $B_{1}$ can be covered by the finite balls with radius $\frac{\sqrt{2}}{2}$. Since $\{e_{i}\}_{i = 1}^{+ \infty}$ is infinity, hence at least one of such ball contains at least $e_{j}$ and $e_{k}$ with $j \neq k$. Let $x$ be the center of this ball, then we have $\|e_{j}  - e_{k}\| \leq \|e_{j} - x\| + \|e_{k} - x\| < \frac{\sqrt{2}}{2} + \frac{\sqrt{2}}{2} =  \sqrt{2}$. It is contradict with $\forall k, j$, if $k \neq j$,  we have  $\|e_{i} - e_{j} \| = \sqrt{2}$. Hence $\exists \epsilon > 0$, $B_{1}$ cannot be covered by finitely many balls with radius $\epsilon$. Then we know  that $B_{1}$ is  not compact.





\end{document}

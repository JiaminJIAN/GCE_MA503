%GCE of WPI
%by Jiamin JIAN

\documentclass[12pt,a4paper]{ctexart}
\usepackage{CJK}
\usepackage{lipsum}
\usepackage{amsmath}
\usepackage{geometry}
\usepackage{titlesec}
\usepackage{amssymb}
\usepackage{epsfig}
\usepackage{float}
\usepackage{graphicx}
\usepackage{tabularx}
\usepackage{longtable}
\usepackage{amstext}
\usepackage{blkarray}
\usepackage{amsfonts}
\usepackage{bbm}
\usepackage{listings}
\geometry{left=2.5cm,right=2.5cm,top=2.5cm,bottom=2.5cm}

\begin{document}


\begin{center}
\textbf{Analysis Exercise}
\vspace{8pt}
\end{center}

\vspace{12pt}

$\textbf{Exercise 2:}$

Let $(X, \eta)$ and $(Y, \rho)$ be two Polish spaces. $C(X, Y)$ is the set of all continuous mappings $f: X \mapsto Y$. For $f, g \in C(X, Y)$, we define
\begin{equation*}
    d(f, g) = \sup_{x \in X} \rho(f(x), g(x)).
\end{equation*}

(i) Prove that $(C(X, Y), d)$ is a Polish space.

(ii) If $K \subset Y$ is compact, is $C(X, K)$ compact in $C(X, Y)$?

\vspace{8pt}

$\textbf{Solution:}$

(i) We need to show that $C(X, Y)$ is separable and completely metrizable. Firstly we show that $(C(X, Y), d)$ is a metric space. 
\begin{itemize}
    \item For any $f, g \in C(X, Y)$, if $d(f, g) = 0$, we have $\sup_{x \in X} \rho(f(x), g(x)) = 0$, thus $f(x) = g(x)$ for any $x \in X$. We know that $f \equiv g$. If $f = g$, $d(f, g) = \sup_{x \in X} \rho(f(x), g(x)) = 0$. Hence $d(f, g) = 0 \iff f = g$.
    \item For any $f, g \in C(X, Y)$, $$d(f, g) = \sup_{x \in X} \rho(f(x), g(x)) = \sup_{x \in X} \rho(g(x), f(x)) = d(g, f).$$
    \item For any $f, g, h \in C(X, Y)$, and for any $x \in X$, we have $\rho(f(x), g(x)) \leq \rho(f(x), h(x)) + \rho(h(x), g(x))$. Then we know that $$\rho(f(x), g(x)) \leq \sup_{x \in X} \rho(f(x), h(x)) + \sup_{x \in X} \rho(h(x), g(x)).$$ By the arbitrary of $x \in X$, $d(f, g) = \sup_{x \in X} \rho(f(x), g(x)) \leq d(f, h) + d(h, g)$.
\end{itemize}

Next we show that $C(X, Y)$ is complete. Suppose $\{f_{n}\}$ is a Cauchy sequence in $C(X, Y)$, then $\forall \epsilon > 0, \exists N \in \mathbb{N}, \forall p > q > N$,
\begin{equation*}
    \sup_{x \in X} \rho(f_{p}(x), f_{q} (x) ) < \epsilon.
\end{equation*}
For any $y \in X$, we have
\begin{equation*}
    \rho(f_{p}(y), f_{q} (y)) \leq \sup_{x \in X} \rho(f_{p}(x), f_{q} (x)) < \epsilon,
\end{equation*}
thus $f_{n}(y)$ is a Cauchy sequence in $Y$. As $Y$ is a Polish space, $Y$ is complete, then $f_{n}(y)$ converges to some $f(y)$ in $Y$. From this we can define a function
\begin{equation*}
    f: X \mapsto Y.
\end{equation*}
Next we show that $f$ is also continuous. Since
\begin{equation*}
    \rho(f(x), f(y)) \leq \rho(f(x), f_{n}(x)) + \rho(f_{n}(x) , f_{n}(y)) + \rho(f_{n}(y), f(y)),
\end{equation*}
and $\{f_{n}\}$ is a continuous function sequence, for the above $\epsilon$, there exists a $N^{*} \in \mathbb{N}$ and $\delta > 0$, for any $x \in B(y, \delta)$ and $n > N^{*}$, we have
\begin{equation*}
    \rho(f(x), f(y)) < 3 \epsilon.
\end{equation*}
Hence $f \in C(X, Y)$. And for the above $\epsilon$ and $p > q > N$, since $\rho(f_{p}(y), f_{q} (y)) < \epsilon$, let $p \to \infty$, we have $\rho(f(y), f_{q}(y)) \leq \epsilon$. By the arbitrary of $y \in X$, for $q > N$, we can get
\begin{equation*}
    \sup_{y \in X} \rho(f(y), f_{q}(y)) \leq \epsilon,
\end{equation*}
which shows that $f_{n} \to f$ in $C(X, Y)$. Thus $C(X, Y)$ is complete.

Next we need to show that $C(X, Y)$ is separable. As $X$ and $Y$ are polish space, we can take a countable dense set $\hat{X}$, $\hat{Y}$ from $X, Y$. And we define $M(\hat X, \hat Y)$ be the set of all mappings $\hat X \mapsto \hat Y$ with the metric
\begin{equation*}
    \hat d (f, g) = \sup_{x \in \hat X} \rho(f(x), g(x)), \quad \forall f, g \in M(\hat X, \hat Y).
\end{equation*}
For $f \in C(X, Y)$, we define
\begin{equation*}
    B(f, \delta) = \{g \in C(X, Y): d(f, g) < \delta \},
\end{equation*}
where $\delta > 0$ is a constant. To prove that $C(X, Y)$ is separable, we need to show that: $\forall \epsilon > 0$, there exists a sequence of functions $\{f_{n}\}_{n = 1}^{\infty}$ such that
\begin{equation*}
    \bigcup_{n = 1}^{\infty} B(f_{n}, \epsilon) = C(X, Y).
\end{equation*}
Define the projector $P: M(X, Y) \mapsto M(\hat X, Y)$ as
\begin{equation*}
    P f(x) = f(x), \quad \forall f \in M(X, Y), \forall x \in \hat X,
\end{equation*}
where $M(X, Y)$ is the set of all mappings from $X$ to $Y$. Since $\hat X$ is countable, by the Axiom of Choice, $\forall \epsilon > 0$, $\forall f \in C(X, Y)$, there exists $h \in M(\hat X, \hat Y)$ such that
\begin{equation*}
    \hat{d} (Pf, h) < \frac{\epsilon}{3}.
\end{equation*}
As $M(\hat X, \hat Y)$ is countable, we index it as $M(\hat X, \hat Y) = \{h_{n}: n \in \mathbb{N}\}$. For the above $\epsilon$, setting
\begin{equation*}
    A_{n} = \{f \in C(X, Y): \hat d (Pf, h_{n}) < \frac{\epsilon}{3}\},
\end{equation*}
and we have $\bigcup_{n = 1}^{\infty} A_{n} = C(X, Y)$. By the Axiom of Choice, for any $n \in \mathbb{N}$, we can take a $f_{n} \in A_{n}$. Now we can prove that for the above $\epsilon$, $$\bigcup_{n = 1}^{\infty} B(f_{n}, \epsilon) = C(X, Y).$$
By the definition of $B(f_{n}, \epsilon)$, we have
\begin{equation*}
    \bigcup_{n = 1}^{\infty} B(f_{n}, \epsilon) \subset C(X, Y).
\end{equation*}
And for any $f \in C(X, Y)$, as $C(X, Y) = \bigcup_{n = 1}^{\infty} A_{n}$, there exists $k \in \mathbb{N}$ such that $f \in A_{k}$, then we have
\begin{equation*}
    d(f, f_{k}) \leq \hat d(Pf, h_{k}) + \hat d (h_{k}, P f_{k}) < \frac{2 \epsilon}{3} < \epsilon.
\end{equation*}
Thus we also have $C(X, Y) \subset \bigcup_{n = 1}^{\infty} B(f_{n}, \epsilon)$. Then we know that $C(X, Y)$ is separable.
\vspace{4pt}

(ii) The statement is not true. We can give a counter example as follows. Set $K = [0, 1],Y = \mathbb{R}$ and $X = [0, 1]$, and we define a function sequence $f_{n}: X \mapsto K$ by
\begin{equation*}
f_{n}(x) =
\left\{
             \begin{array}{cl}
             0, & x \in [0, \frac{1}{2} - \frac{1}{n}) \\
             n x - \frac{n}{2} + 1, & x \in [\frac{1}{2} - \frac{1}{n}, \frac{1}{2}) \\
             1, & x \in [\frac{1}{2}, 1]
             \end{array}
\right.
\end{equation*}
then we know that $K \subset Y$ and $K$ is compact and $\{f_{n}\}$ is a continuous function sequence from $X$ to $K$. And we define
\begin{equation*}
f (x) =
\left\{
             \begin{array}{cl}
             0, & x \in [0, \frac{1}{2}) \\
             1, & x \in [\frac{1}{2}, 1]
             \end{array}
\right.
\end{equation*}
thus when $n \to \infty$, $f_{n}(x)$ converges to $f(x)$ almost everywhere. But $f(x)$ is not a continuous function on $X$, $f(x) \notin C(X, K)$, thus for any subsequence $\{f_{n_{k}}\}$ of $\{f_{n}\}$, we know that $\{f_{n_{k}}\}$ is not converges in $C(X, K)$. Hence we know $C(X,K)$ is not compact in $C(X, Y)$.
\end{document}

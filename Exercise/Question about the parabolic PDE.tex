%GCE of WPI
%by Jiamin JIAN

\documentclass[12pt,a4paper]{ctexart}
\usepackage{CJK}
\usepackage{lipsum}
\usepackage{amsmath}
\usepackage{geometry}
\usepackage{titlesec}
\usepackage{amssymb}
\usepackage{epsfig}
\usepackage{float}
\usepackage{graphicx}
\usepackage{tabularx}
\usepackage{longtable}
\usepackage{amstext}
\usepackage{blkarray}
\usepackage{amsfonts}
\usepackage{bbm}
\usepackage{listings}
\geometry{left=2.5cm,right=2.5cm,top=2.5cm,bottom=2.5cm}

\begin{document}


\begin{center}
\textbf{A question about the parabolic PDE}
\vspace{8pt}
\end{center}

\vspace{12pt}

$\textbf{Question:}$

Assume that $b, l \in C_{b}^{1,2} (\mathbb{R}^{+}, \mathbb{R})$, consider the parabolic PDE
\begin{equation} \label{P1}
(P1) \quad
    \begin{cases}
    \partial_{t} v = b \partial_{x} v + \partial_{xx} v + l, \quad \forall (t, x) \in (\mathbb{R}^{+}, \mathbb{R}) \\
    v(0, x) = 0, \quad \forall x \in \mathbb{R}
    \end{cases}
\end{equation}
It is standard that there is a classical solution. Assume that $\partial_{xxx} v$ exists, by taking derivative to $x$ to the equation (\ref{P1}) on the both side, with $\partial_{x} v = \hat v$, we have
\begin{equation*}
    \begin{cases}
    \partial_{t} \hat v = b \partial_{x} \hat v + \partial_{xx} \hat v + \partial_{x} l + \hat v \cdot \partial_{x} b, \quad \forall (t, x) \in (\mathbb{R}^{+}, \mathbb{R}) \\
    \hat v(0, x) = 0, \quad \forall x \in \mathbb{R}
    \end{cases}
\end{equation*}
Therefore, by the uniqueness of the solution, we conclude that the solution of
\begin{equation} \label{P2}
(P2) \quad
    \begin{cases}
    \partial_{t} \hat u = b \partial_{x} \hat u + \partial_{xx} \hat u + \partial_{x} l + \hat u \cdot \partial_{x} b, \quad \forall (t, x) \in (\mathbb{R}^{+}, \mathbb{R}) \\
    \hat u (0, x) = 0, \quad \forall x \in \mathbb{R}
    \end{cases}
\end{equation}
satisfies $\hat u (t, x) = \partial_{x} v(t, x)$ for any $(t, x) \in (\mathbb{R}^{+}, \mathbb{R})$.

Does the above conclusion holds without assuming $v \in C^{1, 3}$? That is let $b, l \in C_{b}^{1,3} (\mathbb{R}^{+}, \mathbb{R})$, does $\hat u$ of (\ref{P2}) and $v$ of (\ref{P1}) satisfies
\begin{equation*}
    \hat u (t, x) = \partial_{x} v(t, x).
\end{equation*}

\vspace{8pt}

$\textbf{Solution:}$

We define
\begin{equation*}
    u(t, x) = g(t) + \int_{0}^{x} \hat u (t, y) \, d y, \quad \forall (t, x) \in (\mathbb{R}^{+}, \mathbb{R})
\end{equation*}
where $g(\cdot)$ is the function we want to find to make $u(t, x)$ is the solution of equation (\ref{P1}). Suppose $u(t, x)$ is the solution of equation (\ref{P1}), for the initial condition, we need
\begin{equation*}
    u(0, x) = g(0) + \int_{0}^{x} \hat u (0, y) \, d y = g(0) = 0.
\end{equation*}
And for $(t, x) \in (\mathbb{R}^{+}, \mathbb{R})$, we have
\begin{eqnarray*}
\partial_{t} u & = & g^{'} (t) + \int_{0}^{x} \partial_{t} \hat u(t, y) \, d y \\
& = & g^{'} (t) + \int_{0}^{x} (b \partial_{x} \hat u + \partial_{xx} \hat u + \partial_{x} l + \hat u \cdot \partial_{x} b ) (t, y) \, d y  \\
& = & g^{'} (t) + (b \hat u + \partial_{x} \hat u + l) |_{0}^{x}   \\
& = & g^{'} (t) + b \hat u + \partial_{x} \hat u + l - (b \hat u + \partial_{x} \hat u + l)(t, 0).
\end{eqnarray*}
By the definition of $u(t, x)$, we can get that
\begin{equation*}
    \partial_{x} u (t, x) = \hat u (t, x), \quad \forall (t, x) \in (\mathbb{R}^{+}, \mathbb{R}),
\end{equation*}
then we have
\begin{equation*}
    \partial_{t} u - (b \partial_{x} u + \partial_{xx} u + l) = g^{'}(t) - (b \hat u + \partial_{x} \hat u + l)(t, 0).
\end{equation*}
Thus the sufficient condition of $u(t, x)$ is the solution of equation (\ref{P1}) is
\begin{equation} \label{C1}
(C1) \quad
    \begin{cases}
    g^{'}(t) = (b \hat u + \partial_{x} \hat u + l)(t, 0) \\
    g(0) = 0
    \end{cases}
\end{equation}
where $\hat u(t, x)$ is the solution of equation (\ref{P2}). We define
\begin{equation}
    g(t) = \int_{0}^{t} (b \hat u + \partial_{x} \hat u + l)(s, 0) \, d s, \quad \forall t \in \mathbb{R}^{+},
\end{equation}
which satisfies the formula (\ref{C1}). Thus
\begin{equation*}
    u(t, x) = \int_{0}^{t} (b \hat u + \partial_{x} \hat u + l)(s, 0) \, d s + \int_{0}^{x} \hat u (t, y) \, d y, \quad \forall (t, x) \in (\mathbb{R}^{+}, \mathbb{R})
\end{equation*}
is the solution of (\ref{P1}) and it satisfies
\begin{equation*}
    \partial_{x} u (t, x) = \hat u (t, x), \quad \forall (t, x) \in (\mathbb{R}^{+}, \mathbb{R}).
\end{equation*}
What's more, we have $u(t, x) \in C_{b}^{1,3} (\mathbb{R}^{+}, \mathbb{R})$ and it is the unique solution of (\ref{P1}).


\end{document}
